\lecture[Universelle Eigenschaft unendlicher Produkte. Satz von Tychonoff. Abschluss, Dichtheit. Einbettungen. Kompaktifizierung. Vollständige Regularität. Universelle Eigenschaft der Stone-Čech-Kompaktifizierung. Fortsetzung stetiger Funktionen.]{Di 04 Mai 2021 12:12}{Kompaktifizierung}

\begin{theorem}[Universelle Eigenschaft des Produkts]\label{thm:universelle-eigenschaft-des-produkts}
    Seien $(X_i)_{i\in I}$ topologische Räume, $A$ ein topologischer Raum und seien $f_i : A \to  X_i$ Funktionen. Sei
        \begin{equation*}
        f: \left| \begin{array}{c c l} 
        A & \longrightarrow & \prod_{i \in I} X_i \\
        a & \longmapsto &  (f_i(a))_{i \in I}
        \end{array} \right.
    \end{equation*}
    Dann ist $f$ stetig genau dann, wenn alle  $f_i$ stetig sind.
    \[
    \begin{tikzcd}
        & & X_1 \\
        & & X_2 \\
    A \ar[dashed]{r}{f} \ar[bend left = 40]{uurr}{f_1} \ar[bend left = 20]{urr}{f_2} \ar[bend right = 20,swap]{drr}{f_i} \ar[dashed, bend right = 40]{ddrr}& \prod_{i \in I}X_i \ar[near end]{uru}{\pr_1} \ar[swap]{ur}{\pr_2} \ar{dr}{\pr_i} \ar[bend right = 20,dashed]{ddr} \ar[dashed]{r}& \vdots \\
          & & X_i \\
          & & \vdots
    \end{tikzcd}
    .\] 
\end{theorem}
\begin{remark*}
    Die Universelle Eigenschaft ist genau genommen die Folgende: \\
    Seine $(X_i)_{i \in I}$ topologische Räume. Ein topologischer Raum $X$  zusammen mit Abbildung  $\pr_i : X \to  X_i$ wird Produkt der $X_i$ genannt, wenn es für jedes  $A$ und stetige Abbildungen  $f_i : A \to  X_i$ genau eine induzierte Abbildung $f: A \to  X$ gibt. \\
    Diese Eigenschaft ist nun universell im Sinne der Kategorientheorie, d.h. bis auf eindeutig bestimmten Isomorphismus gibt es nur ein Paar $(X, (\pr_i)_{i \in I})$, das die oben genannten Eigenschaften bestimmt. \\
    Wir haben zwar oben nicht die Eindeutigkeit des Produkts gezeigt, aber dessen Existenz (was aus der Kategorientheorie nicht ohne weiteres folgt), indem wir ein explizites solches Objekt konstruiert haben.
\end{remark*}
\begin{remark*}
    Insbesondere sollte man sich merken, dass die kanonischen Projektionen $\pr_i$ wichtiger Teil der Information eines Produktes sind. Bei unsere expliziten Konstruktion 'kanonisch', denkbar ist jedoch auch, eine völlig andere Trägermenge des Produkts zu wählen, dann ist die Angabe der Projektionen essentiell.
\end{remark*}
\begin{proof}
    '$\implies$' Sei $j\in I$, setze
        \begin{equation*}
        pr_j: \left| \begin{array}{c c l} 
        \prod_{i\in I}  & \longrightarrow & X_j \\
        (x_i)_{i \in I} & \longmapsto &  x_j
        \end{array} \right.
    \end{equation*}
    als Projektion auf die $j$-te Komponente.
     \begin{claim}
        $\pr_j$ ist stetig
    \end{claim}
    \begin{subproof}
        Ist $U\subset X_j$ offen, dann ist $pr_j^{-1}(U) = U\times \prod_{i\neq j} X_i\in \mathcal{S}$ ein Element der Subbasis der Produkttopologie, also offen. Also ist $pr_j$ stetig.
    \end{subproof}
    Nun ist $f_j = pr_j \circ  f$ stetig als Verknüpfung stetiger Funktionen.
    \begin{recap}
        \emphasize{Verknüpfungen stetiger Funktionen sind stetig:}\\
        Seien $f:X\to Y$ und $g:Y\to Z$ stetig, dann ist $g\circ  f : X \to  Z$ stetig.
        \begin{proof}
            Ist $U\subset Z$ offen, so ist
            \[
                (g \circ  f) ^{-1}(U) = f^{-1}(g^{-1}(U)) \subset X
            .\] 
            offen, indem wir zunächst $g$ stetig und dann  $f$ stetig verwenden.
        \end{proof}
    \end{recap}
    '$\impliedby$' Es genügt zu zeigen, dass $f^{-1}(Y)\subset A$ offen ist für alle $Y\in \mathcal{S}$. Sei also solch ein $Y\in \mathcal{S}$ beliebig, dann ist dieses von der Form
    \[
    Y = U\times \prod_{i\neq j} X_i
    .\] 
    Dann ist $f^{-1}(Y) = f^{-1}_j(A)\subset A$ offen, da $f_j$ stetig ist.
\end{proof}
\begin{theorem}[Satz von Tychonoff]\label{thm:tychonoff}
    Sei $(X_i)_{i \in I}$ eine Familie kompakter Räume. Dann ist $\prod _{i \in I} X_i$ kompakt.
\end{theorem}
\begin{proof}
    Wir verwenden wieder den \nameref{thm:alexander} (\autoref{thm:alexander}). Sei $\mathcal{U}$ eine Überdeckung durch Elemente aus  $\mathcal{S}$. Sei $\mathcal{U}_j \subset \mathcal{U}$ gegeben durch die Elemente $V$ von  $\mathcal{U}$ der Form
    \[
    V = W \times \prod_{i\neq j} X_i \qquad \text{mit } W\subset X_j \text{ offen}
    .\] 
    Dann ist
    \[
    \mathcal{U} = \bigsqcup_{i \in  I} \mathcal{U}_j
    .\] 
    Ist nun
    \[
        \pr_i (\mathcal{U}_i)  = \left \{\pr_i(V) \mid  V\in \mathcal{U}_i\right\} 
    .\] 
    eine offene Überdeckung von $X_i$, so existiert - weil  $X_i$ kompakt - eine endliche Teilüberdeckung  $\pr_i(V_1)\cup \ldots\cup \pr_i(V_k)$ von $X_i$ mit  $V_j \in \mathcal{U}_i$. Dann ist $V_1,\ldots,V_k$ eine endliche Teilüberedckung von $\prod_{i \in I} X_i$. \\
    Wir sind also fertig, außer im Fall \\
    $\mathbb{A}$: $\pr_i(\mathcal{U}_i)$ ist  \underline{keine} Überdeckung von $X_i$ für alle  $i\in I$.  \\
    Dann finden wir $x_i \in X_i \setminus \bigcup_{V\in \mathcal{U}_i} \pr_i(V)$ für jedes $i\in I$. Dann ist aber der Punkt
    \[
        (x_i)_{i \in I}\in \prod_{i \in I}X_i
    .\] 
    nicht von $\mathcal{U}$ überdeckt: Ist $(x_i)_{i \in I} \in V\in \mathcal{U}$, dann gibt es $i\in I$ mit $V\in \mathcal{U}_i$, und daraus folgt bereits $x_i \in \pr_i(V)$, \contra.
\end{proof}
\begin{remark}
    Eigentlich haben wir die Notation $\pr_j$ für die Projektion  $\prod _{i \in I}X_i \to  X_j$ eingeführt, manchmal schreiben wir aber auch einfach nur $p_j$.
\end{remark}
\begin{example}
    \begin{enumerate}[a)]
        \item Seien $X_1,\ldots,X_n$ diskrete Räume. Dann ist auch $\prod_{i \in I}X_i$ diskret.
            \begin{proof}
                Es ist
                \[
                    \left \{(x_1,\ldots,x_n)\right\}  = \left \{x_1\right\} \times \ldots\times \left \{x_n\right\} 
                .\] 
                Element der Produkttopologie, weil die $\left \{x_i\right\} \subset X_i$ offen sind. Also sind alle Punkte offen.
            \end{proof}
        \item Betrachte $\left \{0,2\right\} $ mit der diskreten Topologie. Dann ist
            \[
            \prod_{\N} \left \{0,2\right\} =: \left \{0,2\right\} ^{\N}
            .\] 
            kompakt nach dem \nameref{thm:tychonoff}. Dann ist $\prod_{\N} \left \{0,2\right\} $ aber nicht diskret, weil wir sonst die offene Überdeckung
            \[
            \prod_{\N} \left \{0,2\right\} = \bigcup_{x\in \left \{0,2\right\} ^{\N}}  \left \{x\right\} 
            .\] 
            hätten, die keine endliche Teilüberdeckung besitzt.
    \end{enumerate}
\end{example}
\begin{remark*}
    Das Beispiel zeigt die wichtige Eigenschaft, dass \underline{nicht} (notwendigerweise) alle Mengen der Form $\prod_{i \in I}U_i$ für $U_i\subset X_i$ offen auch im Produkt $\prod_{i \in I}X_i$ offen sind.
\end{remark*}

\begin{theorem}\label{thm:produkte-von-Hausdorff-Räumen-sind-Hausdorff}
    Ist $\left \{X_i\right\} _{i \in I}$ eine Familie von Hausdorffräumen, so ist auch  $\prod _{i \in I} X_i$ Hausdorffsch.
\end{theorem}
\begin{proof}
    Ist $(x_i)_{i \in I} \neq  (y_i)_{i \in I} \in  \prod _{i \in I}X_i$, dann gibt es $i\in I$ mit $x_i \neq  y_i$. Da $X_i$ Hausdorffsch ist, existieren  $U_i, V_i \subset X_i$ offen mit $x_i \in U_i, y_i \in V_i$ und $U_i \cap  V_i = \emptyset$. Dann sind aber beretis
    \[
    U_i \times  \prod_{i\neq j} X_j \qquad V_i \times \prod_{i\neq j} X_j
    .\] 
    zwei disjunkte, offene Umgebungen von $(x_i)_{i \in I}$ und $(y_i)_{i \in I}$.
\end{proof}

\begin{goal*}
    Wir wollen uns im Folgenden Fragen, wann wir Räume in 'schöne' Räume einbetten können, wobei 'schön' für uns Kompakt + Hausdorff heißen soll.
\end{goal*}

\begin{definition}[Abschluss, Dichtheit] \label{def:abschluss-dichtheit}
    Sei $X$ ein topologischer Raum und $Y\subset X$ eine Teilmenge.
    \begin{enumerate}[1)]
        \item Der \vocab[Menge!Abschluss]{Abschluss} $\overline{Y}$ ist definiert als
            \[
            \overline{Y} := \bigcap_{\substack{Y\subset A \\ A\subset X \text{ abg.}} } A
            .\] 
            Als Schnitt abgeschlossener Mengen ist $\overline{Y}$ selbst abgeschlossen (wie der Name suggeriert).
        \item  $Y$ ist \vocab[Menge!dicht]{dicht} in $X$, falls  $\overline{Y} = X$.
    \end{enumerate}
\end{definition}
\begin{definition}[Einbettung]\label{def:einbettung}
    Sei $f:X\to Y$ stetig. Dann ist $f$ eine  \vocab{Einbettung}, falls $f: X \to  f(X)$ ein Homöomorphismus ist.
\end{definition}
\begin{definition}[Kompaktifizierung] \label{def:kompaktifizierung}
    Sei $ι: Y\hookrightarrow X$ eine Einbettung. Dann ist $X$ eine \vocab{Kompaktifizierung} von $Y$, falls
    \begin{enumerate}[1)]
        \item $X$ ist kompakt und Hausdorffsch.
        \item  $ι(Y)\subset X$ ist dicht (in $X$).
    \end{enumerate}
\end{definition}
\begin{definition}[Vollständige Regularität]\label{def:vollständig-regulär}
    Ein topologischer Raum $X$ ist  \vocab[Topologischer Raum!vollständig regulär]{vollständig regulär}, falls
    \begin{enumerate}[1)]
        \item $X$ ist Hausdorffsch
        \item  $\forall A\subset X$ abgeschlossen und $x\in X\setminus A$ existiert eine stetige Abbildung $f:X \to  [0,1]$, sodass $f(x) = 1$ und  $f\mid _A \equiv 0$
    \end{enumerate}
\end{definition}
\begin{remark}
    Jeder vollständig reguläre Raum ist regulär. Hierzu betrachte $f^{-1}\left( \left( \frac{1}{2},1 \right]  \right) $ sowie $f^{-1}\left( \left[ 0,\frac{1}{2}  \right)  \right) $. Diese sind offenbar disjunkt, offen, und Umgebungen von $x$ bzw.  $A$.
\end{remark}
\begin{lemma}\label{lm:teilraum-von-vollständig-regulärem-raum-ist-vollständig-regulär}
    Ist $X$ vollständig regulär und  $Y\subset X$, dann ist auch $Y$ vollständig regulär.
\end{lemma}
\begin{proof}
    \begin{enumerate}[1)]
        \item Da $X$ Hausdorffsch ist, ist auch  $Y$ Hausdorffsch.
        \item Sei  $A\subset Y$ abgeschlossen und $y\in Y \setminus A$. Dann existiert $A'\subset X$ abgeschlossen mit $A' \cap Y = A$. Da $X$ vollständig regulär ist,  gibt es  $f: X \to  [0,1]$ stetig mit $f\mid _{A'} \equiv  O$ und $f(y) = 1$. Dann erfüllt  $f\mid _Y : Y \to  [0,1]$ unsere gewünschten Bedingungen, weil
            \[
                \left(            f\mid _Y \right) \mid _A \equiv O \qquad f\mid _Y(y) = 1
            .\] 
    \end{enumerate}
\end{proof}
\begin{theorem}\label{thm:vollständig-regulär-wenn-kompaktifizierbar}
    $X$ ist genau dann vollständig regulär, wenn  $X$ eine Kompaktifizierung besitzt.
\end{theorem}
\begin{proof}
    Eine Richtung sei hier schon skizziert: Sei $Y$ eine Kompaktifizierung von  $X$. Da $Y$ kompakt und Hausdorffsch, ist  $Y$ normal (nach \autoref{thm:kompakter-hausdorff-raum-ist-normal}). Wir zeigen später, dass dann  $Y$ auch vollständig regulär ist. Mit \autoref{lm:teilraum-von-vollständig-regulärem-raum-ist-vollständig-regulär} ist also auch $X\subset Y$ vollständig regulär.
\end{proof}
\begin{remark*}
    Hier verwenden wir entscheidend, dass wir nicht nur $X\hookrightarrow Y$ injektiv abgebildet, sondern eingebettet im Sinne von \autoref{def:einbettung} haben, damit wir $X$ auch homöomorph mit einem Teilraum $X\subset Y$ identifizieren können.
\end{remark*}
Wir wollen nun zu einem beliebigen Raum eine Kompaktifizierung konstruieren.
Sei $X$ ein topologischer Raum. Sei 
\[
    \mathcal{C}(X) := \left \{f: X \to  [0,1] \mid  f \text{ stetig}\right\} 
.\] 
Nach dem \nameref{thm:tychonoff} ist $\prod_{\mathcal{C}(X)}[0,1] $ kompakt und nach \autoref{thm:produkte-von-Hausdorff-Räumen-sind-Hausdorff} Hausdorffsch. Definiere nun eine Abbildung
\[
    ι : X \to  \prod_{\mathcal{C}(X)} [0,1]
.\] 
durch die Komponenten $ι_f(x) = f(x)$. (wir benutzen also in der  $f$-ten Komponente einfach die Abbildung  $f$). Da alle  $f\in \mathcal{C}(X)$ stetig sind, ist $ι$ stetig (nach \autoref{thm:universelle-eigenschaft-des-produkts}). Setze nun
\[
    βX := \overline{ι(X)} \subset \prod_{\mathcal{C}(X)} [0,1]
.\] 
$β(X)$ ist kompakt und Hausdorffsch als abgeschlossener Teilraum eines kompakten Hausdorffraums. \\
\begin{dtheoremdef}[Stone-Čech-Kompaktifizierung]
    Für einen topologischen Raum $X$ heißt der eben konstruierte Raum $βX = β(X)$ \vocab{Stone-Čech-Kompaktifizierung} von $X$. $\beta X$ ist ein kompakter Hausdorffraum.
\end{dtheoremdef}
\begin{proof*}
    Klar nach eben gesagtem, wir verwenden \nameref{thm:tychonoff} und \autoref{thm:produkte-von-Hausdorff-Räumen-sind-Hausdorff}.
\end{proof*}
\begin{warning}
Diese ist jedoch nur eine Kompaktifizierung im Sinne von \autoref{def:kompaktifizierung} falls $X$ vollständig regulär ist.
\end{warning}
\begin{remark*}
    Wir wissen schon, dass es sich im Allgemeinen nicht um eine Kompaktifizierung nach \autoref{def:kompaktifizierung} handeln kann, weil wir im Beweis von \autoref{thm:vollständig-regulär-wenn-kompaktifizierbar} gezeigt haben, dass eine Kompaktifizierung nur für vollständig reguläre Räume existieren kann. Der folgende Satz zeigt nun, dass es sich bei der Stonen-Čech-Kompaktifizierung tatsächlich um eine handelt, wenn $X$ vollständig regulär ist:
\end{remark*}
\begin{theorem}\label{thm:stone-cech-kompaktifizierung-ist-einbettung-für-vollständig-reguläre-räume}
     $ι: X \to  \prod_{\mathcal{C}(X)}[0,1]$ ist eine Einbettung, falls $X$ vollständig regulär ist.
\end{theorem}
\begin{proof}
    \textbf{Injektivität}: Seien $x\neq y\in X$. Dann sind $\left \{x\right\} ,\left \{y\right\} \subset X$ abgeschlossen und es existiert $f: X \to  [0,1]$ mit $f(x) = 0$ und  $f(y) = 1$ (hier benutzen wir die vollständige Regularität). Dann ist aber bereits  $ι(x) \neq  ι(y)$ in Komponenten $f$. \\
    \textbf{Einbettung}: Wir müssen noch zeigen, dass $\forall  U\subset X$ offen $ι(U) \subset ι(X)$ offen ist, damit $ι : X \to  f(X)$ ein Homöomorphismus ist. \\
    Sei $U\subset X$ offen, setze $A := X\setminus U$ und sei $x\in U$. Dann finden wir (nach vollständiger Regularität von $X$) eine Funktion  $f: X \to  [0,1]$, sodass $f(x) = 1$ und  $f\mid _{A} = 0$. Setze
    \[
        V := \left( \frac{1}{2},1 \right]_f \times \prod_{\mathcal{C}(X) \setminus \left \{f\right\} } [0,1] \subset \prod_{\mathcal{C}(X)} [0,1]
    .\] 
    als offene Teilmenge von $\prod_{\mathcal{C}(X)} [0,1]$. Dann ist
    \[
        ι(x) \in \underbrace{V \cap  ι(X)}_{\text{offen in } ι(X)} \subset ι(X \setminus A) = ι(U)
    .\] 
    Damit ist $ι(U)\subset X$ Umgebung all seiner Punkte, also selbst offen.
\end{proof}
\begin{remark}
    Ist $K$ kompakt und Hausdorffsch, so ist  $ι(K) \subset \prod_{\mathcal{C}(K)} [0,1]$ kompakt, also abgeschlossen, da $\prod_{\mathcal{C}(K)} [0,1]$ kompakt, und deswegen ist $β(K) = \overline{ι(K)} = ι(K) \cong K$.
\end{remark}
\begin{remark*}
    Dass $ι(K) \cong K$ folgt in vorheriger Bemerkung daraus, dass wir wegen  $K$ kompakt und Hausdorffsch nach \autoref{thm:kompakter-hausdorff-raum-ist-normal} wissen, dass $K$ normal ist, und dann (mit der noch nicht bewiesenene Implikation normal  $\implies$ vollständig regulär) den vorherigen \autoref{thm:stone-cech-kompaktifizierung-ist-einbettung-für-vollständig-reguläre-räume} anwenden können, weswegen $ι$ eine Einbettung ist und somit einen Homöomorphismus  $K \cong ι(K)$ induziert.
\end{remark*}
\begin{proof*}[Beweis von \autoref{thm:vollständig-regulär-wenn-kompaktifizierbar}]
    Wir haben bereits gesehen, dass ein kompaktifizierbarer Raum notwendigerweise vollständig regulär ist (im ersten Teil des Beweises). Ist $X$ nun vollständig regulär, so ist  $β(X)$ ein kompakter Hausdorff-Raum, und nach \autoref{thm:stone-cech-kompaktifizierung-ist-einbettung-für-vollständig-reguläre-räume} handelt es sich bei $ι_X : X \to  β(X)$ genau um eine Einbettung.
\end{proof*}
\begin{lemma}[Fortsetzung stetiger Funktionen]\label{lm:fortsetzung-stetiger-funktionen-in-dichten-hausdorff-räumen-sind-eindeutig}
   Sei $f: X \to  Y$ stetig sowie $U\subset X$.
   \begin{enumerate}[1)]
       \item Dann ist $f(\overline{U}) \subset \overline{f(U)}$ 
       \item Ist $U\subset X$ dicht, $g: X\to Y$ auch stetig und $f\mid _ U = g\mid _U$ sowie $Y$ Hausdorffsch, so ist beretis  $f=g$
   \end{enumerate}
\end{lemma}
\begin{proof*}
    \begin{enumerate}[1)]
        \item Sei $y\in f(\overline{U})$, also gibt es $x\in \overline{U}$ mit $f(x) = y$. Sei  $y\in V\subset X$ eine beliebige offene Umgebung von $y$. Dann ist $f^{-1}(V)$ eine offene Umgebung von $x$ nach Stetigkeit von  $f$. Da  $x\in \overline{U}$ ist $f^{-1}(V)\cap U \neq \emptyset$ und wir wählen $x_0\in f^{-1}(V)\cap U$. Dann ist $f(x_0) \in V \cap f(U)$ und somit $V \cap f(U) \neq  \emptyset$. Da $V$ beliebig war, ist nach Definition  $y\in \overline{f(U)}$.
        \item Nimm an, dass $f\neq g$, dann gibt es $x\in X$ mit $f(x) != g(x)$. Da  $Y$ Hausdorffsch, können wir die beiden Punkte durch offene Mengen trennen, also finden wir  $f(x)\in U_f, g(x)\in U_g$ mit $U_f\cap U_g=\emptyset$ und $U_f,U_g$ offen. Dann sind auch $f^{-1}(U_f),g^{-1}(U_g)$ offene Mengen nach Stetigkeit von $f,g$, also ist auch  $f^{-1}(U_f)\cap g^{-1}(U_g)$ offen. Zudem $x\in f^{-1}(U_f)\cap g^{-1}(U_g)$, da $f(x)\in U_f,g(x)\in U_g$ nach Voraussetzung. Da $U\subset X$ dicht ist, ist $U\cap (f^{-1}(U_f)\cap g^{-1}(U_g))\neq \emptyset$ und wir finden $x_0\in U \cap  f^{-1}(U_f) \cap g^{-1}(U_g)$. Dann ist wegen $f\mid _{U}\equiv g\mid _{U}$ $f(x_0) = g(x_0)$, aber auch $f(x_0)\in U_f, g(x_0)\in U_g$, also $f(x_0) = g(x_0) \in U_f \in  U_g$. Aber nach Voraussetzung ist $U_f \cap  U_g = \emptyset$, \contra. Also $f\equiv g$.
\end{enumerate}
\end{proof*}
\begin{remark*}
    Der Beweis von \autoref{lm:fortsetzung-stetiger-funktionen-in-dichten-hausdorff-räumen-sind-eindeutig} war eine Übungsaufgabe auf Blatt 4.
\end{remark*}
\begin{theorem}[Universelle Eigenschaft von $β$]\label{thm:universelle-eigenschaft-der-stone-cech-kompaktifizierung}
    Sei $f: X \to  K$ stetig, $K$ kompakt und Hausdorffsch. Dann existiert eine eindeutige Fortsetzung $\hat{f}: β(X) \to  K$, so dass
    \[
   \begin{tikzcd}
       X \ar{r}{ι} \ar[swap]{d}{f} &  β(X) \ar[dashed]{dl}{\hat{f}} \\
       K
   \end{tikzcd} 
    .\]
    kommutiert.
\end{theorem}
\begin{recap}
    Ist $f(X)\subset K$ dicht, so ist $\hat{f}$ surjektiv: Es ist $\hat{f}(β(X))$ kompakt, also abgeschlossen und enthält $f(X)$ (weil das Diagramm kommutiert), und daraus folgt  $\overline{f(X)}\subset \hat{f}(β(X))$.
\end{recap}

\begin{proof}
    Die Eindeutigkeit von $\hat{f}$ folgt direkt aus \autoref{lm:fortsetzung-stetiger-funktionen-in-dichten-hausdorff-räumen-sind-eindeutig}, weil $\hat{f}$ über die Kommutativität des Diagramms auf der dichten Teilmenge $ι(X)\subset β(X)$ bereits eindeutig bestimmt ist. \\
\begin{idea}
Ist $K = [0,1]$, so wähle  $\hat{f} = \pr_f \mid _{β(X)}$ als stetige Projektion. Dann kommutiert nämlich
\[
\begin{tikzcd}
    X \ar{r}{ι} \ar[swap]{dr}{f}& \prod_{\mathcal{C}(X)} \left[0,1\right] \ar[dashed]{d}{\pr_f} \\
                & \left[0,1\right]
\end{tikzcd}
.\]
nach Konstruktion von $ι$.
\end{idea}
Das ganze können wir nun zwar nicht direkt für $K$ machen, allerdings für jedes  $g\in \mathcal{C}(K)$. Für jedes $g\in \mathcal{C}(K)$ erhalten wir durch Komposition $g \circ f \in \mathcal{C}(X)$ und damit nach vorheriger Überlegung eine Abbildung $\pr_{g \circ  f}\mid _{β(X)} \colon β(X) \to [0,1]$. Verwenden wir diese als Komponentenabbildung nach $\prod_{\mathcal{C}(K)}[0,1]$, so induzieren wir eine Abbildung $\hat{f} = \prod \pr_{g\circ f}\mid _{β(X)}$:
\[
    \begin{tikzcd}
        X \ar[hook]{rr}{ι_X} \ar[swap]{d}{f} & &  β(X) \ar[phantom]{r}{\subset } \ar[bend right = 20,swap, near start]{ddl}{\pr_{g_i \circ  f}} \ar[bend left = 20, near end]{ddr}{\pr_{g_j \circ  f}} \ar[red,dashed, "\hat{f}" description]{d}& \prod_{\mathcal{C}(K)}[0,1] \\
        K \ar[hook]{rr}{ι_K}\ar[swap]{dr}{g_i} \ar[near end,swap]{drrr}{g_j} & & \prod_{\mathcal{C}(K)}[0,1] \ar[near end]{dl}{\pr_{g_i}} \ar[near start]{dr}{\pr_{g_j}}\\
                             & \left[0,1\right] & \ldots & \left[0,1\right]
    \end{tikzcd}
\]
Das linke obere Quadrat kommutiert auch: Hierzu müssen wir überprüfen, dass die Kompositionen mit den Projektionen auf die Komponenten von $\prod_{\mathcal{C}(K)}[0,1]$ jeweils gleich sind, diese sind aber - nach Konstruktion - jeweils $g_i \circ f$. \\
Wegen $\overline{ι(X)} = β(X)$ ist nun
\begin{IEEEeqnarray*}{rCl}
    \hat{f}(β(X))&  =  & \hat{f}(\overline{ι_X(X)}) \\
                 &  \stackrel{\text{\autoref{lm:fortsetzung-stetiger-funktionen-in-dichten-hausdorff-räumen-sind-eindeutig}}}{\subset}  & \overline{(\hat{f} \circ  ι_X)(X)} \\
                 & \stackrel{\text{kommutiert}}{=} & \overline{(ι_k \circ f)(X)} \\
                 &\subset & \overline{ι_K(K)} \\
                 &\stackrel{\text{$K$ kompakt}}{=}& ι_K(K) \\
                 &\stackrel{\text{\autoref{thm:stone-cech-kompaktifizierung-ist-einbettung-für-vollständig-reguläre-räume}}}{\cong} &K
\end{IEEEeqnarray*}
\[
    \begin{split}
    \end{split}
.\] 
und damit können wir $\hat{f}$ mit $ι_K^{-1}$ verknüpfen um unsere gewünschte Abbildung  $β(X)\to K$ zu erhalten.
\end{proof}
\begin{remark*}[Kategorientheorie-Spam]
    $β(X)$ ist sogar ein Funktor von  $\Top$ (Kategorie der topologischen Räume) nach $\CHaus$ (Kategorie der kompakten Hausdorff-Räume). Das liegt daran, dass wir im Beweis von \autoref{thm:universelle-eigenschaft-der-stone-cech-kompaktifizierung} alle Schritte bis $\hat{f}(β(X))\subset \overline{ι_k(K)} = β(K)$ genauso durchführen können, ohne verwenden zu müssen, dass $K$ kompakter Hausdorff-Raum ist, und wir damit für $f: X \to K$ eine entsprechende Abbildung $\hat{f}: β(X) \to  β(K)$ induzieren, sodass
    \[
    \begin{tikzcd}
        X \ar[swap]{d}{f} \ar{r}{ι_X} & β(X) \ar{d}{\hat{f}}\\
        K \ar[swap]{r}{ι_K} & β(K)
    \end{tikzcd}
    \]
    kommutiert. Alternativ können wir auch \autoref{thm:universelle-eigenschaft-der-stone-cech-kompaktifizierung} auf die Abbildung $ι_k \circ  f : X \to  β(K)$ anwenden, da $β(K)$ nach Konstruktion kompakt und Hausdorffsch ist.
\end{remark*}
\begin{remark*}
    Man sollte nicht zu sehr darüber nachdenken, wie $β(X)$ aussieht: Die Konstruktion des Raumes ist äußerst nicht-konstruktiv und benutzt implizit das Auswahlaxiom (damit wir Tychonoff nutzen können. Man kann sich auch überlegen, dass der Satz von Tychonoff äquivalent ist zum Auswahlaxiom, weswegen wir auch nicht ohne es auskommen, das geht hier aber zu weit). Vielmehr sollte man die bloße Existenz eines solchen Raumes als theoretisches Ergebnis im Hinterkopf behalten, die wir benötigt haben, um die Frage nach der Kompaktifizierbarkeit eines Raumes zu beantworten. Auch der Spezialfall, dass  $β(X) = X$ für kompakte Hausdorff-Räume ist wichtig.
\end{remark*}
