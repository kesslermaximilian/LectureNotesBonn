\lecture[\"Aquivalente Metriken. Abgeschlossene Mengen. Teilraumtopologie. Hom\"oomorphismen. Quotientenräume und -topologie.]{Do 15 Apr 2021 10:14}{Grundbegriffe}


\begin{definition}[Äquivalente Metriken]\label{def:äquivalente-metrik}
    Zwei Metriken $d_1,d_2$ auf $X$ heißen \vocab[Metrik!äquivalente]{äquivalent}, falls Konstanten $c_1,c_2$ existieren, sodass
    \[
        \forall x,y\in X \colon \quad c_1\cdot d_1(x,y) \leq  d_2(x,y) \leq  c_2\cdot d_1(x,y)
    .\] 
\end{definition}
\begin{theorem}\label{thm:äquivalenz-von-metriken-ist-äquivalenzrelation}
    Äquivalenz (von Metriken) ist eine Äquivalenzrelation.
\end{theorem}
\begin{proof}
    \begin{description}
        \item[Reflexivität:] Klar mit $c_1 = c_2 = 1$
        \item[Symmetrie:] Seien $c_1,c_2$ wie in der Definition. Dann gilt mit entsprechender Division, dass
            \[
                \forall x,y \in X \colon : \quad \frac{1}{c_2}\cdot d_2(x,y) \leq  d_1(x,y) \leq  \frac{1}{c_1}d_2(x,y)
            .\] 
        \item[Transitivität:]. Seien $c_1,c_2,c_1',c_2'$ gewählt, sodass $\forall x \; \forall y\colon c_1d_1 (x,y) \leq  d_2 (x,y)\leq  c_2d_1(x,y)$ sowie $c_1'd_2 (x,y)\leq  d_3(x,y) \leq  c_2'd_2(x,y)$ (Also $d_1 \sim  d_2$ und $d_2 \sim d_3$). Dann ist auch
            \[
\forall x \; \forall y \colon \quad                c_1c_1'd_1(x,y) \leq  c_1'd_2(x,y)\leq d_3(x,y) \leq  c_2'd_2 (x,y) \leq  c_2d_1(x,y)
            .\] 
    \end{description}
\end{proof}

\begin{theorem}\label{thm:äquivalente-metriken-erzeugen-dieselbe-topologie}
    Äquivalente Metriken induzieren dieselbe Topologie.
\end{theorem}
\begin{proof}
    Wegen der Symmetrie genügt es zu zeigen, dass Mengen, die offen bezüglich $d_2$ sind, auch offen bezüglich $d_1$ sind. \\
    Sei nun $U\subset X$ offen bezüglich $d_2$ und $x\in U$. Dann existiert ein $ε>0$ mit  $U_{d_2}(x,ε) \subset U$. Ist nun $d_1(x,y) < \frac{ε}{c_2}$, dann ist
    \[
        d_2(x,y) \leq  c_2d_1(x,y) < ε
    .\] 
und damit ist
\[
    U_{d_1}\left(x,\frac{ε}{c_2}\right) \subset U_{d_2} \left( x,ε \right) \subset U
.\] 
und somit ist $U$ auch offen bezüglich  $d_1$.
\end{proof}

\begin{remark}
    Es gibt auch nicht-äquivalente Metriken, die die gleiche Topologie induzieren. Siehe hierzu \autoref{aufgabe-1.2}.
\end{remark}
\begin{remark}
    Je zwei Normen auf $\R^n$ sind äquivalent, induzieren also dieselbe Topologie, das beweisen wir jedoch hier nicht.
\end{remark}

\begin{definition}[Umgebung]\label{def:umgebung}
    Sei $X$ ein topologischer Raum und $U\subset X$ sowie $x\in X$. Dann heißt $U$ \vocab[Umgebung!von $x$]{Umgebung von $x$}, falls es eine offene Teilmenge  $O\subset X$ gibt, mit $x\in O\subset U$.
\end{definition}
\begin{remark}
    Für metrische Räume stimmt dies mit der vorherigen Definiton überein.
\end{remark}


\begin{theorem}\label{thm:offene-menge-ist-umgebung-all-ihrer-punkte}
    Sei $X$ ein topologischer Raum und  $U\subset X$. Dann sind äquivalent:
    \begin{enumerate}[1)]
        \item $U$ ist offen.
        \item $U$ ist Umgebung aller ihrer Punkte.
    \end{enumerate}
\end{theorem}
\begin{proof}
    '$1) \implies 2)$' ist klar, wähle einfach $O = U$. \\
    '$2)\implies_1)$'. Für jedes $x\in U$ existiert also $U_x$ mit  $x\in U_x \subset U$. Dann ist aber
    \[
    U = \bigcup_{x\in U} U_x
    .\] 
    offen als Vereinigung offener Mengen.
\end{proof}

\begin{definition*}[Abgeschlossene Mengen]\label{def:abgeschlossene-menge}
    Sei $X$ ein topologischer Raum. Eine Teilmenge  $A\subset X$ heißt \vocab[Menge!abgeschlossen]{abgeschlossen}, falls ihr Komplement $X \setminus A = \left \{x\in X \mid  x\not\in A\right\} $ offen ist.
\end{definition*}
\begin{remark}
    Für metrische Räume stimmt das mit dem Begriff aus der Analysis überein.
\end{remark}

\begin{theorem}[Dualität]\label{offen-abgeschlossen-ist-dual}
    Ein topologischer Raum lässt sich auch über seine abgeschlossenen Mengen charakterisieren. Diese müssen erfüllen:
    \begin{enumerate}[i)]
        \item $\emptyset,X$ sind abgeschlossen
        \item Für $A_1,\ldots,A_n$ abgeschlossen ist auch $A_1\cup \ldots \cup A_n$ abgeschlossen.
        \item Für eine Familie $\left \{A_i\right\} _{\in I}$ abgeschlossener Mengen ist auch
            \[
            \bigcap_{i\in I} A_i
            .\] 
            abgeschlossen.
    \end{enumerate}
\end{theorem}

\begin{recap}
    Wenn wir von einer Familie von Mengen $\left \{A_i\right\} _{\in I}$ sprechen, meinen wir, dass $I$ eine Menge ist, und für jedes $\in I$ ist $A_i$ eine Teilmenge von  $X$. Formal können wir dies als eine Funktion  $I \to  \mathcal{P}(X)$ darstellen.
\end{recap}
\begin{proof}
    \begin{enumerate}[i)]
        \item $X \setminus \emptyset = X$, $X\setminus X = \emptyset$ sind abgeschlossen.
        \item  \[
                \underbrace{\bigcap_{i=1}^n (X\setminus A_i)}_{\text{offen}} = X \setminus  \bigcup_{i=1}^n A_i \quad \implies \bigcup_{i=1}^n A_i \text{ abgeschlossen}
        .\] 
        \item \[
                \underbrace{\bigcup_{i\in I} (\underbrace{X\setminus A_i}_{\text{offen}})}_{\text{offen}} = X \setminus \bigcap_{i \in I} A_i \quad \implies \bigcap_{i \in I} A_i \text{ abgeschlossen}
        .\] 
    \end{enumerate}
\end{proof}

\begin{theorem}[Stetigkeit mit abgeschlossenen Mengen]\label{thm:stetig-gdw-urbilder-abgeschlossener-mengen-sind-abgeschlossen}
    Sei $f:X \to  Y$ eine Funktion zwischen topologischen Räumen. Dann sind äuqivalent:
    \begin{enumerate}[1)]
        \item $f$ ist stetig
        \item $\forall U\subset Y$ offen ist $f^{-1}(U) \subset X$ offen
        \item  $\forall A\subset Y$ abgeschlossen ist $f^{-1}(A)$ abgeschlossen
    \end{enumerate}
\end{theorem}
\begin{proof}
    \begin{equation*}
        \begin{split}
            f \text{ stetig} &\iff \forall U \subset  Y \text{ offen ist } f^{-1}(U) \text{ offen}  \\
                             &\iff  \forall A \subset Y \text{ abgeschlossen ist } f^{-1}(Y \setminus A) \text{ offen} \\
                             &\iff \forall A\subset Y \text{ abgeschlossen ist } X \setminus f^{-1}(A) \text{ offen} \\
                             &\iff  \forall A\subset Y \text{ abgeschlossen ist } f^{-1}(A) \text{ abgeschlossen}
        \end{split}
    \end{equation*}
\end{proof}


Wir erinnern uns: Ist $(X,d)$ ein metrischer Raum, so auch  $\left(Y, d_{Y\times Y}\right) \quad \forall Y\subset X$. Wie ist dies für topologische Räume?
\begin{warning}
    $(Y, \mathcal{O}_X \cap \mathcal{P}(Y))$ ist im allgemeinen \textbf{kein} topologischer Raum. (wenn $Y$ nicht offen ist, denn dann ist $Y\not\in \mathcal{S}_X \cap \mathcal{P}(X)$)
\end{warning}
\begin{theoremdef}[Teilraumtopologie]\label{def:teilraumtopologie}
    Sei $X$ ein topologischer Raum,  $Y\subset X$. Dann ist
    \[
    \mathcal{O}_Y := \left \{U \cap Y \mid  U\subset X \text{ offen}\right\} 
    .\] 
    eine Topologie auf $Y$, die  \vocab[Topologie!Teilraum-]{Teilraumtopologie} oder auch \vocab[Topologie!Unterraum-]{Unterraumtopologie} genannt wird.  
\end{theoremdef}

\begin{example}
    Betrachte $\R^1 \subset \R^2$ als Unterraum. Schneiden wir eine offene Menge in $\R^2$ mit $\R^1$, so erhalten wir ein offenes Intervall: \\
\begin{minipage}{\textwidth}
\centering    
    \incfig{r1-als-unterraum-von-r2}
    \captionof{figure}{$\R^1$ als Unterraum von $\R^2$}
\end{minipage}
\end{example}
\begin{proof}
    \begin{itemize}
        \item Es sind $\emptyset = \emptyset \cap Y$ und $Y = X \cap Y$ offen.
        \item Es ist 
            \[
                \bigcap_{i=1}^n \left( U_i \cap Y \right)  = \left( \bigcap_{i=1}^n U_i \right)  \cap Y
            .\] 
        \item Es ist
            \[
                \bigcup_{i\in I} \left( U_i \cap Y \right) = \left( \bigcup_{i \in  I} U_i \right) \cap Y
            .\] 
    \end{itemize}
\end{proof}
\begin{warning}
    Für $Z\subset Y\subset X$ muss man zwischen 'offen in $Y$' und  'offen in  $X$' unterscheiden, falls  $Y$ nicht offen ist.
\end{warning}
\begin{remark*}
    Ist $Y\subset X$ offen, so stimmen die beiden vorherigen Konzepte tatsächlich überein, d.h. eine Menge $Z\subset Y$ ist offen in $Y$, genau dann, wenn sie offen in  $X$ ist.
\end{remark*}

\begin{remark}
    Sei $(X,d)$ ein metrischer Raum und  $Y\subset X$ eine Teilmenge. Die Unterraumtopologie auf $Y$ bzgl. der Topologie auf  $X$ ist gleich der Topologie indzuziert von der eingeschränkten Metrik.
\end{remark}
\begin{proof}
    Für $y\in Y$ ist
    \[
        U_{d\mid _{Y\times Y}} (y,ε) = U_d(y,ε)\cap Y
    .\] 
    , deswegen werden von beiden Metriken die gleichen offenen Mengen induziert.
\end{proof}



\begin{example}
    Der \vocab{Einheitskreis} als Unterraum von $\R^2$:
    \[
    \left \{x\in \R^2 \mid  \lVert x \rVert _2 = 1\right\}  =: S^1 \subset \R^2
    .\] 
    Genauso gibt es die  \vocab{$n$-Sphäre} definiert durch
    \[
    \left \{x\in \R^{n+1}\mid  \lVert x \rVert _2 = 1\right\} =: S^n \subset \R^{n+1}
    .\] 
\end{example}

\begin{definition}[Homöomorphie]\label{def:homöomorph}
    Eine Abbildung $f: X \to  Y$ zwischen topologischen Räumen heißt \vocab{Homöomorphismus}, falls $f$ stetig und bijektiv ist und  auch $f^{-1}: Y \to  X$ stetig ist.  \\
    Existiert solch ein $f$, so heißen  $X,Y$  \vocab[Homöomorphismus!homöomorph]{homöomorph} 
\end{definition}
\begin{example}
    Die Räume $(\R^2, \lVert \cdot  \rVert _2)$ und $(\C, \abs{\cdot })$ sind homöomorph mittels der Abbildung
        \begin{equation*}
        \begin{array}{c c l} 
            \R^2 & \longrightarrow & \C \\
            (x,y) & \longmapsto &  x+iy
        \end{array}
    \end{equation*}
\end{example}
\begin{warning}
    Nicht jede stetige Bijketion ist ein Homöomorphismus.
\end{warning}
\begin{example}
    Betrachte für eine Menge $X$ die Identitätsabbildung  $(X, \mathcal{P}(X)) \stackrel{\id_X}{\to} (X, \left \{\emptyset,X\right\})$ von der diskreten in die indiskrete Topologie. Diese ist stetig, aber die Umkehrabbildung nicht (falls $\abs{X} \geq 2$).
\end{example}

\section{Quotientenräume} 
\begin{ddefinition}[Äquivalenzklasse]
Sei $\sim $ eine Äquivalenzrelation auf $X$. Für  $x\in X$ definieren wir die \vocab{Äquivalenzklasse} $[x]$ von $x$ durch:
 \[
     [x] := \left \{x' \in X \mid  x\sim x'\right\} 
.\] 
Wir setzen
\[
X / \sim  := \left \{[x] \mid  x\in X\right\} 
.\] 
als die \vocab[Äquivalenzklasse!Menge der]{Menge der Äquivalenzklassen} von $X$ bezüglich  $\sim $. Definiere nun
    \begin{equation*}
    q: \left| \begin{array}{c c l} 
    X & \longrightarrow & X / \sim  \\
    x & \longmapsto &  [x]
    \end{array} \right.
\end{equation*}
als die \vocab[Projektion!kanonische]{kanonische Projektion} von $X$ auf seine Äquivalenzklassen.
\end{ddefinition}
\begin{fact}
   Wir stellen fest, dass $q$ surjektiv ist.
\end{fact}

 \begin{recap}
     Für eine Surjektion $f: X \to  Y$ ist $x\sim y :\iff  f(x) = f(y)$ eine Äquivalenzrelation auf $X$ und
             \begin{equation*}
             \begin{array}{c c l} 
             X / \sim  & \longrightarrow & Y \\
             \left[x\right] & \longmapsto &  f(x)
             \end{array}
         \end{equation*}
ist eine Bijektion, wir erhalten also eine Korrespondenz zwischen Äquivalenzrelationen und surjektiven Abbildungen aus $X$.
\end{recap}

\begin{theoremdef}[Quotiententopologie]\label{def:quotiententopologie}
    Sei $X$ ein topologischer Raum und  $\sim $ eine Äquivalenzrelation auf $X$. Sei  $q: X \to  X / \sim $ die kanonische Projektion. Dann definiert
    \[
        \mathcal{S}_{X / \sim } := \left \{U\subset X / \sim \mid  q^{-1}(U) \subset X \text{ offen}\right\} 
    .\] 
    eine Topologie auf $X / \sim $, genannt die \vocab[Topologie!Quotienten-]{Quotiententopologie}. 
\end{theoremdef}
\begin{proof}
    Wir prüfen die Axiome einer Topologie:
    \begin{itemize}
        \item Es ist $q^{-1}(\emptyset) = \emptyset$ und $q^{-1}(X / \sim ) = X$, also sind beide Mengen offen.
        \item Sind $U_1, \ldots,U_n\subset X / \sim $ offen, so ist
            \[
                q^{-1}(U_1\cap \ldots \cap A_n) = q^{-1}(U_i) \cap \ldots \cap q^{-1}(U_n)
            .\] 
            offen in $X$, also ist  $U_1\cap \ldots \cap U_n$ offen in $X / \sim $ nach Definition.
        \item Ist $\left \{U_i\right\} _{i\in I}$ eine Familie offener Teilmengen von $X / \sim $, dann ist
            \[
                q^{-1}\left( \bigcup_{i \in  I} U_i \right) = \bigcup_{i \in I} q^{-1}(U_i)
            .\] 
            offen in $X$, also ist  $\bigcup_{i \in I} U_i$ offen in $X / \sim $ nach Definition.
    \end{itemize}
\end{proof}






\begin{remark}
    Die Projektion $q: X \to  X / \sim $ ist stetig und die Quotiententopologie ist maximal (bezüglich Inklusion, lies: 'am feinsten') unter allen Topologien auf $X / \sim $, für die $q$ stetig ist.
\end{remark}
\begin{theorem}[Universelle Eigenschaft der Quotiententopologie]\label{thm:universelle-eigenschaft-der-quotiententopologie}
    Sei $f : X \to  Y$ stetig und $q : X \to  X / \sim $ die kanonische Projektion. Angenommen, es existiert $g : X / \sim \to  Y$ mit $f = g \circ  q$. Dann ist $g$ stetig und in diesem Fall ist $g$ eindeutig. \\
    \begin{minipage}{\textwidth}
    \centering    
     \begin{tikzcd}
         X \ar{r}{f} \ar[swap]{d}{q} & Y  \\
                                      X / \sim \ar[dotted,swap]{ur}{g}
    \end{tikzcd}
    \end{minipage}
\end{theorem}
\begin{remark}
    $g$ existiert genau dann, wenn  $x \sim x' \implies f(x) = f(x')$
\end{remark}
\begin{trivial*}
    Das ist eine universelle Eigenschaft im Sinne der Kategorientheorie, d.h. für einen Raum $X$ und eine Äquivalenzrelation existiert bis auf eindeutigen Isomorphismus stets genau ein topologischer Raum $(X / \sim  , \mathcal{S})$ zusammen mit einer stetigen Abbildung $q : X \to  X / \sim $, sodass $x\sim x' \implies q(x) = q(x')$, sodass das Tripel $(X, X / \sim , q)$ obige Eigenschaft hat. Wir können also obige Eigenschaft auch als Definition der Quotiententopologie verwenden, und aus dieser folgt auch die Eindeutigkeit. Existenz haben wir mit unserer vorherigen Definition gezeigt.
\end{trivial*}

\begin{proof}
    Sei $U\subset Y$ offen. Dann ist
    \[
        q^{-1}(g^{-1}(U)) \stackrel{f = g \circ  q}{=} f^{-1}(U)
    .\] 
    offen, weil $f$ stetig ist. Also ist  $g^{-1}(U)$ offen per Definition ($g^{-1}(U)$ ist nach Definition genau dann offen in $X / \sim $, wenn $q^{-1}(g^{-1}(U))$ offen in $X$ ist) und somit ist $g$ stetig. 
\end{proof}
\begin{example}
Sei $X = [0,1]\subset \R$ das \vocab{Einheitsintervall} (mit der Unterraumtopologie bezüglich $\R$) mit der Äquivalenzrelation erzeugt von $0\sim 1$. Wir 'identifizieren' also die Punkte $\left \{0\right\} ,\left \{1\right\} $ miteinander.
\end{example}
\begin{theorem}[Kreishomöomorphie]\label{thm:kreis-ist-quotientenraum-von-einheitsintervall}
    Der Quotientenraum $[0,1] / (0\sim 1)$ ist homöomorph zu $S_1$.
\end{theorem}


