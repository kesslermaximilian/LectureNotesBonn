\documentclass[a4paper, german, lecturenumbers = true, number small environments = theorem, hide version]{mkessler-script}

\course{Einführung in die Geometrie und Topologie}
\lecturer{Daniel Kasprowski}
\assistant[f]{Arunima Ray}
\author{Maximilian Keßler}

\RequirePackage{mkessler-math}
\RequirePackage{mkessler-fancythm}
\usepackage{epsfig}
%\usepackage{psfrag}
%\usepackage{sseq} (if you need to draw spectral sequences, please use this package, available at http://wwwmath.uni-muenster.de/u/tbauer/)
\usepackage{mathrsfs}
\usepackage{amscd}
\usepackage{amsbsy}
\usepackage{verbatim}
\usepackage{moreverb}

\newtheorem{prop}[theorem]{Proposition}
\newtheorem{cor}[theorem]{Corollary}
\newtheorem{conj}[theorem]{Conjecture}


\theoremstyle{definition}
\newtheorem{hw}{Homework}
\newtheorem{exercise*}[exercise]{$\star$ Exercise}

\theoremstyle{remark}
\newtheorem{aside}[theorem]{Aside}

\newcommand{\nn}{\nonumber}
\newcommand{\nid}{\noindent}
\newcommand{\ra}{\rightarrow}
\newcommand{\la}{\leftarrow}
\newcommand{\xra}{\xrightarrow}
\newcommand{\xla}{\xleftarrow}
\newcommand{\tto}{\longrightarrow}

\newcommand{\weq}{\xrightarrow{\sim}}
\newcommand{\cofib}{\rightarrowtail}
\newcommand{\fib}{\twoheadrightarrow}

\newcommand{\IRep}{\mathrm{IRep}}
\newcommand{\IHom}{\mathrm{IHom}}

\def\llarrow{   \hspace{.05cm}\mbox{\,\put(0,-2){$\leftarrow$}\put(0,2){$\leftarrow$}\hspace{.45cm}}}
\def\rrarrow{   \hspace{.05cm}\mbox{\,\put(0,-2){$\rightarrow$}\put(0,2){$\rightarrow$}\hspace{.45cm}}}
\def\lllarrow{  \hspace{.05cm}\mbox{\,\put(0,-3){$\leftarrow$}\put(0,1){$\leftarrow$}\put(0,5){$\leftarrow$}\hspace{.45cm}}}
\def\rrrarrow{  \hspace{.05cm}\mbox{\,\put(0,-3){$\rightarrow$}\put(0,1){$\rightarrow$}\put(0,5){$\rightarrow$}\hspace{.45cm}}}

\def\cA{\mathcal A}\def\cB{\mathcal B}\def\cC{\mathcal C}\def\cD{\mathcal D}
\def\cE{\mathcal E}\def\cF{\mathcal F}\def\cG{\mathcal G}\def\cH{\mathcal H}
\def\cI{\mathcal I}\def\cJ{\mathcal J}\def\cK{\mathcal K}\def\cL{\mathcal L}
\def\cM{\mathcal M}\def\cN{\mathcal N}\def\cO{\mathcal O}\def\cP{\mathcal P}
\def\cQ{\mathcal Q}\def\cR{\mathcal R}\def\cS{\mathcal S}\def\cT{\mathcal T}
\def\cU{\mathcal U}\def\cV{\mathcal V}\def\cW{\mathcal W}\def\cX{\mathcal X}
\def\cY{\mathcal Y}\def\cZ{\mathcal Z}

\def\sA{\mathscr A}\def\cB{\mathcal B}\def\cC{\mathcal C}\def\cD{\mathcal D}
\def\cE{\mathcal E}\def\cF{\mathcal F}\def\sG{\mathscr G}\def\cH{\mathcal H}
\def\cI{\mathcal I}\def\cJ{\mathcal J}\def\cK{\mathcal K}\def\cL{\mathcal L}
\def\cM{\mathcal M}\def\cN{\mathcal N}\def\cO{\mathcal O}\def\cP{\mathcal P}
\def\cQ{\mathcal Q}\def\cR{\mathcal R}\def\cS{\mathcal S}\def\cT{\mathcal T}
\def\cU{\mathcal U}\def\cV{\mathcal V}\def\cW{\mathcal W}\def\cX{\mathcal X}
\def\cY{\mathcal Y}\def\cZ{\mathcal Z}

\def\fG{\mathfrak G}\def\fH{\mathfrak H}
\def\fS{\mathfrak S}\def\fN{\mathfrak N}\def\fX{\mathfrak X}\def\fY{\mathfrak Y}

\def\op{\textrm{op}}\def\ob{\textrm{ob}}

%\def\Iso{\mathcal Iso}\def\cInn{\mathcal Inn}

\def\fg{\mathfrak g}\def\fh{\mathfrak h}\def\fri{\mathfrak i}\def\fp{\mathfrak p}
\def\fA{\mathfrak A}\def\fU{\mathfrak U}

\def\AA{\mathbb A}\def\BB{\mathbb B}\def\CC{\mathbb C}\def\DD{\mathbb D}
\def\EE{\mathbb E}\def\FF{\mathbb F}\def\GG{\mathbb G}\def\HH{\mathbb H}
\def\II{\mathbb I}\def\JJ{\mathbb J}\def\KK{\mathbb K}\def\LL{\mathbb L}
\def\MM{\mathbb M}\def\NN{\mathbb N}\def\OO{\mathbb O}\def\PP{\mathbb P}
\def\QQ{\mathbb Q}\def\RR{\mathbb R}\def\SS{\mathbb S}\def\TT{\mathbb T}
\def\UU{\mathbb U}\def\VV{\mathbb V}\def\WW{\mathbb W}\def\XX{\mathbb X}
\def\YY{\mathbb Y}\def\ZZ{\mathbb Z}

\def\TOP{\mathcal{TOP}}\def\GRP{\mathcal{GRP}}\def\GRPD{\mathcal{GRPD}} \def\CAT{\mathcal{CAT}} \def\SET{\mathcal{SET}}

\def\id{\mathrm{id}}\def\Id{\mathrm{Id}}
\def\inverse{^{-1}}



\begin{document}
    \maketitle
    \begin{abstract}
    {\color{red} Achtung:} Diese Version des Skripts benutze ich zur Bearbeitung! Einige Dinge fehlen, dafür gibt es TODO-Notes. Für Inhalte, benutzt die \href{https://kesslermaximilian.github.io/LectureNotesBonn/2021_Topologie.pdf}{normale Version}
    \end{abstract}
    \newpage
    \listoftodos
    \newpage
    \summaryoflectures
    \newpage
    % start lectures
    \setcounter{section}{0}
    \setcounter{dummy}{0}
    \setcounter{smalldummy}{0}
    \setcounter{figure}{0}
    \setcounter{claim}{0}
    \setcounter{lecture}{0}
    \lecture[Motivationsfragen. Brown'sche Bewegung. Ereignisse, Wahrscheinlichkeiten, Modell von Zufallsexperimenten.]{Mo 12 Apr 2021 10:16}{Grundbegriffe}

\begin{itemize}
    \item Es gibt ein Helpdesk, auch explizit für Studentinnen
    \item die Vorlesung wird aufgenommen, und zwar ohne Videos der Teilnehmenden sowie des Dozenten, die Aufzeichnung werden anschließend in Sciebo hochgeladen.
    \item Es gibt ein Diskussionsforum für Fragen (auf eCampus).
    \item Ab heute Abend, 18 Uhr (Mo 12 Apr 2021 18:00), kann man sich auf eCampus für die Übungsgruppen registrieren und endet am Dienstag Abend um 24 Uhr (Di 12 Apr 2021 24:00), es wird versucht, die Studenten gleichmäßig zu verteilen.
    \item Falls ihr in der Warteliste landet und gewünscht ist, in der Gruppe abzugeben, schreibt eine Mail mit den gewünschten Abgabepartner, dann kann eine gemeinse Einteilung erfolgen.
    \item Es gibt auch das Modul \verb?AlmaIIb?. Registriert euch noch nicht, dies ist für den 2. Teil der Vorlesung notwendig. 
    \item Die Abgabe der Übungsblätter erfolgt einheitlich jeden Freitag um 12 Uhr.
    \item Gruppenabgaben sind erlaubt, bis zu einer Größe von maximal 4 StudentInnen.
    \item Das 1. Blatt ist freiwillig und gibt Bonuspunkte.
    \item Für die Klausurzulassung werden 50\% der Punkte benötigt. Von den Programmieraufgaben müssen mindestens 4 von 6 zufriedenstellend bearbeitet werden.
    \item Programmieraufgaben gibt es ab dem 2. Übungsblatt auf jedem 2. Blatt. Die Bearbeitungszeit beträgt dann 2 Wochen.
\end{itemize}


\section*{Einleitung}
In der Vorlesung werden wir sehen:
\begin{description}
    \item[Teil 1: Diskrete Stochastik]
        \begin{itemize}
            \item Zufallsvariablen
            \item Bedingte Wahrscheinlichkeiten
            \item Unabhängigkeit von Variablen
            \item Monte-Carlo Methoden
        \end{itemize}
    \item[Teil 2: Numerische Analysis]
        \begin{itemize}
            \item Iterative Verfahren
            \item Interpolation von Daten (durch Polynome, trigonometrische Funktionen, \ldots)
            \item Numerische Verfahren für die Integration
        \end{itemize}
\end{description}


\section{Diskrete Stochastik}
\subsection{Einleitung}
\begin{goal}
    Beschreibung von Systemen, die einen Anteil an \vocab{Zufall} haben, d.h. nicht 100\% deterministisch sind.
\end{goal}
\begin{example}
    \begin{itemize}
        \item Spiele: Kartenspiele, Glücksspiele, \ldots
        \item Statistik: Umfragen, Versicherung
        \item Komplexe Systeme: Wettermodelle, Finanzmärkte
    \end{itemize}
\end{example}

\underline{Was sind Quellen von Zufall?}
\begin{itemize}
    \item Zu komplexe Systeme. Dann sieht der Gesamteffekt zufällig aus.
    \item Fehlende Informationen (z.B. bei einem Kartenspiel)
    \item Chaotische Systeme (Wetter
    \item Intrinsisch unvorhersagbare Systeme (z.B. radioaktiver Zerfall)
\end{itemize}
\begin{question}
    \begin{enumerate}[(1)]
        \item Wie modelliert man ein System mit Zufall?
        \item Wie simuliert man ein System mit Zufall? (anwendungstechnischer)
        \item Welche Voraussagen kann man machen?
    \end{enumerate}
\end{question}


\begin{example}
    Die \vocab{Brown'sche Bewegung}. Das System ist implizit ein Pollen mit vielen Wassermolekülen ($\sim 10^{23})$, die sich im Prinzip deterministisch bewegen. \\
    $\implies$ Wir erhalten ein Gleichungssystem mit $(N+1)\cdot 6$ (3 Positionen, 3 Geschwindigkeit) Variablen. Dieses ist de facto unlösbar. \\

    Was wollen wir hier eigentlich untersuchen? -> Die Bewegung des Pollens, jedoch nicht die der einzelnen Wassermoleküle. \\
    In einer \vocab{Modellierung} ersetzt man die Stöße, die ,durch die Wassermoleküle entstehen durch \vocab{zufällige Stöße}. 
\end{example}

\underline{Diskretes Modell:} Die Zeit bewegt sich in $n\in \left \{0,1,2,\ldots\right\} $. Sei
\[
    Z(n) := (\text{Position des Pollens zur Zeit $n$}) \in  \Z^3
.\] 
OBdA setzen wir $Z(0) = 0$. \\
\underline{Dynamik}: $Z(n+1) = Z(n) + \xi_n$, wobei wir $\xi_n$ aus dem Ergebnis eines Würfelwurfs bestimmen werden:
 \[
\xi_n = \begin{cases}
    (1,0,0) & \text{wenn Würfel}=1 \\
    (-1,0,0) & \text{wenn Würfel}=2 \\
    (0,1,0) & \text{wenn Würfel}=3 \\
    (0,-1,0) & \text{wenn Würfel}=4 \\
    (0,0,1) & \text{wenn Würfel}=5 \\
    (0,0,-1) & \text{wenn Würfel}=6
\end{cases}
.\] 

\begin{question}
    Welche Fragen können wir mit solch einem System nun beantworten? Was pasiert, wenn $n\gg 1$?
\end{question}
\begin{enumerate}[\protect\circled{\alph*}]
    \item Typischerweise erhalten wir $\abs{Z(n)} =  O(\sqrt{n}) $ 
    \item Wenn wir die Frequenz von $[Z(n)]_i$ betrachten, (d.h. bei welcher Koordinate in Richtung $i$ befinden wir uns nach  $n$ Würfen) sehen wir typischerweise: 
        \begin{figure}[h]
            \centering
\begin{tikzpicture}[
    declare function={binom(\k,\n,\p)=\n!/(\k!*(\n-\k)!)*\p^\k*(1-\p)^(\n-\k);}
]
\begin{axis}[
    samples at={-15,...,15},
    yticklabel style={
        /pgf/number format/fixed,
        /pgf/number format/fixed zerofill,
        /pgf/number format/precision=2
    },
    ybar=0pt, bar width=1
]
\addplot [fill=orange, fill opacity=0.5] {binom(x+30,60,0.5)}; \addlegendentry{$[Z(n)]_i$}
    \addplot[draw=red,thick,smooth] {1/(sqrt(30*pi)) *exp(-1/30*x^2)}; \addlegendentry{\text{Gaussglocke}}
\end{axis}
\end{tikzpicture}
\caption{Binomialverteilung und Gaussglocke}
\end{figure}
Für $n\gg 1$ sieht diese Verteilung dann ungefähr wie die Gaussglocke aus. \\
\end{enumerate}
\underline{Skalierung:} Wir setzen nun
\[
    B(t) = \lim_{n \to \infty} \frac{Z(\left\lfloor nt \right\rfloor )}{\sqrt{n} }
.\] 
und dies ist dann die Brownsche Bewegung.

Nun möchten wir Vorhersagen treffen können:
\begin{question}
        Ist $Z(n)$ in einer gegebenen Menge  $A$?
\end{question}
Das kann man (im Allgemeinen) nicht einfach mit 'Ja' oder 'Nein' beantworten. Stattdessen müssen wir fragen:
\begin{question}
Wenn man $Z(n)$ beobachtet, wie häufig wird  $Z(n)$ in  $A$ sein?
\end{question}
Diese Frage lässt sich mit einer Zahl $\in [0,1]$ beantworten.

\subsection{Ereignisse und Wahrscheinlichkeiten}
Wir benötigen 3 Grundelemente:
\begin{enumerate}[(1)]
    \item Die Menge $\Omega$ von möglichen \vocab{Ergebnissen}. die Elemente von $\Omega$ heißen auch  \vocab{Elementarereignisse}.
    \item Die Menge $\mathcal{F}$ der \vocab{Ereignisse}. Ein Ereignis  $E$ ist eine Eigenschaft, die mit einer Teilmenge $G\subset \Omega$ assoziiert ist: $ω\in G \iff $ Eigenschaft $E$ ist erfüllt.
    \item Eine \vocab{Wahrscheinlichkeitsverteilung (auch W-maß)}:
        \[
            \mathbb{P}: \mathcal{F} \to  [0,1]
        .\] 
\end{enumerate}
\begin{remark*}
    Wir werden noch sehen, dass gewisse Dinge für unsere Begriffe erfüllt sein müssen, dazu aber später mehr.
\end{remark*}

\begin{example}
    Eine Urne hat 12 nummerierte Kugeln (von 1 bis 12).
    \begin{enumerate}[(1)]
        \item Das \vocab{Zufallsexperiment} besteht daraus, dass wir eine Kugel aus der Urne ziehen und die Zahl notieren, die wir sehen. D.h.
            \[
            \Omega = \left \{1,\ldots,12\right\} 
            .\] 
            Ein Elementarereignis ist nun z.B. gegeben durch $ω = \left \{5\right\}  \equiv 5$ (wir vereinfachen die Notation).
        \item Mögliche Ereignisse sind z.B:
            \begin{equation}
                \begin{split}
                    A &= \text{'Die Zahl ist gerade'} \\
                    B&= \text{'Die Zahl ist }\leq 5 \text{'}\\
                    C &= \text{'Die Zahl ist 8'}
                \end{split}
            \end{equation}
            Die assoziierten Mengen sind dann
            \begin{equation}
                \begin{split}
                    A &= \left \{2,4,6,8,10,12\right\}  \\
                     B &= \left \{1,2,3,4,5\right\} \\
                      C & = \left \{8\right\} 
                \end{split}
            \end{equation}
        \item Für die Wahrscheinlichkeiten nehmen wir an, dass jede Kugel die gleiche Chance hat, gezogen zu werden, d.h.
            \[
                \forall G\in \mathcal{F}: \mathbb{P}(G) = \frac{\abs{G}}{\abs{\Omega} }
            .\] 
            Wir erhalten nun die Wahrscheinlichkeiten
            \[
                \mathbb{P}(A) = \frac{6}{12}=\frac{1}{2} \qquad \mathbb{P}(B) = \frac{5}{12} \qquad \mathbb{P}(C) = \frac{1}{12}
            .\] 
    \end{enumerate}
\end{example}

\begin{notation}
    $A\equiv \left \{ω\in \Omega \mid  ω\in A\right\} \equiv  \left \{ω\in A\right\} \equiv \left \{A \text{ tritt ein}\right\} $
\end{notation}







    \lecture[$σ$-Algebren, Messräume. Wahrscheinlichkeitsverteilungen, Wahrscheinlichkeitsräume. Einschluss-Ausschluss-Prinzip. Endliche (diskrete) Wahrscheinlichkeitsräume.]{Mi 14 Apr 2021 10:17}{}
Wir kennen nun die Grundbegriffe $\Omega, \mathcal{F}, \mathbb{P}$ zur Beschreibung von Zufallsexperimenten, die wir uns nun genauer ansehen wollen:
\begin{question}
    Welche Struktur muss $\mathcal{F}$ besitzen.
\end{question}
Sein $A,B\in \mathcal{F}$, dann können wir das Ereignis $A \cap B$ betrachten, d.h. beide der Eigeschaften treten ein. Genauso sollte
 \[
A^{c} := \Omega \setminus A 
.\] 
, das \vocab{Komplement von $A$}, bzw. das \vocab{Gegenereignis} von $A$ ebenfalls in  $\mathcal{F}$  sein. Aus den beiden vorherigen Eigenschaften folgt bereits, dass
\[
    A \cup B= (A^{c} \cap B^{c})^{c}
.\] 
ebenfalls in $\mathcal{F}$ sein wird. \\
Eine Menge $\mathcal{F}$ mit solchen Eigenschaften heißt \vocab{Algebra}.
\begin{dnotation}
Seien nun $A,B$ und $(A_i)_{i\in I}$  Ereignisse, wobei $I$ endlich oder abzählbar sei. Dann notieren wir die folgenden Ereignisse:
\begin{enumerate}[label=\protect\circled{\alph*}]
    \item \emphasize{$A \cup B$} : $ω\in A \cup B \iff  ω\in A \lor ω\in B$, d.h. $A\cup B$ tritt ein, genau dann, wenn  $A$ eintritt oder  $B$ eintritt.
        \item  \emphasize{$\bigcup_{i \in  I} A_i$}: $ω\in \bigcup_{i \in  I} A_i$, wenn es ein $i\in I$ gibt, sodass $\omega \in A_i$
    \item  \emphasize{$A \cap B$}: $\omega\in A \cap B \iff  $ A \underline{und} B treten ein.
        \item \emphasize{$\bigcap_{i \in I} A_i$}: $\omega\in \bigcap_{i \in I}A_i \iff \forall i \in I \colon$ $A_i$ tritt ein.
            \item \emphasize{$A = \emptyset$} ist das Ereignis, das  \underline{nie} eintritt. \\
                \emphasize{$A = \Omega$} ist das Ereignis, dass \underline{immer} eintritt.
\end{enumerate}
\end{dnotation}

\begin{definition}[$\sigma$-Algebra]\label{def:sigma-algebra}
    Eine  \vocab{$\sigma$-Algebra} ist eine nicht leere Menge $\mathcal{F}$ von Teilmengen von $\Omega$ mit den Eigenschaften:
    \begin{enumerate}[label=\protect\circled{\alph*}]
        \item $\Omega \in \mathcal{F}$
        \item $\forall A\in \mathcal{F} \colon A^{c}\in \mathcal{F}$.
        \item Falls $(A_i)_{i \in I}\in \mathcal{F}$, dann auch $\bigcup_{i=1} ^{\infty}A_i \in \mathcal{F}$
    \end{enumerate}
    Wir nennen $(\Omega,\mathcal{F})$ dann einen \vocab{Messraum}. 
\end{definition}

\begin{lemma}\label{lm:weitere-eigenschaften-einer-sigma-algebra}
    Sei $\mathcal{F}$ eine $\sigma$-Algebra, dann ist:
    \begin{enumerate}[label=\protect\circled{\alph*}]
        \item $\emptyset\in \mathcal{F}$
        \item $A,B \in \mathcal{F} \implies A \cup B \in \mathcal{F}$ und $A\cap B \in \mathcal{F}$.
        \item $(A_i)_{i \in I}\in \mathcal{F} \implies \bigcap_{i=1}^{\infty}A_i \in \mathcal{F}$.
    \end{enumerate}
\end{lemma}
\begin{proof}
    \begin{enumerate}[label=\protect\circled{\alph*}]
        \item $\emptyset = \Omega^{c} \in \mathcal{F}$ nach Eigenschaften \circled{a} und \circled{b} aus der Definition.
        \item $A \cup B = A \cup B \cup \emptyset \cup \emptyset \ldots \in \mathcal{F}$ nach Eigenschaften  \circled{b} und \circled{c}. $A \cap B = (A^{c}\cup B ^{c})^{c} \in \mathcal{F}$
        \item $\bigcap_{i=1}^{\infty}A_i = \left( \bigcup_{i=1}^{\infty}A_i^{c} \right) ^{c}\in \mathcal{F}$ nach \circled{b} und \circled{c}.
    \end{enumerate}
\end{proof}

Wir haben nun $(\Omega, \mathcal{F})$ näher untersucht, es fehlt nun noch $\mathbb{P}$. 
\begin{question}
    Welche Eigenschaften soll $\mathbb{P}$ (das Wahrscheinlichkeitsmaß bzw. die Wahrscheinlichkeitsverteilung) besitzen?
\end{question}
Seien $A,B \in \mathcal{F}$ mit $A\cap B = \emptyset$, d.h. $A$ und $B$ können nicht gleichzeitig eintreten. Dann fordern wir
\[
    \mathbb{P}(A \cap B) = \mathbb{P}(A) + \mathbb{P}(B) \quad \text{(endliche Additivität)}
.\] 
Dazu wollen wir, dass $\Omega \in \mathcal{F}$ immer eintritt, d.h. $\mathbb{P}(\Omega) = 1 \equiv  100\%$ (Normierung).

\begin{definition}[Wahrscheinlichkeitsverteilung]\label{def:wahrscheinlichkeitsverteilung}
Sei $(\Omega, \mathcal{F})$ ein Messraum. Eine Abbildung $\mathbb{P} : \mathcal{F} \to  \R_+$ ist eine \vocab{Wahrscheinlichkeitsverteilung} auf $(\Omega, \mathcal{F})$, falls
    \begin{enumerate}[(1)]
        \item $\mathbb{P}(\Omega) = 1$
        \item Sind $(A_i)_{i \in I}\in \mathcal{F}$ paarweise disjunkt, so ist:
            \[
                \mathbb{P}\left( \bigcup_{i=1}^{\infty}A_i \right) = \sum_{i=1}^{\infty} \mathbb{P}(A_i) \quad (\sigma\text{-Additivität})
            .\] 
    \end{enumerate}
\end{definition}
\begin{remark*}
    Die Definition macht implizit Gebrauch davon, dass die linke Seite überhaupt definiert ist. Dies folgt jedochdaraus, dass $\mathcal{F}$ eine $\sigma$-Algebra ist.
\end{remark*}

\begin{definition}[Wahrscheinlichkeitsraum]\label{def:wahrscheinlichkeitsraum}
    Ein \vocab{Wahrscheinlichkeitsraum $(\Omega, \mathcal{F},\mathbb{P})$} besteht aus einer Menge $\Omega$, einer  $\sigma$-Algebra $F\subset \mathbb{P}\mathcal{(\Omega)}$ und einem Wahrschenilichkeitsmass $\mathbb{P}$ auf $(\Omega, \mathcal{F})$ 
\end{definition}


\begin{lemma}\label{lm:weitere-eigenschaften-eines-wahrscheinlichkeitsraums}
    Sei $(\Omega, \mathcal{F}, \mathbb{P})$ ein Wahrscheinlichkeitsraum. Dann ist
\begin{enumerate}[label=\protect\circled{\alph*}]
    \item $\mathbb{P}(\emptyset)=0$ 
    \item $\forall A,B\in \mathcal{F}$ mit $A\cap B = \emptyset$ ist
        \[
            \mathbb{P}(A\cup B ) = \mathbb{P}(A) + \mathbb{P}(B)
        .\] 
    \item      $\forall A,B\in \mathcal{F}$ mit $A\subset B$ ist 
        \begin{equation*}
            \begin{split}
                \mathbb{P}(B) &= \mathbb{P}(A) + \mathbb{P}(B \setminus A)  \\
                \mathbb{P}(A^{c}) &= 1 - \mathbb{P}(A) \\
                \mathbb{P}(A) &\leq  \mathbb{P}(B) \leq  1
            \end{split}
        \end{equation*}
    \item $\forall A,B \in \mathcal{F}$ ist
        \begin{equation*}
            \begin{split}
                \mathbb{P}(A \cup B) &= \mathbb{P}(A) + \mathbb{P}(B) - \mathbb{P}(A\cap B) \\
                                     &\leq  \mathbb{P}(A) + \mathbb{P}(B)
            \end{split}
        \end{equation*}
    \item Wenn $A_n \nearrow A$ ,d.h. $A_1\subset A_2\subset \ldots$ mit $\bigcup_{i =1}^{\infty} A_i = A$ (monotone Konvergenz von Mengen), oder $A_n \searrow A$ (d.h.  $A_1\supset A_2 \supset \ldots$ mit $\bigcap_{i=1}^{\infty} A_i = A $ ), so ist
        \[
            \lim_{n \to \infty} \mathbb{P}(A_n) = \mathbb{P}\left( \lim_{n \to \infty} A_n \right)  = \mathbb{P}(A)
        .\] 
\end{enumerate}
\end{lemma}
\begin{proof}
    \begin{enumerate}[label=\protect\circled{\alph*}]
        \item Wir wissen:
            \[
                1= \mathbb{P}(\Omega) = \mathbb{P}\left( \Omega \cup \emptyset \cup \emptyset \cup \emptyset \ldots  \right)  = \mathbb{P}(\Omega) + \mathbb{P}(\emptyset) + \mathbb{P}(\emptyset) + \ldots
            .\] 
            subtrahieren von $\mathbb{P}(\Omega) =1$ liefert dann $\mathbb{P}(\emptyset) = 0$.
        \item Sei $A \cap B = \emptyset$, dann ist:
            \begin{equation*}
                \begin{split}
                    \mathbb{P}(A\cup B ) &= \mathbb{P}(A \cup B \cup \emptyset \cup \emptyset \cup \ldots) \\
                                         &\stackrel{σ-\text{Additivität}}{=} \mathbb{P}(A) + \mathbb{P}(B) + \mathbb{P}(\emptyset) + \mathbb{P}(\emptyset) + \ldots \\
                                         &= \mathbb{P}(A) + \mathbb{P}(B)
                \end{split}
            \end{equation*}
        \item Sei $A\subset B$. Dann ist $B = A \cup (B \setminus A)$ eine disjunkte Vereinigung, also erhalten wir
            \[
                \mathbb{P}(B) = \mathbb{P}(A) + \underbrace{\mathbb{P}(B \setminus A)}_{\geq 0} \geq  \mathbb{P}(A)
            .\] 
            Mit $B = \Omega$ ergibt sich  $1 = \mathbb{P}(A) + \mathbb{P}(A^{c})$
        \item Es ist
            \begin{equation}
                \begin{split}
                    \mathbb{P}(A \cup B) &= \mathbb{P}(A) + \mathbb{P}((A \cup B) \setminus A)  \\
                                         &= \mathbb{P}(A) + \mathbb{P}(B \setminus (A \cap B)) \\
                                         &= \mathbb{P}(A) + \mathbb{P}(B) - \underbrace{\mathbb{P}(A \cap B)}_{\geq 0} \\
                                         &\geq \mathbb{P}(A) + \mathbb{P}(B)
                \end{split}
            \end{equation}
        \item Übung
    \end{enumerate}
\end{proof}
\begin{corollary}[Einschluss-Ausschluss-Prinzip]\label{cor:einschluss-ausschluss-prinzip}
    Seien $A_1,\ldots,A_n \in \mathcal{F}$. Dann gilt
    \[
        \mathbb{P}(A_1 \cup \ldots \cup A_n) = \sum_{k=1}^{n} (-1)^{k-1} \sum_{1\leq i_1<i_2<\ldots<i_k \leq n} \mathbb{P}(A_{i_1} \cap A_{i_2} \cap \ldots \cap A_{i_k})
    .\] 
\end{corollary}
\begin{proof}
    Per Induktion, der Induktionsanfang lautet  $\mathbb{P}(A_1) = \mathbb{P}(A_1)$ und ist offensichtlich wahr. \\
    Die Aussage gelte nun für ein $n\in \N$, dann erhalten wir
    \begin{equation}
        \begin{split}
            \mathbb{P}\left( \bigcup_{i=1}^{n+1} A_i \right)  &= \mathbb{P}\left( \left(\bigcup_{i=1}^{n}A_i \right) \cup A_{n+1}\right)  \\
                                                              &= \mathbb{P}\left(\bigcup_{i=1}^{n} A_i\right) + \mathbb{P}(A_{n+1}) - \mathbb{P}\left( \left( \bigcup_{i=1}^n A_i \right) \cap A_{n+1} \right)  \\
                                                              &= \mathbb{P}\left( \bigcup_{i=1}^{n} A_i \right)  + \mathbb{P}(A_{n+1}) - \mathbb{P}\left( \bigcup_{i=1}^n \underbrace{(A_i \cap _{A_{n+1}})}_{=: \tilde{A}_i} \right)  \\
                                                              &= \sum_{k=1}^{n} (-1)^{k-1} \sum_{1\leq i<\ldots<i_k \leq n} \mathbb{P}(A_{i_1} \cap \ldots \cap A_{i_k}) + \mathbb{P}(A_{n+1}) \\
                                                              &\qquad -\sum_{k=1}^n (-1)^{k-1} \sum_{1\leq i_1 < \ldots < i_k \leq n} \mathbb{P}(\underbrace{\tilde{A}_{i_1} \cap \ldots \cap \tilde{A}_{i_k}}_{A_{i_1} \cap \ldots \cap A_{i_k} \cap A_{n+1}})
        \end{split}
    \end{equation}
    Andererseits ist aber auch:
    \begin{alignat*}{5}
           &\quad&  &\sum_{k=1} ^{n+1} (-1)^{k-1} \sum_{1\leq i_1<\ldots<i_k \leq  n+1} \mathbb{P}(A_{i_1} \cap \ldots \cap A_{i_k}) &\quad & \\
           &= & &\sum\limits_{k=1}^{n}(-1)^{k-1} \sum\limits_{1\leq i_1< \ldots < i_k \leq  \color{red} n} \mathbb{P}(A_{i_1} \cap \ldots \cap A_{i_k}) & \quad &\Big\}\text{\parbox{2cm}{Terme mit $i_k\leq n$}} \\
           &+& &\underbrace{\sum_{k=2}^{n+2} (-1)^{k-1} \sum_{1\leq i_1<\ldots<i_{k-1}\leq n} \mathbb{P}(A_i \cap \ldots \cap A_{i_{k-1}} \cap A_{n+1})}_{\stackrel{l := k-1}{=} -\sum\limits_{l=1}^n (-1)^{l-1}\sum\limits_{1\leq i_1<...<i_l\leq n} \mathbb{P}(A_{i_1} \cap \ldots \cap A_{i_l} \cap A_{n+1})}& \quad &\Bigg\}\text{\parbox{2cm}{\small Terme mit $i_k = n+1$ und  $k\geq 2$}}\\
           &+& & \mathbb{P}(A_{n+1}) \qquad& \quad  &\Big\}\text{\parbox{2cm}{\small Terme mit $i_k = n+1$ und  $k=1$}}
    \end{alignat*}
    und damit sehen wir, dass die beiden Ausdrücke übereinstimmen, also ist der Induktionsschritt erbracht.
\end{proof}


\subsection{Diskrete Verteilungen}
\begin{itemize}
    \item Sei nun $\Omega$ endlich oder abzählbar.
    \item Falls wir $\mathcal{F}$ nicht explizit angeben, dann wird $\mathcal{F} = \mathcal{P}(\Omega)$ gewählt, d.h.
        \[
            \operatorname{Card} (\mathcal{P}(\Omega)) \equiv \abs{\mathcal{P}(\Omega)} = 2 ^{ \abs{\Omega}} 
        .\] 
\end{itemize}
\begin{example}[Münzwurf]
    Es sei $\Omega = \left \{K,Z\right\}$, wobei $K$ für Kopf stehe und $Z$ für Zahl. Dann ist
    \[
    \mathcal{F} = \left \{\left \{K\right\} ,\left \{Z\right\} ,\left \{Z,K\right\} ,\emptyset\right\} 
    .\] 
    Sei $p\in [0,1]$ die Wahrscheinlichkeit, dass man Kopf erhält. Da $\mathbb{P}$ für alle Element aus $\mathcal{F}$ definiert sein muss, erhalten wir
    \[
        \mathbb{P}(\emptyset) = 0 \qquad \mathbb{P}(K) = p, \qquad \mathbb{P}(Z) = \mathbb{P}(K^{c}) = 1-p \qquad \mathbb{P}(\left \{Z,K\right\} ) = \mathbb{P}(\Omega) = 1
    .\] 
\end{example}
\begin{question}[Charakterisierung von diskreter Wahrscheinlichkeit]
Was müssen wir fordern, sodass $\mathbb{P}$ auf $\mathcal{P}(\Omega)$ gibt?.
\end{question}
\begin{example}
    $\Omega = \left \{1,2,\ldots,10\right\}$ würde genügen, da dann $\abs{\mathcal{P}(\Omega)}= 2^{\abs{\Omega} }=2^{10} = 1024 $
    endlich (diskret) ist.
\end{example}

    % end lectures
    %\input{fragestunden.tex}
\end{document}
