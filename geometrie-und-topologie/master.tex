\makeatletter
% patched command of loading a package
\def\find@load#1[#2]#3[#4]{%
    % load the package
    \find@fileswith@pti@ns{#1}[#2]{#3}[#4]%
    % check if commands exists now
    \ifcsname\find@command\endcsname
        \typeout{Package #3 introduces command \find@command.}%
        % command has been found, revert to original version without checks
        \let\@fileswith@pti@ns\find@fileswith@pti@ns
    \else
        % somehow, \@fileswith@pti@ns is restored after loading a package
        % thus, patch it again
        \let\@fileswith@pti@ns\find@load
    \fi
}%

\newcommand*{\findpackagebycommand}[1]{%
    % using this multiple times - esp. when the package has not been found yet -
    % will break things. Thus, check first that \find@command has never been defined before
    \ifx\find@command\undefined
        \def\find@command{#1}%
        % first, check if this command is already defined
        \ifcsname\find@command\endcsname
            % in this case, just issue a warning and do nothing
            \@latex@warning@no@line{Command \find@command \space is already defined}%
        \else
            % overwrite the internal \@fileswith@pti@ns command, which does the actual loading
            % \@fileswith@pti@ns is used internally by \usepackage and \RequirePackage
            \let\find@fileswith@pti@ns\@fileswith@pti@ns
            \let\@fileswith@pti@ns\find@load
        \fi
    \else
        % used multiple times - prevent and give a warning
        \@latex@warning@no@line{You can use \protect\findpackagebycommand \space only once}%
        \errmessage{Invalid use of command findpackagebycommand.}%
    \fi
}

\makeatother


\findpackagebycommand{aside}



\documentclass[a4paper, german, lecturenumbers = true, number small environments = theorem, hide version]{mkessler-script}

\course{Einführung in die Geometrie und Topologie}
\lecturer{Daniel Kasprowski}
\assistant[f]{Arunima Ray}
\author{Maximilian Keßler}

\RequirePackage{mkessler-math}
\RequirePackage{mkessler-fancythm}
\usepackage{epsfig}
%\usepackage{psfrag}
%\usepackage{sseq} (if you need to draw spectral sequences, please use this package, available at http://wwwmath.uni-muenster.de/u/tbauer/)
\usepackage{mathrsfs}
\usepackage{amscd}
\usepackage{amsbsy}
\usepackage{verbatim}
\usepackage{moreverb}

\newtheorem{prop}[theorem]{Proposition}
\newtheorem{cor}[theorem]{Corollary}
\newtheorem{conj}[theorem]{Conjecture}


\theoremstyle{definition}
\newtheorem{hw}{Homework}
\newtheorem{exercise*}[exercise]{$\star$ Exercise}
\newtheorem{aufgabe}{Aufgabe}

\theoremstyle{remark}
\newtheorem{aside}[theorem]{Aside}

\newcommand{\nn}{\nonumber}
\newcommand{\nid}{\noindent}
\newcommand{\ra}{\rightarrow}
\newcommand{\la}{\leftarrow}
\newcommand{\xra}{\xrightarrow}
\newcommand{\xla}{\xleftarrow}
\newcommand{\tto}{\longrightarrow}

\newcommand{\weq}{\xrightarrow{\sim}}
\newcommand{\cofib}{\rightarrowtail}
\newcommand{\fib}{\twoheadrightarrow}

\newcommand{\IRep}{\mathrm{IRep}}
\newcommand{\IHom}{\mathrm{IHom}}

\def\llarrow{   \hspace{.05cm}\mbox{\,\put(0,-2){$\leftarrow$}\put(0,2){$\leftarrow$}\hspace{.45cm}}}
\def\rrarrow{   \hspace{.05cm}\mbox{\,\put(0,-2){$\rightarrow$}\put(0,2){$\rightarrow$}\hspace{.45cm}}}
\def\lllarrow{  \hspace{.05cm}\mbox{\,\put(0,-3){$\leftarrow$}\put(0,1){$\leftarrow$}\put(0,5){$\leftarrow$}\hspace{.45cm}}}
\def\rrrarrow{  \hspace{.05cm}\mbox{\,\put(0,-3){$\rightarrow$}\put(0,1){$\rightarrow$}\put(0,5){$\rightarrow$}\hspace{.45cm}}}

\def\cA{\mathcal A}\def\cB{\mathcal B}\def\cC{\mathcal C}\def\cD{\mathcal D}
\def\cE{\mathcal E}\def\cF{\mathcal F}\def\cG{\mathcal G}\def\cH{\mathcal H}
\def\cI{\mathcal I}\def\cJ{\mathcal J}\def\cK{\mathcal K}\def\cL{\mathcal L}
\def\cM{\mathcal M}\def\cN{\mathcal N}\def\cO{\mathcal O}\def\cP{\mathcal P}
\def\cQ{\mathcal Q}\def\cR{\mathcal R}\def\cS{\mathcal S}\def\cT{\mathcal T}
\def\cU{\mathcal U}\def\cV{\mathcal V}\def\cW{\mathcal W}\def\cX{\mathcal X}
\def\cY{\mathcal Y}\def\cZ{\mathcal Z}

\def\sA{\mathscr A}\def\cB{\mathcal B}\def\cC{\mathcal C}\def\cD{\mathcal D}
\def\cE{\mathcal E}\def\cF{\mathcal F}\def\sG{\mathscr G}\def\cH{\mathcal H}
\def\cI{\mathcal I}\def\cJ{\mathcal J}\def\cK{\mathcal K}\def\cL{\mathcal L}
\def\cM{\mathcal M}\def\cN{\mathcal N}\def\cO{\mathcal O}\def\cP{\mathcal P}
\def\cQ{\mathcal Q}\def\cR{\mathcal R}\def\cS{\mathcal S}\def\cT{\mathcal T}
\def\cU{\mathcal U}\def\cV{\mathcal V}\def\cW{\mathcal W}\def\cX{\mathcal X}
\def\cY{\mathcal Y}\def\cZ{\mathcal Z}

\def\fG{\mathfrak G}\def\fH{\mathfrak H}
\def\fS{\mathfrak S}\def\fN{\mathfrak N}\def\fX{\mathfrak X}\def\fY{\mathfrak Y}

\def\op{\textrm{op}}\def\ob{\textrm{ob}}

%\def\Iso{\mathcal Iso}\def\cInn{\mathcal Inn}

\def\fg{\mathfrak g}\def\fh{\mathfrak h}\def\fri{\mathfrak i}\def\fp{\mathfrak p}
\def\fA{\mathfrak A}\def\fU{\mathfrak U}

\def\AA{\mathbb A}\def\BB{\mathbb B}\def\CC{\mathbb C}\def\DD{\mathbb D}
\def\EE{\mathbb E}\def\FF{\mathbb F}\def\GG{\mathbb G}\def\HH{\mathbb H}
\def\II{\mathbb I}\def\JJ{\mathbb J}\def\KK{\mathbb K}\def\LL{\mathbb L}
\def\MM{\mathbb M}\def\NN{\mathbb N}\def\OO{\mathbb O}\def\PP{\mathbb P}
\def\QQ{\mathbb Q}\def\RR{\mathbb R}\def\SS{\mathbb S}\def\TT{\mathbb T}
\def\UU{\mathbb U}\def\VV{\mathbb V}\def\WW{\mathbb W}\def\XX{\mathbb X}
\def\YY{\mathbb Y}\def\ZZ{\mathbb Z}

\def\TOP{\mathcal{TOP}}\def\GRP{\mathcal{GRP}}\def\GRPD{\mathcal{GRPD}} \def\CAT{\mathcal{CAT}} \def\SET{\mathcal{SET}}

\def\id{\mathrm{id}}\def\Id{\mathrm{Id}}
\def\inverse{^{-1}}



\begin{document}
    \maketitle
    \begin{abstract}
    {\color{red} Achtung:} Diese Version des Skripts benutze ich zur Bearbeitung! Einige Dinge fehlen, dafür gibt es TODO-Notes. Für Inhalte, benutzt die \href{https://kesslermaximilian.github.io/LectureNotesBonn/2021_Topologie.pdf}{normale Version}
    \end{abstract}
    \newpage
    \listoftodos
    \newpage
    \summaryoflectures
    \newpage
    % start lectures
    \setcounter{section}{20}
    \setcounter{dummy}{8}
    \setcounter{smalldummy}{0}
    \setcounter{figure}{29}
    \setcounter{claim}{1}
    \setcounter{lecture}{22}
    %! TEX root = ./master.tex
\lecture[]{Di 13 Jul 2021 12:12}{Beweis des Satzes von Seifert-van-Kampen}

\begin{remark*}
    An dieser Stelle haben wir nun den Beweis von \autoref{thm:seifert-van-kampen} gemacht. Aus zeitlichen Gründen ist der noch nicht aufgeschrieben, es fehlt aber auch \textit{nur} der Beweis, danach geht es normal weiter.
\end{remark*}


\begin{theorem}[allgemeine Version von Seifert-van-Kampen]\label{thm:seifert-van-kampen-allgemein}
    Sei $X$ ein Raum,  $x\in X$, und $\mathcal{U} = \left \{U_α\right\} _{α\in I}$ eine offene Überdeckung von $X$ mit  $x\in U_{α}$ für jedes $α\in I$. Sei für $α,β\in I$ stets $U_α \cap  U_β$ wegzusammenhängend. Die inklusionen
    \[
        \left \{ι_α\colon  \pi_1(U_α,x) \to  \pi_1(X,x)\right\}
    .\] 
    induzieren eine Abbildung
    \[
        \psi \colon  \coprod \pi_1(U_α, x) \twoheadrightarrow \pi_1(X,x)
    .\] 
    Dann gilt:
    \begin{enumerate}[1)]
        \item $\psi $ ist surjektiv.
        \item Falls darüber hinaus $U_α \cap  U_β \cap  U_γ$ wegzusammenhängend für $α,β,γ\in I$, dann ist der Kern von $\psi $ der Normalteiler erzeugt von
            \[
                \left \{ι_{α,β}(\omega ) ι_{β,α}(\omega )^{-1} \mid  \omega \in \pi_1(U_α \cap U_β, x), α,β\in I\right\} 
            .\] 
    \end{enumerate}
    wobei
    \begin{IEEEeqnarray*}{rCl}
        ι_{α,β}\colon  U_α \cap  U_β \hookrightarrow U_α \\
        ι_{β,α}\colon  U_β \cap  U_α \hookrightarrow U_β
    \end{IEEEeqnarray*}
    die Inklusionen sind.
\end{theorem}

\begin{oral}
    Man kann das im Wesentlichen aus dem üblichen Satz von \nameref{thm:seifert-van-kampen}   folgern, indem man induktiv auf endlich viele Mengen verallgemeinert. Für unendliche Mengen macht man dann ein Kolimes-Argument, indem man ausnutzt, dass wegen Kompaktheit jeder Weg und jede Homotopie schon in endlichen vielen der $U_i$ liegen muss.

    Für einen vollständigen Beweis siehe  \cite[Satz 1.20]{algebraic-topology-hatcher}.
\end{oral}

Wir werden diese allgemeine Version nicht beweisen, wir begnügen uns mit einigen Berechnungen.

\subsection{Berechnung der Fundamentalgruppe eines CW Komplexes}

\begin{theorem}[Graphen]\label{thm:fundamentalgruppe-von-graphen}
    Sei $X$ ein wegzusammenhängender Graph, und  $x_0\in X^0$, d.h.
    \[
    X = \underbrace{\bigcup_{i \in  I^0} e_i^0 }_{X^0} \cup \underbrace{\bigcup_{i \in  I^1} e_i^1
}_{1-\text{Zellen}} 
    .\]
    \begin{enumerate}[(i)]
        \item Es existiert $J\subset I^1$, so dass
            \[
            Y \coloneqq  X^0 \cup \bigcup_{j\in J} e_j^1
            .\] 
            ein maximaler \vocab{Baum} (ein zusammenziehbarer Graph, kombinatorisch: ein kreisfreier, zusammenhängender Graph) in $X$ ist. 
        \item Für $J$ wie in  $(i)$ gilt
             \[
                 \pi_1(X,x_0) \cong \coprod_{i\in I^1 \setminus J} \Z \cong \star_{i\in I^1 \setminus J} \Z
            .\] 
    \end{enumerate}
\end{theorem}

\begin{oral}
    Auch hier begnügen wir uns mit Beispielen, und führen keinen Beweis.
\end{oral}

\begin{example}
    \begin{itemize}
        \item 
    Betrachte folgenden Graph:
    \missingfigure{Graph mit Baum}
    der grün markierte Teilgraph ist nun ein Baum. Für die Berechnung der Fundamentalgruppe ziehen wir nun den Baum zusammen, also erhalten wir noch:
    \[
       X \simeq Y \coloneqq 
       \begin{tikzpicture}[baseline = -0.2]
        \fill (0,0) circle (2pt);
        \draw (0,0) .. controls (1,0) and (0,1) .. (0,0);
        \draw (0,0) .. controls (0,1) and (-1,0) .. (0,0);
        \draw (0,0) .. controls (-1,-1) and (1,-1) .. (0,0);
    \end{tikzpicture}
\]
Der zusammengezogene Raum $Y$ ist dann tatsächlich homotop zum ursprünglichen Raum $X$, und  $Y$ ist nun frei mit so vielen Erzeugern, wie es Kanten in  $I^1 \setminus J$ gibt.
\item Betrachte  $Z$
    \[
    \begin{tikzcd}
        \arrow[loop left]{a} \ar{r}{b} x & y \ar[loop right]{c}
    \end{tikzcd}
\]
Dann ist $\pi_1(Z) = \left< α,β \mid  \right> $. Die Erzeuger sind gegeben durch $a, bcb^{-1}$
\item Ist $Z'$ der folgende Graph:
    \missingfigure{Graph}
    So sind die Erzeuger gegeben durch $a,bc b^{-1}, bd$
    \end{itemize}
\end{example}

    % end lectures
\end{document}
