\documentclass[a4paper, german, lecturenumbers = true, number small environments = theorem, hide version]{mkessler-script}

\course{Einführung in die Geometrie und Topologie}
\lecturer{Daniel Kasprowski}
\assistant[f]{Arunima Ray}
\author{Maximilian Keßler}

\RequirePackage{mkessler-math}
\RequirePackage{mkessler-fancythm}
\usepackage{epsfig}
%\usepackage{psfrag}
%\usepackage{sseq} (if you need to draw spectral sequences, please use this package, available at http://wwwmath.uni-muenster.de/u/tbauer/)
\usepackage{mathrsfs}
\usepackage{amscd}
\usepackage{amsbsy}
\usepackage{verbatim}
\usepackage{moreverb}

\newtheorem{prop}[theorem]{Proposition}
\newtheorem{cor}[theorem]{Corollary}
\newtheorem{conj}[theorem]{Conjecture}


\theoremstyle{definition}
\newtheorem{hw}{Homework}
\newtheorem{exercise*}[exercise]{$\star$ Exercise}
\newtheorem{aufgabe}{Aufgabe}

\theoremstyle{remark}
\newtheorem{aside}[theorem]{Aside}

\newcommand{\nn}{\nonumber}
\newcommand{\nid}{\noindent}
\newcommand{\ra}{\rightarrow}
\newcommand{\la}{\leftarrow}
\newcommand{\xra}{\xrightarrow}
\newcommand{\xla}{\xleftarrow}
\newcommand{\tto}{\longrightarrow}

\newcommand{\weq}{\xrightarrow{\sim}}
\newcommand{\cofib}{\rightarrowtail}
\newcommand{\fib}{\twoheadrightarrow}

\newcommand{\IRep}{\mathrm{IRep}}
\newcommand{\IHom}{\mathrm{IHom}}

\def\llarrow{   \hspace{.05cm}\mbox{\,\put(0,-2){$\leftarrow$}\put(0,2){$\leftarrow$}\hspace{.45cm}}}
\def\rrarrow{   \hspace{.05cm}\mbox{\,\put(0,-2){$\rightarrow$}\put(0,2){$\rightarrow$}\hspace{.45cm}}}
\def\lllarrow{  \hspace{.05cm}\mbox{\,\put(0,-3){$\leftarrow$}\put(0,1){$\leftarrow$}\put(0,5){$\leftarrow$}\hspace{.45cm}}}
\def\rrrarrow{  \hspace{.05cm}\mbox{\,\put(0,-3){$\rightarrow$}\put(0,1){$\rightarrow$}\put(0,5){$\rightarrow$}\hspace{.45cm}}}

\def\cA{\mathcal A}\def\cB{\mathcal B}\def\cC{\mathcal C}\def\cD{\mathcal D}
\def\cE{\mathcal E}\def\cF{\mathcal F}\def\cG{\mathcal G}\def\cH{\mathcal H}
\def\cI{\mathcal I}\def\cJ{\mathcal J}\def\cK{\mathcal K}\def\cL{\mathcal L}
\def\cM{\mathcal M}\def\cN{\mathcal N}\def\cO{\mathcal O}\def\cP{\mathcal P}
\def\cQ{\mathcal Q}\def\cR{\mathcal R}\def\cS{\mathcal S}\def\cT{\mathcal T}
\def\cU{\mathcal U}\def\cV{\mathcal V}\def\cW{\mathcal W}\def\cX{\mathcal X}
\def\cY{\mathcal Y}\def\cZ{\mathcal Z}

\def\sA{\mathscr A}\def\cB{\mathcal B}\def\cC{\mathcal C}\def\cD{\mathcal D}
\def\cE{\mathcal E}\def\cF{\mathcal F}\def\sG{\mathscr G}\def\cH{\mathcal H}
\def\cI{\mathcal I}\def\cJ{\mathcal J}\def\cK{\mathcal K}\def\cL{\mathcal L}
\def\cM{\mathcal M}\def\cN{\mathcal N}\def\cO{\mathcal O}\def\cP{\mathcal P}
\def\cQ{\mathcal Q}\def\cR{\mathcal R}\def\cS{\mathcal S}\def\cT{\mathcal T}
\def\cU{\mathcal U}\def\cV{\mathcal V}\def\cW{\mathcal W}\def\cX{\mathcal X}
\def\cY{\mathcal Y}\def\cZ{\mathcal Z}

\def\fG{\mathfrak G}\def\fH{\mathfrak H}
\def\fS{\mathfrak S}\def\fN{\mathfrak N}\def\fX{\mathfrak X}\def\fY{\mathfrak Y}

\def\op{\textrm{op}}\def\ob{\textrm{ob}}

%\def\Iso{\mathcal Iso}\def\cInn{\mathcal Inn}

\def\fg{\mathfrak g}\def\fh{\mathfrak h}\def\fri{\mathfrak i}\def\fp{\mathfrak p}
\def\fA{\mathfrak A}\def\fU{\mathfrak U}

\def\AA{\mathbb A}\def\BB{\mathbb B}\def\CC{\mathbb C}\def\DD{\mathbb D}
\def\EE{\mathbb E}\def\FF{\mathbb F}\def\GG{\mathbb G}\def\HH{\mathbb H}
\def\II{\mathbb I}\def\JJ{\mathbb J}\def\KK{\mathbb K}\def\LL{\mathbb L}
\def\MM{\mathbb M}\def\NN{\mathbb N}\def\OO{\mathbb O}\def\PP{\mathbb P}
\def\QQ{\mathbb Q}\def\RR{\mathbb R}\def\SS{\mathbb S}\def\TT{\mathbb T}
\def\UU{\mathbb U}\def\VV{\mathbb V}\def\WW{\mathbb W}\def\XX{\mathbb X}
\def\YY{\mathbb Y}\def\ZZ{\mathbb Z}

\def\TOP{\mathcal{TOP}}\def\GRP{\mathcal{GRP}}\def\GRPD{\mathcal{GRPD}} \def\CAT{\mathcal{CAT}} \def\SET{\mathcal{SET}}

\def\id{\mathrm{id}}\def\Id{\mathrm{Id}}
\def\inverse{^{-1}}



\begin{document}
    \maketitle
    \begin{abstract}
    {\color{red} Achtung:} Diese Version des Skripts benutze ich zur Bearbeitung! Einige Dinge fehlen, dafür gibt es TODO-Notes. Für Inhalte, benutzt die \href{https://kesslermaximilian.github.io/LectureNotesBonn/2021_Topologie.pdf}{normale Version}
    \end{abstract}
    \newpage
    \listoftodos
    \newpage
    \summaryoflectures
    \newpage
    % start lectures
    \setcounter{section}{13}
    \setcounter{dummy}{16}
    \setcounter{smalldummy}{3}
    \setcounter{figure}{15}
    \setcounter{claim}{2}
    \setcounter{lecture}{12}
    %! TEX root = ./master.tex
\lecture[]{Mi 02 Jun 2021 10:15}{}

\subsection{Gleichgewicht von Markovketten}
Oft ist $\mu$ nicht explizit bekannt, z.B. wenn
\[
\mu = \lim_{n \to \infty} \mu_0 P^n
.\] 
wobei $P$ die Übergangsmatrix einer Markovkette ist.
 \begin{oral}
     Selbst wenn der obige Limes existiert und eindeutig ist (d.h. nicht von $\mu_0$ abhängt), heißt das nicht, dass wir eine explizite Formel für  $\mu$ kennen. Allerdings können wir natürlich approximieren, indem wir mit einem  $\mu_0$ starten und  $\mu_0P^n$ für ein hinreichend großes  $n$ bestimmen.
\end{oral}

Betrachten wir eine \underline{homogene Markovkette} mit Übergangsmatrix $P$ und Anfangsverteilung $\mu_0$.
\begin{definition}[Stationäre Verteilung]\label{def:stationäre-verteilung}
    \begin{enumerate}[label=\protect\circled{\alph*}]
        \item Eine Wahrscheinlichkeitsverteilung $\mu$ auf  $\mathcal{S}$ ist eine \vocab{stationäre Verteilung}  einer Markovkette mit Übergangsmatrix $P$, falls  $μ = μP$. 
        \item $\mu$ erfüllt die \vocab{Detailed-Balance-Bedingung} bezüglich $P$, falls
            \[
                \mu(x) P(x,y) = \mu(y) P(y,x) \qquad \forall x,y \in \mathcal{S}
            .\] 
    \end{enumerate}
\end{definition}
\missingfigure{Gleichgewicht zwischen $x,y$ durch  $\mu(x) P(x,y) \equiv $ Massenflus von $x$ nach $y$}


\begin{theorem}
    Falls $\mu$ die Detailed-Balance-Bedingung erfüllt, so ist  $\mu$ stationär.
\end{theorem}

\begin{proof}
    Es ist
    \begin{IEEEeqnarray*}{rCl}
        (\mu P)(x)& = &\sum_{y\in \mathcal{S}} \mu(y) P(y,x)\\
                  &\stackrel{\text{Detailed Balance}}{=} & \sum_{y\in \mathcal{S}} \mu(x) P(x,y) \\
                  & = & \mu(x) \underbrace{\sum_{y\in \mathcal{S}} P(x,y)}_{=1 (P  \text{ stochastisch})} \\
                  & = & \mu(x)
    \end{IEEEeqnarray*}
\end{proof}

\begin{warning}
    $\mu$ stationär  $\not \implies$ $\mu$ erfüllt die Detailed Balance Bedingung
\end{warning}

\begin{example}
    Sei $\mathcal{S} = \left \{1,2,3\right\} $ und $p\in \left( \frac{1}{2},1 \right) $.
    \[
    \begin{tikzcd}
        \circled{1} \ar[bend left = 20, blue]{rr}{p}& & \circled{2} \ar[bend left = 20, green!70!black]{ll}{1-p} \\
                                                    & \circled{3} \arrow[blue,swap]{ul}{p} \arrow[green!70!black]{ur}{1-p}
    \end{tikzcd}
    .\] 
    also
    \[
        P = \begin{pmatrix} 0 & p & 1-p \\ 1-p & 0 & p \\ p & 1-p & 0 \end{pmatrix}  
    .\] 
    Dann ist $\mu = \left( \frac{1}{3},\frac{1}{3},\frac{1}{3} \right) $ eine stationäre Verteilung, wie man leicht prüft (Syemmtriegründe oder einfaches Nachrechnen). Allerdings ist 
    \[
        \mu(1) P(1,2) = \frac{1}{3}p \neq  \frac{1}{3}(1-p) = \mu(2)P(2,1)
    .\] 
    also erfüllt $\mu$  \underline{nicht} die Detailed-Balance-Bedingung. 
\end{example}
\todo{Ordentliches Diagramm}

\subsection{Konvergenz ins Gleichgewicht}
Um Konvergenz messen zu können, brauchen  wir einen Abstandsbegriff für Wahrscheinlichkeitsverteilungen. Sei hierzu
\[
    \mathcal{M}(\mathcal{S}) \coloneqq  \left \{\mu = (\mu(x))_{x\in \mathcal{S}} \mid  \mu(x) \geq 0 \; \forall x, \sum_{x\in \mathcal{S}} \mu(x) = 1\right\} 
.\] 
der Raum aller Wahrscheinlichkeitsverteilungen.

\begin{definition}
    Die \vocab{totale Variatonsdistanz} zweier Wahrscheinlichkeitsverteilungen $\mu,\nu$ auf  $\mathcal{S}$ ist defniert durch:
    \begin{IEEEeqnarray*}{rCl}
        d_{TV}(\mu,\nu) &\coloneqq  &\frac{1}{2} \lVert \mu - \nu \rVert _1 \\
                        & = & \frac{1}{2} \sum_{x\in \mathcal{S}} \abs{\mu(x) - \nu(x)} 
    \end{IEEEeqnarray*}
\end{definition}

\begin{remark}
    \begin{enumerate}[label=\protect\circled{\alph*}]
        \item $d_{TV}$ ist eine Metrik.
        \item  $\forall \mu,\nu \in \mathcal{M}(\mathcal{S})$ ist 
            \[
                d_{TV}(\mu,\nu) \leq  \frac{1}{2} \sum_{x} (\mu(x) + \nu(x)) = 1
            .\] 
    \end{enumerate}
\end{remark}

    %! TEX root = ./master.tex
\lecture[]{Do 10 Jun 2021 10:15}{}

\begin{theorem}[$\pi_1$ auf Wegzusammenhangskomponenten]\label{thm:pi-1-ist-gleich-auf-wegusammenhangskomponente}
    Seien $x,x' \in X$ durch einen Weg verbunden, d.h. $\exists w\colon  I \to X$ mit $w(0) = x$,  $w(1) = x'$. Dann sind die Gruppen  $\pi_1(X,x)$ und $\pi_1(X,x')$ homöomorph.
\end{theorem}

\begin{remark}
    Liegen $x,x'$ in verschiedenen Wegzusammenhangskomponenten, so gibt es (im Allgemeinen) keine Beziehung zwischen  $\pi_1(X,x)$ und $\pi_1(X,x')$. Betrachte hierzu z.B. $X = X_1 \coprod X_2$ mit $x\in X_1$ und $x' \in X_2$, so ist $p_1(X,x) = p_1(X_1,x)$ und $p_1(X,x') = p_1(X_2,x')$. Es genügt daher zu sehen, dass es Räume mit verschiedenen Fundamentalgruppen gibt.
    \missingfigure{Disjunkte Vereinigung von Räumen mit verschiedener Fundamentalgruppe}
\end{remark}

\begin{proof}[Beweis von \autoref{thm:pi-1-ist-gleich-auf-wegusammenhangskomponente}]
    Sei $v\colon  I \to  X$ eine Weg von $x \circ  x'$. Dann definiere
        \begin{equation*}
        v_*: \left| \begin{array}{c c l} 
            \pi_1(X,x') & \longrightarrow & (X,x) \\
            \left[w\right] & \longmapsto &  \left[v \star w \star \overline{v}\right]
        \end{array} \right.
    \end{equation*}
    wobei wie üblich $\overline{v}\colon [0,1] \to X$ gegeben ist durch $\overline{v}(t) = v(1-t)$.
    \missingfigure{Illustration von $v \star w \star \overline{v}$}
    Hierbei definieren wir
    \[
        (v \star w \star \overline{v})(t) = \begin{cases}
            v(3t) & 0 \leq  t \leq  \frac{1}{3}\\
            w(3t-1) & \frac{1}{3}\leq t \leq  \frac{2}{3} \\
            v(3(1-t)) & \frac{2}{3} \leq  t \leq  1
        \end{cases}
    .\] 
    Damit diese Konstruktion ein Gruppenisomorphismus ist, zeigen wir:
    \begin{description}
        \item[Wohldefiniertheit] Für $w \stackrel{H}{\sim } w'$ ist auch $v \star w \star \overline{v} \sim  v \star w' \star \overline{v}$ mittels der Homotopie
            \[
                H(t,s) = \begin{cases}
                    v(3t) & 0\leq t\leq \frac{1}{3} \\
                    H(3t-1,s) & \frac{1}{3}\leq  t \leq  \frac{2}{3} \\
                    v(3(1-t)) & \frac{2}{3}\leq t\leq 1
                \end{cases}
            .\] 
        \item[Gruppenhomomorphismus] Es ist:
            \begin{IEEEeqnarray*}{rCl}
                v \star (w \star w') \star \overline{v} &\stackrel{\overline{v}\star v \sim c_{x'}}{=} &v \star (w \star (\overline{v} \star v) \star w') \star \overline{v} \\
                                                        & \stackrel{\text{assoz.}}{=}  & (v \star w \star \overline{v}) \star (v \star w' \star \overline{v}) \\
                                                        & = & v_*(w) \circ v_*(w')
            \end{IEEEeqnarray*}
        \item[Isomorphismus] $\overline{v}$ induziert analog zu $v$ eine Abbildung
             \[
                 \overline{v}_*\colon  \pi_1(X,x) \to  \pi_1(X,x')
            .\] 
            Wir behaupten, dass $\overline{v}_*$ ein Inverses ist, also $v_* \circ  \overline{v}_* = \id_{\pi_1(X,x)}$, dann folgt wegen $\overline{\overline{v}} = v$ auch sofort $\overline{v}_* \circ  v_ = \id_{\pi_1(X,x')}$. 

            Sei also $[w] \in  \pi_1(X,x)$, dann ist
            \begin{IEEEeqnarray*}{rCl}
                (                v_* \circ  \overline{v}_*)([w]) & = & [v \star \overline{v} \star w \star v \star \overline{v}] \\
                                                                 & = & [w]
            \end{IEEEeqnarray*}
            also ist $v_* \circ  \overline{v}_*$ tatsächlich die Identität.
    \end{description}
\end{proof}

\begin{warning}
    Der Isomorphismus $v_*\colon  \pi_1(X,x') \to  \pi_1(X,x)$ hängt von $v$ ab (genauer von der Homotopieklasse von  $v$ relativ Endpunkten). Die Gruppen sind aber in jedem Fall isomorph.    
\end{warning}
\todo{Frage: gilt hier potenziell die Umkehrung?}
\todo{Nein, schreibe hierzu kleine Bemerkung}

\begin{recap}
    Ist $G$ eine Gruppe,  $g\in G$, so ist
        \begin{equation*}
        c_g: \left| \begin{array}{c c l} 
        G & \longrightarrow & G \\
        h & \longmapsto &  ghg^{-1}
        \end{array} \right.
    \end{equation*}
    eine Automorphismus von $G$, die  \vocab[Gruppe!Konjugation]{Konjugation mit $g$}. Diese heißen \vocab[Gruppe!innerer Automorphismus]{innere Automorphismen} 
\end{recap}

Sind $v,v' \colon  I \to  X$ Wege von $x$ nach  $x'$, dann gilt:
\begin{IEEEeqnarray*}{rCl}
    v_*([w]) &=& v\star [w] \star \overline{v}\\
             & = & [v \star \overline{v'}]  \star v_*'([w]) \star [v' \star \overline{v}]\\& = & c_{[v \star \overline{v'}]} (v_*'[w])
\end{IEEEeqnarray*}

\begin{oral}
    Man muss aufpassen, bei wegzusammenhängenden Räumen einfach von 'der' Fundamentalgruppe zu sprechen. Es gibt zwar nur eine, allerdings sind die Isomorphismen nicht zwingend kanonisch, weswegen man trotzdem implizit noch die Wahl des Basispunktes mit sich rumschleppt. Das motiviert die Betrachtung der nächsten Definition:
\end{oral}


\begin{definition}[Fundamentalgruppoid]\label{def:fundamentalgruppoid}
    Der \vocab{Fundamentalgruppoid} $\Pi(X)$ ist die Kategorie mit  $\ob(\Pi(X)) = X$ und
     \[
         \Mor_{\Pi(X)}(x,x') = \faktor{\left \{w\colon  I \to  X \mid  w(0) = x, w(1) = x'\right\}}{\sim \text{ rel } \left \{0,1\right\} }
    .\] 
    d.h. die Morphismen sind genau die Wege von $x$ nach $x'$ modulo Homotopie bezüglich endpunkten. Die Verknüpfung der Morphismen ist durch $\star$ gegeben.
\end{definition}

\begin{remark*}
    \begin{enumerate}[1)]
        \item Ein \vocab{Gruppoid} ist eine Kategorie, in der jeder Morphismus invertierbar ist, d.h. in der jedere Morphismus ein Isomorphismus ist. In obigem Fall ist das gegeben, weil wir für $w$ den Weg $\overline{w}$ als Inverses haben.
            \begin{warning}
                Wir fordern nicht, dass zwischen je zwei Objekten ein Isomorphismus existiert. Das ist auch in $\Pi(X)$ nicht der Fall, wenn  $X$ nicht wegzusammenhängend ist.
            \end{warning}
            Ist $\cat{G}$ ein Gruppoid, so ist $\Mor_{\cat{G}}(A,A)$ für $A\in \Ob(\cat{C})$ stets eine Gruppe. Vergleiche hierzu auch das Beispiel, bei dem wir eine Gruppe als Kategorie mit einem Objekt aufgefasst haben.
        \item Per Definition ist $\Mor_{\Pi(X)}(x,x) \cong \pi_1(X,x)$ (als Gruppen).
        \item Per Definition ist nun
            \[
                \pi_0(X) \cong \faktor{\Ob(\Pi(X))}{\text{Isomorphi}}
            .\] 
    \end{enumerate}
\end{remark*}

\begin{oral}
    Obige Konstruktion ist funktoriell (Zuordnung $X \to  \Pi(X)$). Dadurch 'verpacken' wir die Basispunktwahl in eine Kategorie und umgehen so willkürliche Wahlen, können die Informationen aus der Kategorie (über die Fundamentalgruppe) aber jederzeit 'zurückgewinnen' (bzw. haben sie schon).
\end{oral}


\section{Überlagerungen Teil 1}

\begin{definition}[Überlagerung]\label{def:überlagerung}
    Eine \vocab{Überlagerung} ist eine stetige, surjektive Abbildung
    \[
    p\colon  E \to  X
    .\] 
    mit den folgenden Eigenschaften:
    Für jedes $x\in X$ gibt es eine Umgebung $U$ von  $x$, einen diskreten Raum  $F$ und einen Homöomorphismus
     \[
         v\colon  p^{-1}(U) \stackrel{\cong}{\longrightarrow} U \times F
    .\] 
    über $U$, d.h.
    \[
\begin{tikzcd}
    p^{-1}(U) \ar{rr}{v} \ar[swap]{dr}{p}& &  U \times F \ar{dl}{\pr_U} \\
                          & U
\end{tikzcd}
\]
\missingfigure{Triviale Überlagerung und Überlagerung $\exp\colon  \R \to  S^1$ skizzieren, erste Hälfte siehe unten}
\end{definition}

\begin{figure}[ht]
    \centering
    \incfig{definition-überlagerung}
    \caption{Veranschaulichung einer (trivialen) Überlagerung}
    \label{fig:definition-überlagerung}
\end{figure}


\begin{remark}
    $F$ ist homöomorph zu  $F_x = p^{-1}(\left \{x\right\} )$, der \vocab{Faser über $x$} mittels
    \[
        v|_{p^{-1}(x)} \colon  p^{-1}(x) \to  \left \{x\right\} \times F
    .\] 
\end{remark}


\begin{lemma}[Überlagerung von Teilräumen]\label{lm:überlagerung-von-teilräumen}
    Sei $p\colon E \to  X$ eine Überlagerung, $Y\subset X$ ein Teilraum. Dann ist auch
    \[
        p|_{p^{-1} (Y)} \colon  p^{-1} (Y) \to  Y
    .\] 
    eine Überlagerung.
\end{lemma}

\begin{proof}
    Sei $x\in Y$. Dann $\exists U\subset X$ Umgebung von $x$ und  $F$ diskret und ein Homöomorphismus
     \[
         v\colon  p^{-1} (U) \to  U\times F
    .\] 
    über $U$. Dann ist $U\cap Y$ eine umgebung von $x$ in $Y$ und somit
     \[
         v|_{p^{-1} (U\cap Y)}\colon p^{-1} (U\cap Y) \to  (U\cap Y) \times F
    .\] 
    ein Homöomorphismus.
\end{proof}

\begin{example}
    Sei $X$ ein topologischer Raum,  $F$ diskret. Dann ist
     \[
    \pr_X \colon  X \times F \to  X
    .\] 
    eine Überlagerung.
\end{example}

\begin{remark}
    Ist $p\colon  E \to  X$ eine Überlagerung, so heißt diese \vocab[Überlagerung!trivial]{trivial}, falls ein Homöomorphismus $u\colon  E \to  X\times F$ über $X$ existiert, wobei  $F$ diskret.
\end{remark}

\begin{definition}
    $f\colon X \to  Y$ ist ein \vocab{lokaler Homöomorphismus}, falls $\forall x\in X$ eine offene Umgebung $x\in V \subset X$ existiert mit
    \begin{enumerate}[i)]
        \item $f(V)\subset Y$ ist offen
        \item $f|_V\colon  V \to  f(V)$ ist ein Homöomorphismus
    \end{enumerate}
\end{definition}

\begin{oral}
    Die Abbildung $f|_V \to  f(V)$ erfüllt etwas stärkere Eigenschaften als eine \nameref{def:einbettung}, weil wir hier zusätzlich fordern, dass $f(V)\subset Y$ offen ist.

    Es ist z.B. $\R \hookrightarrow \R^2$ eine Einbettung, jedoch kein lokaler Homöomorphismus.
\end{oral}

\begin{lemma}\label{lm:überlagerung-ist-lokaler-homöomorphismus}
    Eine Überlagerung ist ein lokaler Homöomorphismus.
\end{lemma}

\begin{proof}
    Für $x\in X$ existiert $x\in U\subset X$ offen mit $u\colon p^{-1} (U) \cong U\times p^{-1} ((x))$ (per Definition der Überlagerung) 

    Sei $e\in E$ mit $x = p(e)$. Wähle  $V = u^{-1}(U\times \left \{e\right\} )$. Dann ist $p|_V \colon  V \to  U$ ein Homöomorphismus.
\[
    \begin{tikzcd}
        V & &U \times \left \{e\right\} \\
        p^{-1} (U) & &U\times p^{-1} (\left \{x\right\} ) \\
                   & U 
    \end{tikzcd}
\]
Da $p^{-1} (U)\subset Y$ offen ist und $U\times \left \{e\right\} \subset U\times p^{-1} (\left \{x\right\} ) \stackrel{u^{-1}}{\cong} p^{-1} (U)$ offen, ist $V\subset Y$ offen.
\end{proof}
\todo{Kommutatives Diagramm des Beweises, Beweis nochmal durchgehen}

\begin{lemma}\label{lm:offene-überdeckung-von-überlagerung-mit-homöomorphismen}
    Sei $p\colon  E \to  X$ eine Überlagerung. Dann gibt es eine offene Überdeckung $\left \{U_i\right\} _{i \in I}$ von $X$, so dass gilt:
     \[
         \forall x\in U_i, y\in p^{-1} (x)
    .\] 
    gibt es eine stetige Funktion $s\colon  U_i \to  E$ mit 
    \begin{itemize}
        \item $s(x) = y$
        \item  $p \circ s = \id_{U_i}$
    \end{itemize}
\end{lemma}

\begin{oral}
    Der Unterschied der Aussage des Lemmas ist, dass wir hier die  $U_i$ und das  $y$ fest wählen.
\end{oral}

\begin{proof}[Beweis von \autoref{lm:offene-überdeckung-von-überlagerung-mit-homöomorphismen}]
Für jedes $x\in X$ gibt es eine offene Umgebung $U_x$, auf der $p$ trivial ist. Dann ist $\left \{U_x\right\} _{x \in X}$ eine offene Überdeckung.
\end{proof}
\todo{Kommutatives Diagramm}


\begin{remark}
    Nicht jeder lokale Homöomorphismus ist eine Überlagerung. Die Abbildung $(0,2) \stackrel{\exp}{\longrightarrow} S^1$ ist ein lokaler Homöomorphismus, aber keine Überlagerung.

    Das 'Problem' ist hierbei, dass $1\in S^1$ nur ein Urbild unter $\exp$ hat, jede Umgebung von  $1\in (0,2)$ im Urbild ist hat jedoch drei Urbilder von den Punkten nahe $1$, ist also nicht einfach nur ein Intervall.
\end{remark}

\begin{notation*}
    Ist $p\colon  E \to  X$ eine Überlagerung, so nennen wir
    \begin{itemize}
        \item $X$ die  \vocab[Überlagerung!Basis]{Basis} oder den \vocab[Überlagerung!Basisraum]{Basisraum}  
        \item $E$ den  \vocab[Überlagerung!Totalraum]{Totalraum}
        \item $p$ die  \vocab[Überlagerung!Überlagerungsabbildung]{Überlagerungsabbildung} oder \vocab[Überlagerung!Überlagerungsprojektion]{Überlagerungsprojektion}  
        \item $p^{-1} (x)$ die \vocab{Faser über $x$}, $F_x$ 
        \item $\abs{p^{-1} (x)} $ die \vocab[Überlagerung!Blätterzahl]{Blätterzahl} . Diese ist lokal konstant.
    \end{itemize}
\end{notation*}

\begin{example}
    \begin{enumerate}[1)]
        \item Die triviale Überlagerung
        \item Die unendlich-blättrige Überlagerung
                \begin{equation*}
                \exp: \left| \begin{array}{c c l} 
                \R & \longrightarrow & S^1 \\
                t & \longmapsto &  e^{2\pi it}
                \end{array} \right.
            \end{equation*}
        \item Es gibt auch eine $k$-blättrige Überlagerung des Einheitskreises:
                \begin{equation*}
                    ()^k : \left| \begin{array}{c c l} 
                S^1 & \longrightarrow & S^1 \\
                z & \longmapsto &  z^k
                \end{array} \right.
            \end{equation*}
            (wir fassen hier $S^1 \subset \C$ auf, um $z^k$ zu definieren).
    \missingfigure{Skizzen zur endlichblättrigen Überlagerung}
        \item Sind  $E,E' \stackrel{p,p'}{\longrightarrow} X$ Überlagerungen, so auch
            \[
            E \coprod E' \stackrel{p \coprod p'}{\longrightarrow} X
            .\] 
        \item Die Projektion $S^n \stackrel{p}{\longrightarrow} \R \mathbb{P}^n \cong \faktor{S^n}{x \sim  -x}$ ist eine 2-blättrige Überlagerung.
            \missingfigure{Überlagerung $S^n \stackrel{p}{\longrightarrow} \R\mathbb{P}^n$}
    \end{enumerate}    
\end{example}

\begin{oral}
    Man kann zeigen, dass die endlichen Überlagerungen die einzigen \textit{zusammenhängenden} Überlagerungen von $S^1$ sind. Für 'schöne' Räume werden wir diese auch noch klassifizieren.
\end{oral}

In der nächsten Woche behandeln wir \textit{Liftungssätze}, d.h. wir fragen uns
\[
\begin{tikzcd}
    & E \ar{d}{p} \\
    T \ar[dashed]{ur}{\exists \tilde{f}?} \ar[swap]{r}{f} & X
\end{tikzcd}
\]
das ganze gilt z.B. für Wege, also wenn $T = I$, wir können uns dann ein Urbild  $e \mapsto f(0) = x$ wählen und erhalten  $\tilde{f}$ mit $\tilde{f}(0) = e$, sodass obiges kommutiert.

\begin{oral}[ca.]
    Wir werden dann feststellen, dass die Hebung von $w\colon  I \to  X$ nicht zwingend eine Schleife ist, auch wenn $w$ es war. Wenn wir dann auch noch Homotopien heben können, so können wir schließen, dass  $w$ nicht nullhomotop war, weil das sonst auch für den gehobenen Weg gelten müsste, dieser aber nichtmal dieselben Endpunkte hat.
\end{oral}

    % end lectures
    %\input{fragestunden.tex}
\end{document}
