\documentclass[a4paper, german, lecturenumbers = true, number small environments = theorem, hide version]{mkessler-script}

\course{Einführung in die Geometrie und Topologie}
\lecturer{Daniel Kasprowski}
\assistant[f]{Arunima Ray}
\author{Maximilian Keßler}

\RequirePackage{mkessler-math}
\RequirePackage{mkessler-fancythm}
\usepackage{epsfig}
%\usepackage{psfrag}
%\usepackage{sseq} (if you need to draw spectral sequences, please use this package, available at http://wwwmath.uni-muenster.de/u/tbauer/)
\usepackage{mathrsfs}
\usepackage{amscd}
\usepackage{amsbsy}
\usepackage{verbatim}
\usepackage{moreverb}

\newtheorem{prop}[theorem]{Proposition}
\newtheorem{cor}[theorem]{Corollary}
\newtheorem{conj}[theorem]{Conjecture}


\theoremstyle{definition}
\newtheorem{hw}{Homework}
\newtheorem{exercise*}[exercise]{$\star$ Exercise}

\theoremstyle{remark}
\newtheorem{aside}[theorem]{Aside}

\newcommand{\nn}{\nonumber}
\newcommand{\nid}{\noindent}
\newcommand{\ra}{\rightarrow}
\newcommand{\la}{\leftarrow}
\newcommand{\xra}{\xrightarrow}
\newcommand{\xla}{\xleftarrow}
\newcommand{\tto}{\longrightarrow}

\newcommand{\weq}{\xrightarrow{\sim}}
\newcommand{\cofib}{\rightarrowtail}
\newcommand{\fib}{\twoheadrightarrow}

\newcommand{\IRep}{\mathrm{IRep}}
\newcommand{\IHom}{\mathrm{IHom}}

\def\llarrow{   \hspace{.05cm}\mbox{\,\put(0,-2){$\leftarrow$}\put(0,2){$\leftarrow$}\hspace{.45cm}}}
\def\rrarrow{   \hspace{.05cm}\mbox{\,\put(0,-2){$\rightarrow$}\put(0,2){$\rightarrow$}\hspace{.45cm}}}
\def\lllarrow{  \hspace{.05cm}\mbox{\,\put(0,-3){$\leftarrow$}\put(0,1){$\leftarrow$}\put(0,5){$\leftarrow$}\hspace{.45cm}}}
\def\rrrarrow{  \hspace{.05cm}\mbox{\,\put(0,-3){$\rightarrow$}\put(0,1){$\rightarrow$}\put(0,5){$\rightarrow$}\hspace{.45cm}}}

\def\cA{\mathcal A}\def\cB{\mathcal B}\def\cC{\mathcal C}\def\cD{\mathcal D}
\def\cE{\mathcal E}\def\cF{\mathcal F}\def\cG{\mathcal G}\def\cH{\mathcal H}
\def\cI{\mathcal I}\def\cJ{\mathcal J}\def\cK{\mathcal K}\def\cL{\mathcal L}
\def\cM{\mathcal M}\def\cN{\mathcal N}\def\cO{\mathcal O}\def\cP{\mathcal P}
\def\cQ{\mathcal Q}\def\cR{\mathcal R}\def\cS{\mathcal S}\def\cT{\mathcal T}
\def\cU{\mathcal U}\def\cV{\mathcal V}\def\cW{\mathcal W}\def\cX{\mathcal X}
\def\cY{\mathcal Y}\def\cZ{\mathcal Z}

\def\sA{\mathscr A}\def\cB{\mathcal B}\def\cC{\mathcal C}\def\cD{\mathcal D}
\def\cE{\mathcal E}\def\cF{\mathcal F}\def\sG{\mathscr G}\def\cH{\mathcal H}
\def\cI{\mathcal I}\def\cJ{\mathcal J}\def\cK{\mathcal K}\def\cL{\mathcal L}
\def\cM{\mathcal M}\def\cN{\mathcal N}\def\cO{\mathcal O}\def\cP{\mathcal P}
\def\cQ{\mathcal Q}\def\cR{\mathcal R}\def\cS{\mathcal S}\def\cT{\mathcal T}
\def\cU{\mathcal U}\def\cV{\mathcal V}\def\cW{\mathcal W}\def\cX{\mathcal X}
\def\cY{\mathcal Y}\def\cZ{\mathcal Z}

\def\fG{\mathfrak G}\def\fH{\mathfrak H}
\def\fS{\mathfrak S}\def\fN{\mathfrak N}\def\fX{\mathfrak X}\def\fY{\mathfrak Y}

\def\op{\textrm{op}}\def\ob{\textrm{ob}}

%\def\Iso{\mathcal Iso}\def\cInn{\mathcal Inn}

\def\fg{\mathfrak g}\def\fh{\mathfrak h}\def\fri{\mathfrak i}\def\fp{\mathfrak p}
\def\fA{\mathfrak A}\def\fU{\mathfrak U}

\def\AA{\mathbb A}\def\BB{\mathbb B}\def\CC{\mathbb C}\def\DD{\mathbb D}
\def\EE{\mathbb E}\def\FF{\mathbb F}\def\GG{\mathbb G}\def\HH{\mathbb H}
\def\II{\mathbb I}\def\JJ{\mathbb J}\def\KK{\mathbb K}\def\LL{\mathbb L}
\def\MM{\mathbb M}\def\NN{\mathbb N}\def\OO{\mathbb O}\def\PP{\mathbb P}
\def\QQ{\mathbb Q}\def\RR{\mathbb R}\def\SS{\mathbb S}\def\TT{\mathbb T}
\def\UU{\mathbb U}\def\VV{\mathbb V}\def\WW{\mathbb W}\def\XX{\mathbb X}
\def\YY{\mathbb Y}\def\ZZ{\mathbb Z}

\def\TOP{\mathcal{TOP}}\def\GRP{\mathcal{GRP}}\def\GRPD{\mathcal{GRPD}} \def\CAT{\mathcal{CAT}} \def\SET{\mathcal{SET}}

\def\id{\mathrm{id}}\def\Id{\mathrm{Id}}
\def\inverse{^{-1}}



\begin{document}
    \maketitle
    \begin{abstract}
    {\color{red} Achtung:} Diese Version des Skripts benutze ich zur Bearbeitung! Einige Dinge fehlen, dafür gibt es TODO-Notes. Für Inhalte, benutzt die \href{https://kesslermaximilian.github.io/LectureNotesBonn/2021_Topologie.pdf}{normale Version}
    \end{abstract}
    \newpage
    \listoftodos
    \newpage
    \summaryoflectures
    \newpage
    % start lectures
    \setcounter{section}{13}
    \setcounter{dummy}{16}
    \setcounter{smalldummy}{3}
    \setcounter{figure}{15}
    \setcounter{claim}{2}
    \setcounter{lecture}{12}
    %! TEX root = ./master.tex
\lecture[]{Di 08 Jun 2021 12:15}{}
\begin{example}
    \begin{enumerate}[1)]
        \item Wir können eine Gruppe $G$ als Kategorie auffassen, indem wir  $\Ob{\cat{G}} = \left \{\star\right\} $ und $\Mor_{\cat{G}}(\star,\star) = G$ setzen, wobei natürlich $g \circ  h = gh$. Man könnte das in etwa so skizzieren:
            \[
\begin{tikzcd}
            \bullet \ar[loop left]{}{f} \ar[loop above]{}{\id} \ar[loop right]{}{g} \ar[loop below]{}{f \circ g}
\end{tikzcd}
\]
\item Ein \vocab{G-Objekt} in einer Kategorie $\cat{C}$ ($G$ ist eine Gruppe) ist ein Funktor  $\cat{G} \to  \cat{C}$ (dieser Funktor besteht aus einem Objekt von $\cat{C}$ zusammen mit Endomorphismen dieses Objekts für jedes $g\in G$)
    \begin{remark*}
        Ein typisches Beispiel sind Gruppenwirkungen, wählen wir hier $\cat{C} = \Set$, so sind die $G$-Objekte genau  $G$-Mengen bzw.  $G$ wirkt dann auf die entsprechende Menge  $\mathcal{F}(\cat{G})\in \Set$.
    \end{remark*}
\item Sind $\cat{C}$, $\cat{D}$ Kategorien, so gibt es die \vocab{Produktkategorie} $\cat{C} \times  \cat{D}$, die wir erhalten, indem wir $\Ob{\cat{C} \times \cat{D}} = \Ob{\cat{C}} \times \Ob{\cat{D}} $ (dieses Produkt müssen wir potenziell als das von Klassen auffassen) setzen und als Morphismen
    \[
        \Mor_{\cat{C}\times \cat{D}}((X,Y),(X',Y')) \coloneqq  \Mor_{\cat{C}}(X,X') \times \Mor_{\cat{D}}(Y,Y') 
    .\] 
    setzen mit komponentenweiser Komposition, d.h. $(f,g) \circ  (f', g') = (f \circ  f', g \circ  g')$.
\item Das Wedge-Produkt ($\twedge$) können wir nun als Funktor
     \[
    \twedge \Top_{\star} \times \Top_{\star} \to  \Top_{\star}
    .\] 
    auffassen, mittels
    \[
        ((X,x_0),(Y,y_0)) \mapsto X \bigcup\limits_{\left \{\star\right\} }Y / (x_0 \sim  y_0)
    .\] 
wobei der Punkt $x_0 \sim y_0$ der Basispunkt des neuen Raumes ist.


    Analog geht das auch für das Wedge-Produkt, d.h. $\tsmash \Top_{\star} \times  \Top_{\star} \to  \Top_{\star}$ 
    \begin{example}
        Fehlt.
    \end{example}
\item Wir können die Funktorkategorie $\Mor(\cat{C},\cat{D})$ bilden, indem wir $\Ob(\Mor(\cat{C},\cat{D}))$ als die Menge der Funktoren von $\cat{C}$ nach $\cat{D}$, und $\Mor_{\Mor_{\cat{C},\cat{D}}}(\mathcal{F},\mathcal{G})$ als die Menge der natürlichen Transformationen von $\mathcal{F}$ nach $\mathcal{G}$ wählen.
    \end{enumerate}
\end{example}
\todo{Wedge-Produkt hier nochmal überarbeiten.}


\section{Homotopien und die Fundamentalgruppen}
\begin{oral}
    Wir reden von nun an über 'Abbildungen', wobei wir damit meinen, dass jede Abbildung automatisch stetig ist.
\end{oral}

\begin{definition}[Homotopie]\label{def:homotop}
    Zwei Abbildungen $f,g\colon  X \to  Y$ heißen \vocab{homotop}, falls es eine Abbildung
    \[
        H \colon  X \times [0,1] \to  Y
    .\] 
    gibt mit 
    \[
        H_0 \coloneqq  H(-,0) = f \qquad H_1 = g
    .\] 
    Die Abbildung $H$ nennen wir dann  \vocab{Homotopie}. 
\end{definition}

\begin{dnotation}
    Da wir uns $t\in I$ als 'zeitlichen Verlauf' vorstellen, notieren wir üblicherweise $H_t \coloneqq  H(-,t)\colon  X \to  Y$ als die Abbildung zum Zeitpunkt $t\in [0,1]$. 

    Zudem notieren wir $I \coloneqq  [0,1]$ als das Einheitsintervall.
\end{dnotation}

\begin{lemma}
    Homotopie ist eine Äquivalenzrelation auf $\Mor_{\Top}(X,Y)$.
\end{lemma}
\begin{proof}
    \begin{description}
        \item[Reflexivität] Ist $f\colon  X \to  Y$ stetig, so auch  $H \colon  X \times I \stackrel{\pr_X}{\to } X \stackrel{f}{\to } Y $ und klarerweise ist $H_0 = H_1 = f$.
        \item[Symmetrie] Ist $H \colon  X \times I \to  Y$ stetig, so auch
            \[
                H' \colon  X \times I \stackrel{(\id, 1-t)}{\longrightarrow}  X\times I \longrightarrow  Y
            .\] 
        \item[Transitivität] Sind $H,G \colon  X \times I \to  Y$ stetig und $H_1 = G_0$, so ist auch die Abbildung
                \begin{equation*}
                HG: \left| \begin{array}{c c l} 
                X\times I & \longrightarrow & Y \\
                (x,t) & \longmapsto &  \begin{cases}
                    H(x,2t) & 0 \leq  t \leq  \frac{1}{2} \\
                    G(x, 2t-1) & \frac{1}{2}\leq t\leq 1
                \end{cases}
                \end{array} \right.
            \end{equation*}
            und wir prüfen leicht $(HG)_0 = H_0$ sowie $(HG)_1 = G_1$.
    \end{description}
\end{proof}

\begin{definition}[Homotop relativ einer Menge]\label{def:homotopie-relativ-einer-menge}
    \begin{enumerate}[i)]
        \item 
    Zwei punktierte Abbildungen $f,g \colon  (X,x_0) \to  (Y,y_0)$ heißen (punktiert) \vocab{homotop}, falls es eine Abbildung
    \[
    H \colon  X \times I \to  Y
    .\] 
    mit $H_0 = g, H_1 = g$ gibt, und zusätzlich $H_t(x_0) = y_0 \forall t\in I$ (wir lassen also den Basispunkt zu jedem Zeitpunkt fest).
\item Sei $A\subset X$ und $f,g\colon  X \to  I$. Die Abbildungen $f,g$ heißen  \vocab{homotop relativ $A$}, falls es
    \[
    H \colon  X \times I \to  Y
    .\] 
    mit $H_0 = g, H_1 = g$ und $H(a,t) = H(a,t')$ für alle  $a\in A, t\in I$ gibt (d.h, die Homotopie bleibt auf $A$ konstant). Inbesondere gilt dies nur, wenn  $f|_A = g|_A$
    \end{enumerate}
\end{definition}

\begin{definition}[Homotopiekategorie]\label{def:homotopiekategorie}
    Die \vocab{(naive) Homotopiekategorie $\hTop$}      ist die Kategorie mit $\Ob(\hTop) = \Ob(\Top)$ und 
    \[
        \Mor_{\hTop}(X,Y) = \Mor_{\Top}(X,Y) / \sim
    \]
    d.h. wir identifizieren Abbildungen modulo Homotopie.
\end{definition}

\begin{proof}
    Übung. Einige Wohldefiniertheiten müssen geprüft werden.
\end{proof}

\begin{definition}[Homotopieäquivalenz]\label{def:homotopieäquivalenz}
    \begin{enumerate}[a)]
        \item 
            Eine Abbildung $f\colon  X \to  Y$ heißt \vocab{Homotopieäquivalenz}, falls $[f]$ ein Isomorphismus in  $\hTop$ ist, d.h. falls es eine Abbildung $g\colon  Y \to  X$ gibt, so dass $g \circ  f \sim  \id_X$ und $f \circ  g \sim \id_Y$ jeweils homotop zu den Identitäten sind.
\item Existiert eine Homotopieäquivalenz $f\colon  X \to Y$, so heißen $X$ und  $Y$  \vocab{homotopieäquivalent}. 
    \end{enumerate}
\end{definition}

\begin{example}
    Der Einpunktraum $\left \{\star\right\}$ ist homotopieäquivalent zu $\R^n$ mittels
        \begin{equation*}
        f: \left| \begin{array}{c c l} 
        \left \{\star\right\}  & \longrightarrow & \R^n \\
        \star & \longmapsto &  0
        \end{array} \right.
    \end{equation*}
        \begin{equation*}
        g: \left| \begin{array}{c c l} 
        \R^n & \longrightarrow & \left \{\star\right\} \\
        x & \longmapsto &  \star
        \end{array} \right.
    \end{equation*}
    Hierbei ist $g \circ  f = \id_{\left \{\star\right\} }$ sowieso schon die Identität, und es ist $f \circ  g = \mathcal{C}_0$ (die konstante Nullabbildung). Mittels
        \begin{equation*}
        H: \left| \begin{array}{c c l} 
        \R^n\times I & \longrightarrow & \R^n \\
        (x,t) & \longmapsto &  tx
        \end{array} \right.
    \end{equation*}
    erhalten wir auch $H_0 = \mathcal{C}_0$ und $H_1 = \id_{\R^n}$, sodass wir eine Homotopie $\mathcal{C}_0 \sim \id$ gefunden haben.
\end{example}

\begin{lemma}\label{lm:homotopieäquivalenz-ist-äquivalenzrelation}
Homotopieäquivalenz ist eine Äquivalenzrelation.    
\end{lemma}

\begin{proof}
    Isomorphie in einer Kategorie ist eine Äquivalenzrelation.
\end{proof}


\begin{notation}
    Wir bezeichen
    \[
        [X,Y] \coloneqq  \Mor_{\Top}(X,Y) / \sim = \Mor_{\hTop}(X,y)
    .\] 
    und analog
    \[
        [X,Y]_{\star} \coloneqq  [(X,x_0),(Y,y_0)] \coloneqq  \Mor_{\Top_{\star}}(X,Y) / \sim 
    .\] 
\end{notation}
\todo{Notation nummerieren}.

\begin{definition}[Schleife]\label{def:schleife}
    Eine \vocab[Schleife]{Schleife in $X$ an  $x\in X$} ist eine Abbildung $w\colon  I \to  X$ mit $w(0) = w(1) = x$. 

    Äquivalent ist  $w$ eine Schleife, wenn  $w\in \Mor_{\Top_{\star}}((S^1,1), (X,x_0))$, indem wir \autoref{thm:kreis-ist-quotientenraum-von-einheitsintervall} anwenden.
\end{definition}


\begin{definition}[Komposition von Schleifen]\label{def:komposition-von-schleifen}
    Für zwei Schleifen $w,w'$ in  $X$ an  $x$ ist  $w \star w'$ die Schleife mit
     \[
         (w \star w')(t) = \begin{cases}
             w(2t) & 0\leq t\leq \frac{1}{2} \\
             w'(2t-1) & \frac{1}{2}\leq t\leq 1
         \end{cases}
    .\] 
\end{definition}

\begin{theorem}\label{thm:star-ist-gruppenstruktur}
    Sei $x_0\in X$ ein Basispunkt. Dann definiert $\star$ eine Gruppenstruktur auf  $[S^1,(X,x_0)]_{\star}$.
\end{theorem}

\begin{proof}
    Als erstes zeigen wir wohldefiniertheit auf den Homotopieklassen, d.h. Wenn wir $w,w'$ bis auf Homotopie ändern, so ist auch deren Komposition bis auf Homotopie stets gleich. Seien hierzu
    \[
        H,G \colon  [0,1] / \left \{0,1\right\} \times I \to  X
    .\] 
    punktierte Homotopien. Dann setzen wir
    \[
        (H \star G) (s,t) = \begin{cases}
            H(2s,t) & 0 \leq  s \leq  \frac{1}{2} \\
            G(2s-1,t) & \frac{1}{2} \leq  s \leq  1
        \end{cases}
    .\] 
    und erhalten wegen $(H \star G)_0 = H_0 \star G_0$ und  $(H\star G) _1 = H_1 \star G_1$ eine entsprechende Homotopie der verknüpften Wege.

    Man beachte, dass wir hierzu $H(1,t) =H_t(1) =  x_0 = G_t(0) = G(0,t)$ benötigen, weil es sich um Schleifen handelte.

    \begin{description}
        \item[Assoziativität] Sind $w,w',w''$ drei Schleifen an  $x_0$, so durchlaufen wir bei der Schleife $(w \star w') \star w''$ auf $[0,\frac{1}{2}]$ den Weg $w \circ  w'$ und auf $[\frac{1}{2},1]$ $w''$. Für $w \star (w' \star w'')$  passiert zwar eigentlich das Gleiche, allerdings in anderen Zeitintervallen, weswegen folgendes hilft:
\[
            \begin{tikzpicture}
                \draw (0,0) -- (4,0) -- (4,2) -- (0,2) -- cycle;
                \draw (1,2) -- (2,0);
                \draw (2,2) -- (3,0);
            \end{tikzpicture}
        \]
    Formal setzen wir also
    \[
        H(t,s) = \begin{cases}
            w\left( \frac{4t}{1+s} \right) & 0 \leq  t \leq  \frac{1+s}{4} \\
            w'(4t-1-s) & \frac{1+s}{4}\leq t \leq  \frac{2+s}{4} \\
            w''\left( \frac{4t-2-s}{2-s} \right) & \frac{2+s}{4}\leq t\leq 1
        \end{cases}
    .\] 
    Wir prüfen wieder, dass für $t = \frac{1+s}{4}$ $w(1) = x_0 = w'(0)$ und analog auch für $t = \frac{2+s}{4}$.
\item[neutrales Element] Sei $w$ eine Schleife an $x_0$, und $c\colon  I \to  X$ konstant $x_0$. Dann ist
    \[
        H(t,s) = \begin{cases}
            w(\frac{2t}{1+s}) & 0 \leq  t\leq \frac{1+s}{2} \\
           x_0 & \frac{1+s}{2}\leq t\leq 1
        \end{cases}
    .\] 
\item[Inverses] Sei $\overline{w} (t) = w(1-t)$, wir behaupten, dass diese Schleife ein Inverses darstellt. Dann ist
    \[
        H(t,s) = \begin{cases}
            w(2t) & 0 \leq  t \leq  \frac{1-s}{2} \\
            w(1-s) & \frac{1-s}{2} \leq  t \leq  \frac{1+s}{2} \\
            w(2(1-t)) = \overline{w}(2t-1) & \frac{1+s}{2}\leq  t \leq  1
        \end{cases}
    .\] 
    eine Homotopie von $w \star \overline{w}$ nach $c$.
    \missingfigure{Illustration der Homotopie mittels Spaghetti-Trick}.
    \end{description}
\end{proof}
\todo{Homotopie-Illustration beschriften}.
\begin{oral}
    Man erkennt, dass es für obigen Beweis wichtig war, dass wir die Verknüpfung nur bis auf Homotopie definieren, weil wir ansonsten Probleme mit der Parametrisierung bekommen.
\end{oral}

\begin{definition}[Fundamentalgruppe]\label{def:fundamentalgruppe}
    Für $(X,x_0) \in \Top_{\star}$ ist 
    \[
        \pi_1(X,x_0) \coloneqq  ([(S^1,1),(X,x_0)]_{\star},\star)
    .\] 
    die \vocab{Fundamentalgruppe} von $X$ an  $x_0$. 
\end{definition}

\begin{oral}
    Wir werden feststellen, dass es sich bei obigem um eine Homotopieinvariante handelt, mit der wir eine weitere charakteristische Eigenschaft von topologischen Räumen gefunden haben, um diese zu unterscheiden. 

    Allerdings wird sich herausstellen, dass die Berechnung solcher Gruppen recht mühsam ist, weswegen wir noch die Technik sogenannter \textit{Überlagerungen} kennenlernen werden.
\end{oral}

\begin{theorem}
    Ist $f\colon  (X,x_0) \to  (Y,y_0)$ eine Abbildung, so induziert diese einen Gruppenhomomorphismus
    \[
        \pi_1 (X,x_0) \stackrel{p_1(f) =: f_*}{\longrightarrow}   p_1(Y,y_0)
    .\] 
    mittels $[w] \mapsto [f \circ  w]$. Damit ist insbesondere $\pi_1$ ein Funktor $\Top_{\star} \to  \Grp$.
\end{theorem}

\begin{proof}
    \begin{description}
        \item[Wohldefiniertheit] Ist $w\stackrel{H}{\sim} w'$, so ist $f \circ  w \stackrel{f \circ  H}{\sim }  f \circ  w''$.
        \item[Gruppenhomomorphismus] Es ist
            \begin{IEEEeqnarray*}{rCl}
                p_1(f) ([w] \star [w']) &=& p_1(f) ([w\star w']) \\
                                        & = & [f \circ  (w \star w')] \\
                                        & \stackrel{\text{(1)}}{=}  & [(f \circ  w) \star (f \circ  w') ] \\
                                        & = & [f \circ  w] \star [f \circ  w'] \\
                                        & = & \pi(f) ([w]) \circ  \pi_1(f)([w'])
            \end{IEEEeqnarray*}
            wobei in (1) Gleichheit gilt, weil es sich bei beiden Seiten um die Abbildung
            \[
            \begin{cases}
                (f \circ w)(2t) & 0\leq t\leq \frac{1}{2} \\
                (f \circ  w') (2t-1) & \frac{1}{2}\leq t\leq 1
            \end{cases}
            .\]
            handelt. 
        \item[Funktorialität] 
            \begin{itemize}
                \item Es ist $\pi_1(\id_X) [w] = [\id_X \circ  w = [w]]$, also bilden wir Identitäten auf Identitäten ab.
                \item Wir rechnen nach:
                    \begin{IEEEeqnarray*}{rCl}
                        \pi_1(f \circ g)([w]) & = & [(f \circ g) \circ  w] \\
                                              &=& [f \circ  (g \circ  w)] \\
                                              & = & \pi_1(f) [g \circ  w] \\
                                              & = & \pi_1(f)(\pi_1(g)[w]))\\
                                              & = & (p_1(f) \circ  \pi_1(g))[w]
                    \end{IEEEeqnarray*}
                    und wir erhalten, dass $\pi_1(f \circ g) = \pi_1(f) \circ  \pi_1(g)$.
            \end{itemize}
    \end{description}
\end{proof}

\begin{oral}
    Wir können uns $\pi_1$ vorstellen als 'Löcher-zählen', dazu später mehr.
\end{oral}


    %! TEX root = ./master.tex
\lecture[]{Do 10 Jun 2021 10:15}{}

\begin{theorem}[$\pi_1$ auf Wegzusammenhangskomponenten]\label{thm:pi-1-ist-gleich-auf-wegusammenhangskomponente}
    Seien $x,x' \in X$ durch einen Weg verbunden, d.h. $\exists w\colon  I \to X$ mit $w(0) = x$,  $w(1) = x'$. Dann sind die Gruppen  $\pi_1(X,x)$ und $\pi_1(X,x')$ homöomorph.
\end{theorem}

\begin{remark}
    Liegen $x,x'$ in verschiedenen Wegzusammenhangskomponenten, so gibt es (im Allgemeinen) keine Beziehung zwischen  $\pi_1(X,x)$ und $\pi_1(X,x')$. Betrachte hierzu z.B. $X = X_1 \coprod X_2$ mit $x\in X_1$ und $x' \in X_2$, so ist $p_1(X,x) = p_1(X_1,x)$ und $p_1(X,x') = p_1(X_2,x')$. Es genügt daher zu sehen, dass es Räume mit verschiedenen Fundamentalgruppen gibt.
    \missingfigure{Disjunkte Vereinigung von Räumen mit verschiedener Fundamentalgruppe}
\end{remark}

\begin{proof}[Beweis von \autoref{thm:pi-1-ist-gleich-auf-wegusammenhangskomponente}]
    Sei $v\colon  I \to  X$ eine Weg von $x \circ  x'$. Dann definiere
        \begin{equation*}
        v_*: \left| \begin{array}{c c l} 
            \pi_1(X,x') & \longrightarrow & (X,x) \\
            \left[w\right] & \longmapsto &  \left[v \star w \star \overline{v}\right]
        \end{array} \right.
    \end{equation*}
    wobei wie üblich $\overline{v}\colon [0,1] \to X$ gegeben ist durch $\overline{v}(t) = v(1-t)$.
    \missingfigure{Illustration von $v \star w \star \overline{v}$}
    Hierbei definieren wir
    \[
        (v \star w \star \overline{v})(t) = \begin{cases}
            v(3t) & 0 \leq  t \leq  \frac{1}{3}\\
            w(3t-1) & \frac{1}{3}\leq t \leq  \frac{2}{3} \\
            v(3(1-t)) & \frac{2}{3} \leq  t \leq  1
        \end{cases}
    .\] 
    Damit diese Konstruktion ein Gruppenisomorphismus ist, zeigen wir:
    \begin{description}
        \item[Wohldefiniertheit] Für $w \stackrel{H}{\sim } w'$ ist auch $v \star w \star \overline{v} \sim  v \star w' \star \overline{v}$ mittels der Homotopie
            \[
                H(t,s) = \begin{cases}
                    v(3t) & 0\leq t\leq \frac{1}{3} \\
                    H(3t-1,s) & \frac{1}{3}\leq  t \leq  \frac{2}{3} \\
                    v(3(1-t)) & \frac{2}{3}\leq t\leq 1
                \end{cases}
            .\] 
        \item[Gruppenhomomorphismus] Es ist:
            \begin{IEEEeqnarray*}{rCl}
                v \star (w \star w') \star \overline{v} &\stackrel{\overline{v}\star v \sim c_{x'}}{=} &v \star (w \star (\overline{v} \star v) \star w') \star \overline{v} \\
                                                        & \stackrel{\text{assoz.}}{=}  & (v \star w \star \overline{v}) \star (v \star w' \star \overline{v}) \\
                                                        & = & v_*(w) \circ v_*(w')
            \end{IEEEeqnarray*}
        \item[Isomorphismus] $\overline{v}$ induziert analog zu $v$ eine Abbildung
             \[
                 \overline{v}_*\colon  \pi_1(X,x) \to  \pi_1(X,x')
            .\] 
            Wir behaupten, dass $\overline{v}_*$ ein Inverses ist, also $v_* \circ  \overline{v}_* = \id_{\pi_1(X,x)}$, dann folgt wegen $\overline{\overline{v}} = v$ auch sofort $\overline{v}_* \circ  v_ = \id_{\pi_1(X,x')}$. 

            Sei also $[w] \in  \pi_1(X,x)$, dann ist
            \begin{IEEEeqnarray*}{rCl}
                (                v_* \circ  \overline{v}_*)([w]) & = & [v \star \overline{v} \star w \star v \star \overline{v}] \\
                                                                 & = & [w]
            \end{IEEEeqnarray*}
            also ist $v_* \circ  \overline{v}_*$ tatsächlich die Identität.
    \end{description}
\end{proof}

\begin{warning}
    Der Isomorphismus $v_*\colon  \pi_1(X,x') \to  \pi_1(X,x)$ hängt von $v$ ab (genauer von der Homotopieklasse von  $v$ relativ Endpunkten). Die Gruppen sind aber in jedem Fall isomorph.    
\end{warning}
\todo{Frage: gilt hier potenziell die Umkehrung?}

\begin{recap}
    Ist $G$ eine Gruppe,  $g\in G$, so ist
        \begin{equation*}
        c_g: \left| \begin{array}{c c l} 
        G & \longrightarrow & G \\
        h & \longmapsto &  ghg^{-1}
        \end{array} \right.
    \end{equation*}
    eine Automorphismus von $G$, die  \vocab[Gruppe!Konjugation]{Konjugation mit $g$}. Diese heißen \vocab[Gruppe!innerer Automorphismus]{innere Automorphismen} 
\end{recap}

Sind $v,v' \colon  I \to  X$ Wege von $x$ nach  $x'$, dann gilt:
\begin{IEEEeqnarray*}{rCl}
    v_*([w]) &=& v\star [w] \star \overline{v}\\
             & = & [v \star \overline{v'}]  \star v_*'([w]) \star [v' \star \overline{v}]\\& = & c_{[v \star \overline{v'}]} (v_*'[w])
\end{IEEEeqnarray*}

\begin{oral}
    Man muss aufpassen, bei wegzusammenhängenden Räumen einfach von 'der' Fundamentalgruppe zu sprechen. Es gibt zwar nur eine, allerdings sind die Isomorphismen nicht zwingend kanonisch, weswegen man trotzdem implizit noch die Wahl des Basispunktes mit sich rumschleppt. Das motiviert die Betrachtung der nächsten Definition:
\end{oral}


\begin{definition}[Fundamentalgruppoid]\label{def:fundamentalgruppoid}
    Der \vocab{Fundamentalgruppoid} $\Pi(X)$ ist die Kategorie mit  $\ob(\Pi(X)) = X$ und
     \[
         \Mor_{\Pi(X)}(x,x') = \faktor{\left \{w\colon  I \to  X \mid  w(0) = x, w(1) = x'\right\}}{\sim \text{ rel } \left \{0,1\right\} }
    .\] 
    d.h. die Morphismen sind genau die Wege von $x$ nach $x'$ modulo Homotopie bezüglich endpunkten. Die Verknüpfung der Morphismen ist durch $\star$ gegeben.
\end{definition}

\begin{remark*}
    \begin{enumerate}[1)]
        \item Ein \vocab{Gruppoid} ist eine Kategorie, in der jeder Morphismus invertierbar ist, d.h. in der jedere Morphismus ein Isomorphismus ist. In obigem Fall ist das gegeben, weil wir für $w$ den Weg $\overline{w}$ als Inverses haben.
            \begin{warning}
                Wir fordern nicht, dass zwischen je zwei Objekten ein Isomorphismus existiert. Das ist auch in $\Pi(X)$ nicht der Fall, wenn  $X$ nicht wegzusammenhängend ist.
            \end{warning}
            Ist $\cat{G}$ ein Gruppoid, so ist $\Mor_{\cat{G}}(A,A)$ für $A\in \Ob(\cat{C})$ stets eine Gruppe. Vergleiche hierzu auch das Beispiel, bei dem wir eine Gruppe als Kategorie mit einem Objekt aufgefasst haben.
        \item Per Definition ist $\Mor_{\Pi(X)}(x,x) \cong \pi_1(X,x)$ (als Gruppen).
        \item Per Definition ist nun
            \[
                \pi_0(X) \cong \faktor{\Ob(\Pi(X))}{\text{Isomorphi}}
            .\] 
    \end{enumerate}
\end{remark*}

\begin{oral}
    Obige Konstruktion ist funktoriell (Zuordnung $X \to  \Pi(X)$). Dadurch 'verpacken' wir die Basispunktwahl in eine Kategorie und umgehen so willkürliche Wahlen, können die Informationen aus der Kategorie (über die Fundamentalgruppe) aber jederzeit 'zurückgewinnen' (bzw. haben sie schon).
\end{oral}


\section{Überlagerungen Teil 1}

\begin{definition}[Überlagerung]\label{def:überlagerung}
    Eine \vocab{Überlagerung} ist eine stetige, surjektive Abbildung
    \[
    p\colon  E \to  X
    .\] 
    mit den folgenden Eigenschaften:
    Für jedes $x\in X$ gibt es eine Umgebung $U$ von  $x$, einen diskreten Raum  $F$ und einen Homöomorphismus
     \[
         v\colon  p^{-1}(U) \stackrel{\cong}{\longrightarrow} U \times F
    .\] 
    über $U$, d.h.
    \[
\begin{tikzcd}
    p^{-1}(U) \ar{rr}{v} \ar[swap]{dr}{p}& &  U \times F \ar{dl}{\pr_U} \\
                          & U
\end{tikzcd}
\]
\missingfigure{Triviale Überlagerung und Überlagerung $\exp\colon  \R \to  S^1$ skizzieren}
\end{definition}

\begin{remark}
    $F$ ist homöomorph zu  $F_x = p^{-1}(\left \{x\right\} )$, der \vocab{Faser über $x$} mittels
    \[
        v|_{p^{-1}(x)} \colon  p^{-1}(x) \to  \left \{x\right\} \times F
    .\] 
\end{remark}
test

    % end lectures
    %\input{fragestunden.tex}
\end{document}
