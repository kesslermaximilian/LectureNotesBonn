\documentclass[a4paper, german, lecturenumbers = true, number small environments = theorem, hide version]{mkessler-script}

\course{Einführung in die Geometrie und Topologie}
\lecturer{Daniel Kasprowski}
\assistant[f]{Arunima Ray}
\author{Maximilian Keßler}

\RequirePackage{mkessler-math}
\RequirePackage{mkessler-fancythm}
\usepackage{epsfig}
%\usepackage{psfrag}
%\usepackage{sseq} (if you need to draw spectral sequences, please use this package, available at http://wwwmath.uni-muenster.de/u/tbauer/)
\usepackage{mathrsfs}
\usepackage{amscd}
\usepackage{amsbsy}
\usepackage{verbatim}
\usepackage{moreverb}

\newtheorem{prop}[theorem]{Proposition}
\newtheorem{cor}[theorem]{Corollary}
\newtheorem{conj}[theorem]{Conjecture}


\theoremstyle{definition}
\newtheorem{hw}{Homework}
\newtheorem{exercise*}[exercise]{$\star$ Exercise}
\newtheorem{aufgabe}{Aufgabe}

\theoremstyle{remark}
\newtheorem{aside}[theorem]{Aside}

\newcommand{\nn}{\nonumber}
\newcommand{\nid}{\noindent}
\newcommand{\ra}{\rightarrow}
\newcommand{\la}{\leftarrow}
\newcommand{\xra}{\xrightarrow}
\newcommand{\xla}{\xleftarrow}
\newcommand{\tto}{\longrightarrow}

\newcommand{\weq}{\xrightarrow{\sim}}
\newcommand{\cofib}{\rightarrowtail}
\newcommand{\fib}{\twoheadrightarrow}

\newcommand{\IRep}{\mathrm{IRep}}
\newcommand{\IHom}{\mathrm{IHom}}

\def\llarrow{   \hspace{.05cm}\mbox{\,\put(0,-2){$\leftarrow$}\put(0,2){$\leftarrow$}\hspace{.45cm}}}
\def\rrarrow{   \hspace{.05cm}\mbox{\,\put(0,-2){$\rightarrow$}\put(0,2){$\rightarrow$}\hspace{.45cm}}}
\def\lllarrow{  \hspace{.05cm}\mbox{\,\put(0,-3){$\leftarrow$}\put(0,1){$\leftarrow$}\put(0,5){$\leftarrow$}\hspace{.45cm}}}
\def\rrrarrow{  \hspace{.05cm}\mbox{\,\put(0,-3){$\rightarrow$}\put(0,1){$\rightarrow$}\put(0,5){$\rightarrow$}\hspace{.45cm}}}

\def\cA{\mathcal A}\def\cB{\mathcal B}\def\cC{\mathcal C}\def\cD{\mathcal D}
\def\cE{\mathcal E}\def\cF{\mathcal F}\def\cG{\mathcal G}\def\cH{\mathcal H}
\def\cI{\mathcal I}\def\cJ{\mathcal J}\def\cK{\mathcal K}\def\cL{\mathcal L}
\def\cM{\mathcal M}\def\cN{\mathcal N}\def\cO{\mathcal O}\def\cP{\mathcal P}
\def\cQ{\mathcal Q}\def\cR{\mathcal R}\def\cS{\mathcal S}\def\cT{\mathcal T}
\def\cU{\mathcal U}\def\cV{\mathcal V}\def\cW{\mathcal W}\def\cX{\mathcal X}
\def\cY{\mathcal Y}\def\cZ{\mathcal Z}

\def\sA{\mathscr A}\def\cB{\mathcal B}\def\cC{\mathcal C}\def\cD{\mathcal D}
\def\cE{\mathcal E}\def\cF{\mathcal F}\def\sG{\mathscr G}\def\cH{\mathcal H}
\def\cI{\mathcal I}\def\cJ{\mathcal J}\def\cK{\mathcal K}\def\cL{\mathcal L}
\def\cM{\mathcal M}\def\cN{\mathcal N}\def\cO{\mathcal O}\def\cP{\mathcal P}
\def\cQ{\mathcal Q}\def\cR{\mathcal R}\def\cS{\mathcal S}\def\cT{\mathcal T}
\def\cU{\mathcal U}\def\cV{\mathcal V}\def\cW{\mathcal W}\def\cX{\mathcal X}
\def\cY{\mathcal Y}\def\cZ{\mathcal Z}

\def\fG{\mathfrak G}\def\fH{\mathfrak H}
\def\fS{\mathfrak S}\def\fN{\mathfrak N}\def\fX{\mathfrak X}\def\fY{\mathfrak Y}

\def\op{\textrm{op}}\def\ob{\textrm{ob}}

%\def\Iso{\mathcal Iso}\def\cInn{\mathcal Inn}

\def\fg{\mathfrak g}\def\fh{\mathfrak h}\def\fri{\mathfrak i}\def\fp{\mathfrak p}
\def\fA{\mathfrak A}\def\fU{\mathfrak U}

\def\AA{\mathbb A}\def\BB{\mathbb B}\def\CC{\mathbb C}\def\DD{\mathbb D}
\def\EE{\mathbb E}\def\FF{\mathbb F}\def\GG{\mathbb G}\def\HH{\mathbb H}
\def\II{\mathbb I}\def\JJ{\mathbb J}\def\KK{\mathbb K}\def\LL{\mathbb L}
\def\MM{\mathbb M}\def\NN{\mathbb N}\def\OO{\mathbb O}\def\PP{\mathbb P}
\def\QQ{\mathbb Q}\def\RR{\mathbb R}\def\SS{\mathbb S}\def\TT{\mathbb T}
\def\UU{\mathbb U}\def\VV{\mathbb V}\def\WW{\mathbb W}\def\XX{\mathbb X}
\def\YY{\mathbb Y}\def\ZZ{\mathbb Z}

\def\TOP{\mathcal{TOP}}\def\GRP{\mathcal{GRP}}\def\GRPD{\mathcal{GRPD}} \def\CAT{\mathcal{CAT}} \def\SET{\mathcal{SET}}

\def\id{\mathrm{id}}\def\Id{\mathrm{Id}}
\def\inverse{^{-1}}



\begin{document}
    \maketitle
    \begin{abstract}
    {\color{red} Achtung:} Diese Version des Skripts benutze ich zur Bearbeitung! Einige Dinge fehlen, dafür gibt es TODO-Notes. Für Inhalte, benutzt die \href{https://kesslermaximilian.github.io/LectureNotesBonn/2021_Topologie.pdf}{normale Version}
    \end{abstract}
    \newpage
    \listoftodos
    \newpage
    \summaryoflectures
    \newpage
    % start lectures
    \setcounter{section}{15}
    \setcounter{dummy}{5}
    \setcounter{smalldummy}{6}
    \setcounter{figure}{21}
    \setcounter{claim}{2}
    \setcounter{lecture}{14}
    %! TEX root = ./master.tex
\lecture[Weghebungssatz. Lebesguelemma. Homotopieliftungssatz.]{Di 15 Jun 2021 12:15}{Hebungssätze}

\begin{theorem}[Weghebungssatz]\label{thm:weghebungssatz}
    Sei $p\colon E \to  X$ eine Überlagerung, und $w\colon  [0,1] \to  X$ ein Weg mit Anfangspunkt $x_0\coloneqq w(0)$. Sei $y\in p^{-1} (x_0)$ (also ein Punkt in der Faser von $x_0$). Dann existiert genau ein Weg $\tilde{w}\colon  [0,1] \to  E$, sodass $\tilde{w}(0) = y$ und $p \circ  \tilde{w} = w$, d.h. es kommutiert
    \[
    \begin{tikzcd}
        & E \ar{d}{p} \\
        I \ar[dashed]{ur}{\tilde{w}} \ar[swap]{r}{w} & X
    \end{tikzcd}
    \]
    Der Weg $\tilde{w}$ heißt \vocab[Weg!Hebung]{Hebung} oder \vocab[Weg!Lift]{Lift} von $w$.  
\end{theorem}

\begin{oral}[auf Nachfrage]
    Wir können uns den Endpunkt des Weges $\tilde{w}$, also $\tilde{w}(1)$, nicht aussuchen. Dieser ist aber eindeutig bestimmt durch die Wahl des Anfangspunktes $\tilde{w}(0) = y$.

    Wir wissen also, dass dieser Endpunkt existiert und eindeutig bestimmt ist, können aber noch keine Aussage darüber treffen.

    Zudem werden wir sehen, dass der Endpunkt nicht notwendigerweise der Anfangspunkt sein wird, selbst wenn es sich bei $w$ um eine Schleife handelt.
\end{oral}

\begin{proof}
    \underline{1. Fall}: Wir nehmen an, dass $p$ trivial ist, d.h. wir finden $F$, sodass kommutiert:
    \[
    \begin{tikzcd}
        E \ar{rr}{\cong}[swap]{u} \ar[swap]{dr}{p} & & X\times F \ar{dl}{\pr_X} \\
    & X
    \end{tikzcd}
    \]
    Wir müssen also zeigen, dass genau eine Hebung $\tilde{w}\colon I \to  X\times F$ existiert mit $\tilde{w}(0) = u(y)$. Sei $u(y) = (x_0,f)$ mit $f\in F$. Wir definieren nun
    \[
        \tilde{w}(t) = (w(t),f)
    .\] 
    d.h. wir heben den Weg einfach nach $X\times \left \{f\right\} \cong X$. Dann ist $\tilde{w}(0) = (w(0),f) = (x_0,f) = u(y)$, und die Projektion ist genau
    \[
        \pr_X(\tilde{w}(t)) = w(t)
    .\] 
    wie gewünscht.

    \underline{Eindeutigkeit}:  Sei $\tilde{\tilde{w}}\colon I \to  X \times F$ ein weiterer Lift mit Anfangspunkt $(w(0),f)$. Weil  $I$ zusammenhängend ist, ist $\pr_F \circ  \tilde{\tilde{w}} \colon  I \to  F$ konstant, weil das Bild zusammenhängend, $F$ aber diskret ist. Also folgt bereits
     \[
         \tilde{\tilde{w}} (t) = (w(t),f) = \tilde{w}(t)
    .\] 
    , denn die zweite Komponente ergibt sich aus vorherigem Argument, und die erste dann sofort aus der Hebungseigenschaft.

    \underline{2. Fall}: $w$ hat Bild in einer trivialisierenden Umgebung  $U$, d.h. in  $U\subset X$ mit $p|_{p^{-1} (U)}$ trivial. 

    Wir fassen $w$ als Weg  $I \to  U$ auf. Nach Fall 1 existiert $\tilde{w}\colon  I \to  p^{-1} (U)$ mit $\tilde{w}(0) = y$. Dann ist $\tilde{w}\colon  I \to  p^{-1} (U) \hookrightarrow E$ eine Hebung von $w$ entlang  $p\colon  E \to  X$.
    \[
    \begin{tikzcd}
        I \ar{r} \ar[swap]{dr}{w} & p^{-1} (U) \ar{d}{p|_{p^{-1} (U)}} \ar[hook]{r} & E \ar{d}{p}\\
                                  & U \ar{r}& X
    \end{tikzcd}
\]

    Da jede Hebung Bild in $p^{-1} (U)$ hat (Verknüpfung mit $p$ liefert ja einen Weg in  $U$, nämlich  $w$), folgt die Eindeutigkeit auch aus Fall 1.

    \underline{3. Fall}: Allgemeiner Fall. Sei $\left \{U_i\right\} _{i \in I}$ eine offene Überdeckung von $X$, so dass  $p|_{p^{-1} (U_i)}$ trivial ist für alle $i\in I$.

    Dann ist $\left \{w^{-1}(U_i)\right\}_{i \in I} $ eine offene Überdeckung von $I$. Es gibt also eine Lebesgue-Zahl  $ε>0$, d.h. ein  $ε>0$, sodass jeder  $ε$-Ball  $U(t,ε)\subset I$ in einem $w^{-1}(U_j)$ liegt. Also finden wir ein $n\in \N$, so dass 
    \[
      \forall k=0,\ldots,n-1 \; \exists i\in I \colon  \quad  \left[ \frac{k}{n}, \frac{k+1}{n} \right] \subset w^{-1}(U_i)
    \]
    d.h. analog, dass
    \[
    w|_{\left[ \frac{k}{n}, \frac{k+1}{n} \right] } \colon  \left[ \frac{k}{n}, \frac{k+1}{n} \right] \to  X
    .\] 
    hat Bild in einer trivialisierenden Umgebung $U_i$.

\begin{figure}[ht]
    \centering
    \incfig{hebung-auf-einzelnen-offenen-mengen}
    \caption{Zerlegung des Weges in trivialisierende Umgebungen, auf denen wir heben können}
    \label{fig:hebung-auf-einzelnen-offenen-mengen}
\end{figure}

    Wir zeigen per Induktion, dass $w|_{\left[ 0,\frac{k}{n} \right] }\colon  \left[ 0, \frac{k}{n} \right]  \to  X$ einen eindeutigen Lift $\tilde{w}$ mit Anfangspunkt $y$ hat.

     \underline{IA}: $k=1$ ist genau die Aussage von Fall 2, wir sind also fertig. 

     \underline{IS} Es gelte die Aussage für $k$. Sei  $\tilde{w}\colon  \left[0,\frac{k}{n}\right]\to  E$ die eindeutige Hebung von $w|_{\left[0,\frac{k}{n}\right]}$ und setze $y_k \coloneqq  \tilde{w}\left( \frac{k}{n} \right) $. Nach Fall 2 hat also der Weg
     \[
     w|_{\left[ \frac{k}{n}, \frac{k+1}{n} \right] }\colon  \left[ \frac{k}{n}, \frac{k+1}{n} \right] \to  X
     .\] 
     einen eindeutigen Lift $\tilde{w}'$ mit $\tilde{w}'\left( \frac{k}{n} \right) = \tilde{w}\left(\frac{k}{n}\right)$. Nun passen $\tilde{w}$ und $\tilde{w}'$ zusammen zu einer Hebung 
     \[
     w|_{\left[ 0, \frac{k+1}{n} \right] }
     .\] 
     zusammen. Zudem ist diese Hebung eindeutige, denn das Anfangsstück auf $\left[0, \frac{k}{n}\right]$ ist nach Induktion schon eindeutig, also auch der Anfangspunkt  $y_k$ von  $\tilde{w}'$, und somit auch $\tilde{w}'$ nach Fall 2.
\end{proof}

\begin{remark*}
    Das Lebesgue-Lemma fand sich auf den Übungsblättern als \autoref{aufgabe-5.4}, wir geben dies im folgenden wieder:
\end{remark*}

\begin{definition*}[Lebesguezahl]\label{def:lebesguezahl}
    Sei $\mathcal{U}$ eine (offene) Überdeckung eines metrischen Raumes $X$. Dann ist eine  $ε>0$ eine  \vocab{Lebesguezahl}, falls $\forall x\in X$ eine (offene) Umgebung $U\in \mathcal{U}$ mit $U(x,ε) \subset U$.
\end{definition*}

\begin{lemma*}[Lebesgue-Lemma]\label{lm:lebesgue}
    Ist $X$ eine kompakter metrischer Raum,  $\mathcal{U}$ eine offene Überdeckung. Dann existiert eine Lebesguezahl $ε>0$.
\end{lemma*}

\begin{remark*}
    Auf dem Übungsblatt haben wir das Lemma für einen \textit{folgenkompakten} metrischen Raum gezeigt, und das ist auch die Eigenschaft, die wir im Beweis verwenden. Allerdings ist \autoref{aufgabe-5.4} auch genau dazu da, zu zeigen, dass Folgenkompaktheit und Kompaktheit für metrische Räume äquivalent sind, in der Formulierung des Lebesgue-Lemmas ist dies alos nicht (mehr) wichtig, sobald wir das wissen.
\end{remark*}

\begin{remark}
    Ist $w\colon  I \to  X$ eine Schleife, so ist $\tilde{w}$ im Allgemeinen trotzdem keine Schleife, hierzu betrachte wieder die Überlagerung $\R \stackrel{\exp }{\longrightarrow} S^1$ und die Schleife $w$, die einmal um den Kreis läuft, die Hebung ist in $\R$ jedoch einfach ein Weg von $2\pi k$ zu $2\pi(k+1)$.


    \begin{minipage}{\textwidth}
    \centering
    \incfig{hebung-von-schleife-zu-weg}
    \captionof{figure}{Hebung der Schleife in $S^1$ zu einem Weg in  $\R$}
    \label{fig:hebung-von-schleife-zu-weg}
    \end{minipage}
\end{remark}

\begin{theorem}[Homotopieliftungssatz]\label{thm:homotopieliftungssatz}
    Sei $p\colon  E \to  X$ eine Überlagerung, und seien $\tilde{w}_0, \tilde{w}_1\colon  I \to  E$ Wege mit $\tilde{w}_0(0) = \tilde{w}_1(0)$, also gleichem Anfangspunkt. Sei $w_i = p \circ  \tilde{w}_i$.

    Sei $H\colon  I \times I \to  X$ eine Homotopie von $w_0$ nach $w_1$ relativ Anfangspunkten.

    \begin{enumerate}[i)]
        \item Es gibt eine eindeutige Hebung $\tilde{H}\colon  I \times I \to  E$ von $H$ mit  $\tilde{H}(0,0) = \tilde{w}_0(0)$.
        \item $\tilde{H}$ ist eine Homotopie von $\tilde{w}_0$ nach $\tilde{w}_1$ relativ Anfangspunkten.
        \item Ist $H$ eine Homotopie relativ Endpunkt, so auch  $\tilde{H}$.
    \end{enumerate}
\end{theorem}


\begin{figure}[ht]
    \centering
    \incfig{hebung-von-homotopie}
    \caption{Hebung der Homotopie $H$ in den Überlagerungsraum  $E$}
    \label{fig:hebung-von-homotopie}
\end{figure}


\begin{proof}[Beweis von \autoref{thm:homotopieliftungssatz}]
    \begin{description}
        \item[Existenz von $\tilde{H}$]. Sei $\left \{U_i\right\} $ eine offene Überdeckung von $X$, so dass  $p$ über jedem  $U_i$ trivial ist. Dann ist  $\left \{H^{-1}(U_i)\right\} _{i \in I}$ eine offene Überdeckung von $I^2$. Nach dem Lebesguelemma existiert $m\in \N$, so dass jedes Quadrat
            \[
            Q_{i,j} \coloneqq  \left[ \frac{i-1}{m}, \frac{i}{m} \right] \times \left[ \frac{j-1}{m}, \frac{j}{m} \right] 
            .\] 
            für $i,j = 1,\ldots,m$.
            \todo{Ausführlichere anwendung von lebesgue-lemma}
            \todo{$i$ wird doppelt benutzt.}
            Wir definieren im Folgenden stetige Abbildungen $\tilde{H}_{ij}\colon  Q_{ij} \to  E$, so dass
            \begin{itemize}
                \item $ p \circ  \tilde{H}_{ij} = H|_{Q_{ij}}$ 
                \item $\tilde{H}_{11}(0,0) = \tilde{w}_0(0) = \tilde{w}_1(0)$
                \item $(\tilde{H}_{ij})|_{Q_{ij} \cap  Q_{i'j'}} = (\tilde{H}_{i'j'})|_{Q_{/j} \cap  Q_{i'j'}}$ (die Abbildungen passen an den Rändern zusammen)
            \end{itemize}
    Das machen wir induktiv in der Reihenfolge $11,12,\ldots$,$1m, 21,22$,\ldots,$2m, \ldots, 3m$, \ldots,$mm$.

    Es liegt $H_{Q_{ij}}$ in einem $U$. Sei  $s\colon  U \to  E$ eine lokale Umkehrfunktion mit
    \[
        s\left(H\left( \frac{i-1}{m}, \frac{j-1}{m} \right) \right) = \begin{cases}
            \tilde{w}_0(0) & i=j=1 \\
            \tilde{H}_{i-1,j}\left( \frac{i-1}{m}, \frac{j-1}{m} \right) & i>1 \\
            \tilde{H}_{i,j-1}\left( \frac{i-1}{m}, \frac{j-1}{m} \right) & j>1
        \end{cases}
    .\] 
    - wir wollen also, dass der Funktionswert an der linken unteren Ecke bereits stimmt.
    \begin{remark}
        Da $\tilde{H}_{i-1,j}$ und $ \tilde{H}_{i,j-1}$ auf $\left[ \frac{i-1}{m}, \frac{j-1}{m} \right) $ übereinstimmen, ist das wohldefiniert für $i,j > 1$.
    \end{remark}
    Setze $\tilde{H}_{ij}\coloneqq  s \circ  H|_{Q_{ij}}$. Per Definition ($s$ ist eine Umkehrfunktion) ist nun  $p \circ  \tilde{H}_{ij} = p \circ  s \circ  H\mid _{Q_{ij}} = H\mid _{Q_{ij}}$.
    \begin{claim}
        Ist $i>1$, so ist  $\tilde{H}_{ij}|_{Q_{ij} \cap  Q_{i-1,j}} = \tilde{H}_{i-1,j} |_{Q_{ij} \cap Q_{i-1,j}}$
    \end{claim}
    \begin{subproof}
        Es ist $Q_{ij} \cap  Q_{i-1,j} = \left \{\left( \frac{i-1}{m}, \frac{j-t}{m} \mid t\in I \right) \right\} $.
        
        Die Wege
        \begin{IEEEeqnarray*}{rCl}
            t &\mapsto &\tilde{H}_{ij}\left( \frac{i-1}{m}, \frac{j-t}{m} \right)  \\
            t & \mapsto & \tilde{H}_{i-1,j} \left( \frac{i-1}{m}, \frac{j-t}{m} \right) 
        \end{IEEEeqnarray*}
        heben $t \mapsto H\left( \frac{i-1}{m}, \frac{j-t}{m} \right) $ und stimmen für $t=0$ überein. Nach dem \nameref{thm:weghebungssatz} sind sie also gleich.

        Analog zeigen wir: Ist $j>1$, so ist  $\tilde{H}_{ij}\mid _{Q_{ij} \cap  Q_{i,j-1}} = \tilde{H}_{i,j-1}\mid_{ Q_{ij}\cap  Q_{i,j-1}}$
\end{subproof}
\item[Eindeutigkeit] Sei $\tilde{\tilde{H}}$ eine weitere Hebung mit $\tilde{\tilde{H}} (0,0) = \tilde{w}_0(0)$. Sei $(t,s) \in I^2$ und $w$ ein Weg von  $(0,0)$ nach  $(t,s)$ in  $I^2$ (z.B. der lineare Weg). Dann sind $\tilde{H} \circ w$ und $\tilde{\tilde{H}} \circ  w$ Hebungen von $H\circ  w$. mit demselben Anfangspunkt. 

    Nach der Eindeutigkeit im \nameref{thm:weghebungssatz} sind also auch schon $\tilde{H} \circ w$ und $\tilde{\tilde{H}} \circ w$ gleich, insbosender stimmen sie an $(t,s)$ überein, und somit  $\tilde{\tilde{H}}(t,s) = \tilde{H}(t,s) $. Da $(t,s)\in I^2$ beliebig war, folgt also wie gewünscht $\tilde{H} = \tilde{\tilde{H}} $.
\end{description}
\begin{enumerate}[i)]
    \setItemnumber{2}
\item Der Weg $\tilde{H}(-,0)$ hebt $H(-,0)= w_0$ mit Anfangspunkt $\tilde{w}_0(0)$. Auch $\tilde{w}_0$ ist ein solcher Lift. Aus der Eindeutigkeit im \nameref{thm:weghebungssatz} folgt also genau $\tilde{H}(-,0) = \tilde{w}_0$.
    \missingfigure{Illustration der verschiedenen Wege}
    Weiter hebt $\tilde{H}(0,-)$ den Weg $H(0,-) = c_{w_0(0)}$, weil $H$ eine Homotopie relativ Anfangspunkt ist. Auch  $c_{\tilde{w}_0(0)}$ ist eine solche Hebung (mit gleichem Anfangspunkt), also ist bereits $\tilde{H}(0,-) = c_{\tilde{w}_0(0)}$.

    Damit folgt bereits, dass $\tilde{H}$ eine Homotopie relativ Anfangspunkt ist, und dass $\tilde{H}(0,1) = c_{\tilde{w}_0(0)}$. Also hebt der Weg $\tilde{H}(-,1)$ den Weg $H(-,1) = w_1$ \textit{mit Anfangspunkt} $\tilde{w}_0(0) = \tilde{w}_1(0)$, und wegen der Eindeutigkeit der Wegeliftung erhalten wir wie gewünscht $\tilde{H}(-,1) = \tilde{w}_1$.
\item Der Weg $\tilde{H}(1,-)$ hebt nun $H(1,-)$. Ist  $H(1,-)$ konstant, dann auch  $\tilde{H}(1,-)$, weil wir auch die konstante Hebung haben, und die Hebung eindeutig ist.
\end{enumerate}
\end{proof}

\section{Beispiele für $\pi_1$}

\begin{definition}
    Ein topologischer Raum $X$ heißt  \vocab[Topologischer Raum!einfach zusammenhängend]{einfach zusammenhängend}, falls $X$ wegzusammenhängend ist und je zwei Wege mit gleichen Anfangs- und Endpunkten homotop sind (relativ Anfangs- und Endpunkt).
\end{definition}

\begin{lemma}
    Ein Raum $X$ ist einfach zusammenhängend genau dann, wenn er wegzusammenhängend ist und  $\pi_1(X,x)$ trivialist für ein (oder äquivalent alle) $x\in X$.
\end{lemma}

\begin{proof}
'$\implies$' Spezialfall der Definition.    

'$\impliedby$' Seien $w, w'\colon I \to X$ Wege mit $w(0) = w'(0) =x$ und  $w(1) = w'(1) = y$. Dann ist 
 \[
     w \simeq (w' \star  \overline{w'}) \star w \simeq w' \star (\overline{w'} \star w) \stackrel{\pi_1(X,y) = 0}{\simeq} w'
.\] 
\end{proof}


    %! TEX root = ./master.tex
\lecture[]{Do 17 Jun 2021 10:15}{}

\begin{notation*}
    Sei $p\colon  E \to  X$ eine Überlagerung, $w\colon  I \to  X$ ein Weg mit $w(0) = x$ und  $e\in p^{-1} (x)$. Dann notieren wir mit $L(w,e)$ die Hebung von  $w$ mit Anfangspunkt  $e$, die nach dem \autoref{thm:weghebungssatz } eindeutig existiert.
\end{notation*}

\begin{theorem}\label{thm:fundamentalgruppe-durch-überlagerung-mit-einfach-zusammenhängendem-raum}
    Sei $p\colon  E \to  X$ eine Überlagerung, wobei $E$ einfach zusammenhängend sei. Sei $x\in X$ und $e\in p^{-1} (x)$. Dann ist die Abbildung
        \begin{equation*}
        \varphi : \left| \begin{array}{c c l} 
            \pi_1(X,x) & \longrightarrow & p^{-1} (x) \\
            \left[w\right] & \longmapsto &  L(w,e)(1)
        \end{array} \right.
    \end{equation*}
    wohldefiniert und bijektiv.
\end{theorem}

\begin{proof}
    \begin{description}
        \item[Wohldefiniertheit:] Angenommen, $w \simeq w'$ relativ Anfangs- und Endpunkt. Nach dem  \autoref{thm:homotopieliftungssatz} ist dann auch $L(w,e) \simeq L(w',e)$ homotop relativ Anfangs- und Endpunkt. Insbesondere haben sie denselben Endpunkt (dieser bleibt während der Homotopie ja konstant), und somit  $L(w,e)(1) = L(w',e)(1)$. 
        \item[Injektivität:] Angenommen, $[w], [w']\in \pi_1(X,x)$ werden auf den gleichen Endpunkt $L(w,e)(1) = L(w',e)(1)$ abgebildet. 

            Da $E$ einfach zusammenhängend, sind nun  $L(w,e)$ und  $L(w',e)$ homotop (sie haben den gleichen Anfangs- und Endpunkt) relativ Endpunkten. Sei $H$ eine solche Homotopie. Dann ist  $p \circ  H$ eine Homotopie von $w$ nach  $w'$ relativ Anfangs- und Endpunkt, also  $[w] = [w']$ und  $\varphi $ ist wie gewünscht bijektiv.
        \item[Surjektivität:] Sei $e' \in p^{-1} (x)$. Da $E$ als einfach zusammenhängender Raum insbesondere wegzusammenhängend ist, gibt es einen Weg  $\tilde{w}\colon I \to  E$ von $e$ nach  $e'$. Dann ist $w \coloneqq  p \circ  \tilde{w}$ eine Schleife an $x$, weil  $e$, $e' \in p^{-1} (x)$, also ist $\varphi ([w]) = L(w,e)(1) = \tilde{w}(1)$ = e'. Da $e'\in p^{-1} (x)$ beliebig war, ist $\varphi $ surjektiv.
    \end{description}
\end{proof}

\begin{oral}
    Mit dieser Bijektion haben wir ein erstes starkes Werkzeug, mit der wir - mittels geschickter Überlagerungen - schon einmal die Mächtigkeit der Fundamentalgruppe bestimmen können.
\end{oral}

\begin{theorem}\label{thm:fundamentalgruppe-von-s1-kreis}
    Es ist $\pi_1(S^1,1) \cong \Z$.
\end{theorem}

\begin{proof}
    Wir betrachten die Überlagerung $\exp \colon  \R \to  S^1$. Zudem ist $\R$ einfach zusammenhängend. Wir wählen das Urbild $0\in \exp ^{-1}(1)$. Nach \autoref{thm:fundamentalgruppe-durch-überlagerung-mit-einfach-zusammenhängendem-raum} ist nun
        \begin{equation*}
        \varphi : \left| \begin{array}{c c l} 
            \pi_1(S^1,1) & \longrightarrow & \exp ^{-1}(1) = \Z \subset \R \\
            \left[w\right] & \longmapsto &  L(w,0)(1)
        \end{array} \right.
    \end{equation*}
    eine Bijektion.
    \begin{claim}
        $\varphi $ ist ein Gruppenhomomorphismus.
    \end{claim}
    \begin{subproof}
    Seien $[w],[w'] \in \pi_1(S^1,1)$. Es ist zunächst
    \[
        L(w\star w', 0) = L(w,0) \star L(w',L(w,0)(1))
    .\]
    und zudem
    \[
        L(w',L(w,0)(1))(t) = L(w',0)(t) + L(w,0)(1)
    .\] 
    , denn ist $L(w',0)$ eine Hebung, so auch  $L(w.,0)+n$ für alle  $n\in \Z$, da $\exp (t+n) = \exp (t)$. Also ist 
    \[
        \varphi ([w] \circ  [w']) =         L(w \star w',0)(1) = L(w',0)(1) + L(w,0)(1) = \varphi ([w]) + \varphi ([w'])
    .\] 
    also ist $\varphi $ tatsächlich ein Gruppenhomomorphismus.
    \end{subproof}
    Also ist $\varphi$ bijektiv und ein Gruppenhomomorphismus, also schon ein Isomorphismus von Gruppen.
\end{proof}
\todo{Anmerkung mit Ende von Wegen einfügen}

\begin{oral}
    Es ist hier ein bisschen Glück bzw. Zufall, dass die Bijektion $\varphi $ sogar ein Gruppenhomomorphismus ist. Im Allgemeinen wir dies nicht so sein, wir werden uns aber im Zuge von \textit{Gruppenwirkungen} dieser Thematik auch im Allgemeineren noch annähern.
\end{oral}


\begin{remark}
    Ein Erzeuger von $\pi_1(S^1,0)$ ist gegeben durch die Abbildung
    \[
        \exp |_{[0,1)} \colon  [0,1) \to  S^1
    .\] 
\end{remark}
\begin{proof}
    Es ist
    \begin{IEEEeqnarray*}{rCl}
        \varphi (\left[ \exp |_{[0,1)} \right] ) & = & L\left( \exp |_{[0,1)},0 \right)(1) \\
                                                 & = & (t \mapsto t)(1) \\
                                                 & = & 1
    \end{IEEEeqnarray*}

    Allgemein kann man so auch zeigen, dass die Abbildung
    \[
        \varphi  (t\mapsto \exp (tk)) = k
    .\] 
\end{proof}

\begin{theorem}\label{thm:r^n-einfach-zusammenhängend}
    $\forall n\in \N$ ist $\R^n$ einfach zusammenhängend, insbesondere $\pi_1(\R^n,0) = 0$.
\end{theorem}

\begin{remark*}
    Vergleiche hierzu auch  \autoref{aufgabe-8.4}, hier zeigen wir das gleiche Resultat, aber mit einem etwas anderen Weg.
\end{remark*}

\begin{theorem}\label{thm:s^n-einfach-zusammenhängend-wenn-n-geq-2}
    Sei $n\in \N$ mit $n\geq 2$. Dann ist $S^n$ einfach zusammenhägend, insbesondere $\pi_1(S^n,1) = 0$. 
\end{theorem}

Als kleine Vorbereitung benötigen wir:

\begin{lemma*}\label{lm :s^n-ohne-punkt-ist-r^n}
Sei $z\in S^n$ beliebig, dann ist $S^n \setminus \left \{z\right\} \cong\R^n$
\end{lemma*}

\begin{proof}
    Wir führen eine stereographische Projektion durch, d.h. wir definieren die Abbildung
        \begin{equation*}
        \varphi : \left| \begin{array}{c c l} 
        S^n \setminus \left \{z\right\}  & \longrightarrow & \left< z \right> ^{\bot} \\
        x & \longmapsto &  z - \frac{1}{\left< x-z,z \right> }(x-z)
        \end{array} \right.
    \end{equation*}
Hierbei ist $\left< z \right> ^{\bot}$ der Unterraum der Dimension $n$ von  $\R^{n+1}$, auf dem $z$ senkrecht steht, und $x$ wir dann abgebildet auf den Schnittpunkt der Geraden durch  $x$ und  $z$ mit diesem Unterraum.

    Ein Alternativer Beweis wäre, zu verwenden, dass
    \[
    S^n \setminus \left \{z\right\} \cong D^n \setminus \left \{0\right\} / \partial D^n \cong  \R^n
    .\] 
    wobei wir im letzten Schritt $x \mapsto  \frac{1-\lVert x \rVert }{\lVert x \rVert^2 }x$ abbilden.
\end{proof}
\todo{Diesen Beweis ausschreiben}

\begin{remark*}[Wie kommt man auf die Formel der stereographischen Projektion?]
Wir wollen für $z\in S^n$ ein $x\neq z \in S^n$ abbilden auf den Schnittpunkt der Geraden durch $x,z$ und dem zu  $z$ senkrecht stehenden Unterraum $\left< z \right> ^{\bot} \cong \R^n$. Bezeichnen wir das Bild von $x$ unter dieser Abbildung mit  $y$, so ergeben sich folgende beiden Bedingungen
 \begin{itemize}
    \item $\left< x,z \right> =0$, damit $x$ senkrecht zu  $z$ steht, also in der entsprechenden Hyperebene  $\cong \R^n$.
    \item Es sind $x,y,z$ kollinear, d.h. es existiert ein  $λ\in \R$ mit
        \[
            y = z + λ(x-z)
        .\] 
        , indem wir die entsprechende Gerade durch den Fußpunkt $z$ und den Richtungsvektor  $(x-z)$ mit  $λ$ parametrisieren.
\end{itemize}
Einsetzen ineinander ergibt die Bedingung
\begin{IEEEeqnarray*}{rCl}
    0 & = & \left< y,z \right> \\
      & = & \left< z + λ(x-z),z \right> \\
      & = & \left< z,z \right> + λ \left< x-z,z \right>  \\
       & = & 1 + λ\left< x-z,z \right> 
\end{IEEEeqnarray*}
was sich äquivalent umformt zu
\[
λ = -\frac{1}{\left< x-z,z \right> }
.\] 
weswegen wir die obige Form der Abbildung erhalten.
\end{remark*}



\begin{proof}[Beweis von \autoref{thm:s^n-einfach-zusammenhängend-wenn-n-geq-2}]
    \underline{Schritt 1}: Sei  $w\colon I \to  S^n$ eine Schleife an $x$, so dass das Bild von  $w$ nicht gleich  $S^n$ ist.
    \begin{claim}
        Dann ist $w$ homotop relativ Endpunkten zur konstanten Schleife.
    \end{claim}
    \begin{subproof}
        Sei $z\in S^n \setminus \Bild(z)$, solch ein Punkt existiert nach Voraussetzung. Wegen \autoref{lm :s^n-ohne-punkt-ist-r^n}  können wir $w$ als Schleife in  $\R^n$ auffassen, und $w$ ist somit homotop relativ endpunkten zur konstanten Schleife.
    \end{subproof}
    \underline{Schritt 2}: Sei $w$ eine beliebige Schleife an $x$. Wir wollen $x$ in Teile zerlegen, die nicht als Bild die gesamte Kugel haben, um auf Schritt 1 zu reduzieren.

    OBdA sei hierzu  $x\neq (0,0,\ldots,1)$ und auch $x\neq  (0,0,\ldots,-1)$, sonst rotiere die Sphäre. D.h. $x$ ist nicht der 'Nord-' oder 'Südpol' der Kugel. Nun betrachte
     \[
         S^n = \underbrace{S^n \setminus \left \{(0,0,\ldots,1)\right\}}_{\coloneqq  U_1} \cup \underbrace{S^n \setminus \left \{(0,0,\ldots,-1\right\}}_{\coloneqq  U_2} 
    .\] 
    Zudem ist
    \[
    V \coloneqq  U_1 \cap U_2 = S^n \setminus \left \{(0,0,\ldots,1) , (0,0,\ldots,-1)\right\}  \cong S^{n-1}\times \R
    .\] 
    wegzusammenhängend für $n\geq 2$. Nach dem Lebesguelemma existiert $n\in \N$, sodass
    \[
        \forall i\leq n-1 \exists k\in \left \{1,2\right\} \text{ mit } w\left( \left[ \frac{i}{n},\frac{i+1}{n} \right]  \right) \subset U_k
    .\] 
    Sei $0 = j_0 \leq  j_1 \leq  \ldots \leq j_l = n$ mit $j_i \in (0,\ldots,n)$, sodass
    \[
        w\left( \frac{j_i}{n} \right) \in V \text{ und } w\left( \left[ \frac{j_i}{n}, \frac{j_i+1}{n} \right]  \right) \in  U_k
    .\] 
    d.h. die $j_i$ sind diejenigen Übergangspunkte, die in  $V$ liegen. Nach  \autoref{thm:r^n-einfach-zusammenhängend} ist $w|_{\left[ \frac{j_i}{n}, \frac{j_i+1}{n} \right] }$ relativ Endpunkten homotop zu einem Weg in $V$, hierzu wählen wir einen Weg in  $V$, der Anfangs- und Endpunkt verbindet, und beide Wege liegen dann in einem $U_k$, also einem einfach zusammenhängendem Raum.

    Zusammensetzen liefert dies, dass $w$ homotop ist relativ Endpunkten zu einer Schleife in  $V$. Nach Schritt 1 ist somit $w$ homotop relativ Endpunkten homotop zur konstanten Schleife, denn  $V \subsetneq S^n$.
\end{proof}

\begin{oral}
    Der Beweis scheitert für $S^1$, weil wir wir zwar auch  $U_1,U_2\subset S^1$ als einfach zusammenhängende Teile konstruieren können, allerdings der Schnitt $V = U_1 \cap U_2$ nicht mehr zusammenhängend ist. Wir können also in $U_1,U_2$ jeweils Teilstücke des Weges zusammenziehen, diese aber nicht nach $V$ bringen, weswegen uns das nichts nützt.
\end{oral}

\begin{remark*}
    Es gibt tatsächlich Wege $w\colon [0,1] \to  S^n$, die als Bild die gesamte Sphäre haben. Im ersten Moment erscheint das unintuitiv, weil das von den Dimensionen nicht passt, allerdings gibt es sogenannte \textit{Raumfüllende Kurven}. Siehe hierzu auch \href{https://en.wikipedia.org/wiki/Space-filling_curve}{https://en.wikipedia.org/wiki/Space-filling\_curve}
\end{remark*}

\section{Überlagerungen Teil 2}
\begin{definition}
    Ein Raum $X$ heißt  \vocab{lokal wegzusammenhängend}, falls für jeden Punkt $x\in X$ und jede Umgebung $U$ von  $x$ eine Umgebung  $V\subset U$ existiert, die wegzusammenhängend ist. 
\end{definition}

\begin{example}
\begin{warning}
    Es gibt wegzusammenhängende Räume, die nicht lokal wegzusammenhängend sind.
\end{warning}
Hierzu betrachte wieder die Sinuskurve des Topologen, füge aber einen weiteren Weg ein, der die beiden Wegzusammenhangskomponenten verbindet, also in etwa so:

\begin{minipage}{\textwidth}
    \centering
            \begin{tikzpicture}[domain=0.001:1, xscale = 6]
                \draw[color=blue!30!white,smooth,samples=100,domain=0.001:0.01,line width = 0.1pt] plot[id=gnuplots/topologists-sine-curve-1] function{sin(1/x)};
                \draw[color=blue!30!white,smooth,samples=1000,domain=0.01:0.1, line width = 0.1pt] plot[id=gnuplots/topologists-sine-curve-2] function{sin(1/x)};
                \draw[color=blue!30!white,smooth,samples=100,domain=0.1:1, line width = 0.1pt] plot[id=gnuplots/topologists-sine-curve-3] function{sin(1/x)};
                \draw[color=red,thick] (0,-1) -- (0,1);
                \draw[->] (0,0) -- (1,0);
                \foreach \x in {1,2,3,4,5,6,7,8,9} {
                    \draw (0.1*\x,-0.1) node[anchor=north]{0,\x} -- (0.1*\x, 0.1);
                }
                \draw (-0.01,-1) node[anchor = east] {-1} -- (0.01,-1);
                \draw (-0.01,1) node[anchor = east] {1} -- (0.01,1);
            \end{tikzpicture}
\end{minipage}
TODO: linie hinzufügen
\end{example}

\begin{example}
    Das ganze gilt natürlich erst recht nicht in die andere Richtung, es ist z.B. $S^0 = \left \{-1,1\right\} $ lokal wegzusammenhängend, aber sicherlich nicht wegzusammenhängend.
\end{example}


\begin{theorem}\label{thm:wegkomponenten-in-lokal-wegzusammenhängendem-raum-sind-offen}
    Sei $X$ lokal wegzusammenhängend. Dann sind alle Wegekomponenten offen in  $X$.
\end{theorem}

\begin{corollary}\label{cor:lokal-wegzusammenhängende-räume-sind-koprodukt-ihrer-wegkomponenten}

    Sei $X$ lokal wegzusammenhängend. Dann ist  $X$ die disjunkte Vereinigung seiner Wegekomponenten (topologische gesehen, nicht nur als Mengen).
\end{corollary}\label{cor:lokal-wegzusammenhängende-räume-sind-koprodukt-ihrer-wegkomponenten}

\begin{proof}[Beweis von \autoref{} ]
    Sei $C\subset X$ eine Wegkomponente von $X$, und $x\in C$. Da $X$ lokal wegzusammenhängend ist, hat  $x$ eine wegzusammenhängende Umgebung  $V$. Dann folgt $V\subset C$, weil $C$ die größte wegzusammenhängende Umgebung von  $x$ ist, also enthält  $C$ eine Umgebung von  $x$, d.h.  $C$ ist Umgebung aller inneren Punkte, und somit ist  $C$ offen in  $X$.
\end{proof}

\begin{theorem}\label{thm:überlagerung-über-lokal-wegzusammenhängendem-raum-zerfällt-in-wegzusammenhängende-komponenten-von-e}
    Sei $p\colon E\to X$ eine Überlagerung und $X$ lokal wegzusammenhägend sowie wegzusammenhängend.
    \begin{enumerate}[1)]
        \item Dann ist $E$ lokal wegzusammenhängend. 
        \item Sei $C\subset E$ eine Wegekomponenten. Dann ist auch
            \[
            p|_C \colon  C \to X
            .\] 
            eine Überlagerung.
    \end{enumerate}
\end{theorem}

\begin{oral}
    Eigenschaft 1) gilt nicht mit Wegzusammenhang, dazu betrachte als Gegenbeispiele eine triviale Überlagerung $S^1 \times F \to  S^1$ für $F\neq \left \{\star\right\} $.
\end{oral}

\begin{proof}
    \begin{enumerate}[1)]
        \item Sei $e\in E$ und $e\in U$ eine Umgebung. Sei $U' \subset E$ offen, $e\in U'$, so dass
            \[
                p|_{U'}\colon U' \to p(U')
            .\] 
            ein Homöomorphismus ist und $p(U')\subset X$ offen (das existiert, weil $p$ ein lokaler Homöomorphismus ist). Sei  $U'' \coloneqq  U \cap U'$. Dann ist $p(U'')$ eine Umgebung von  $p(e) \in X$. Da $X$ lokal wegzusammenhängend existiert eine Umgebung  $V'$ von  $p(e)$ mit  $v'\subset p(U'')$, und da $p|_{U'}$ ein lokaler Homöomorphismus, ist $V \coloneqq  p^{-1} (V')$ eine wegzusammenhängende Umgebung von $e$.
            \todo{Skizze zum Beweis}
        \item  $E$ ist nach 1) lokal wegzusammenhängend. Dann ist aber bereits  $C\subset E$ offen nach \autoref{thm:wegkomponenten-in-lokal-wegzusammenhängendem-raum-sind-offen}.

            Sei $x\in X$ beliebig, da $p\colon  E \to  X$ eine Überlagerung ist, existiert eine Umgebung $U$ von  $x$, über der  $p$ trivial ist. Da $X$ lokal wegzusammenhängend finden wir  eine Umgebung $V\subset U$ von $x$, sodass  $V$ wegzusammenhängend. Es kommutiert dann auch
            \[
            \begin{tikzcd}
                p^{-1} (V) \ar{rr}{u} \ar[swap]{dr}{} & & V\times p^{-1} (x) \ar{dl}{} \\
            & V
            \end{tikzcd}
            \]
            weil das Ganze schon über $U$ galt. Da  $V$ wegzusammenhängend und  $C$ eine Wegekomponente ist, liegt  $u^{-1}(V\times \left \{e\right\} )$ entweder ganz in $C$ oder ganz in  $E\setminus C$. Also ist
            \[
                p|_{C}^{-1}(V) = \bigcup_{e\in p^{-1} (x)\cap C} u^{-1}(V\times \left \{e\right\} ) = u^{-1}(V\times p|_{C}^{-1}(x)) 
            .\] 
            und somit ergibt sich das Diagramm:
            \[
            \begin{tikzcd}
                p\mid _C^{-1}(V) \ar{rr}{u}[swap]{\cong} \ar[swap]{dr}{} & & V\times p|_{C}^{-1}(x) \ar{dl}{} \\
            & V
            \end{tikzcd}
            \]
            Es bleibt noch zu zeigen, dass $p|_C$ auch tatsächlich surjektiv ist. Es ist  $C\neq \emptyset$, weil es sich um eine Wegekomponente handelt. Sei $c\in C$ und $x\in X$, wir wollen ein Urbild von $X$ finden. Sei  $w$ ein Weg von $p(c)$ nach  $x$, dieser existiert, weil  $X$ wegzusammenhängend ist. Sei  $\tilde{w}\colon  I \to E$ ein Lift von $w$ mit Anfangspunkt $c$. Dann ist  $\tilde{w}(t) \in C$ für alle $t$, weil  $C$ eine Wegekomponente ist, und somit
             \[
                 p|_C(\tilde{w}(1)) = x
            .\] 
            und somit ist $p|_C$ surjektiv.
    \end{enumerate}
\end{proof}

\begin{theorem}[Allgemeiner Liftungssatz]
    Sei $p\colon  E \to X$ eine Überlagerung, $x_0\in X$, $e_0\in p^{-1} (x_0)$. Sei $Y$ wegzusammenhängend und lokal wegzusammenhängend. Sei  $y_0\in Y$ und $f\colon  (Y,y_0) \to  (X,x_0)$ eine punktierte Abbildung.

    Dann sind äquivalent:
    \begin{enumerate}[1)]
        \item Es gibt eine Hebunf $\tilde{f}\colon  Y \to  X$ mit $\tilde{f}(y_0) = e_0$
        \item Es ist $f_*(p_1(Y,y_0)) \subset p_*\pi_1(E,e_0))$
    \end{enumerate}
    Ist eine der Bedingungen erfüllt, so ist $\tilde{f}$ eindeutig.
\end{theorem}

\begin{proof}
    '1)$\implies$2)' Sei $\tilde{f}\colon (Y,y_0) \to  (E,e_0)$ eine Hebung, also $p \circ  \tilde{f} = f$. Wegen Funktorialität von $\pi_1$ ergibt sich dann
    \[
        f_* = (p \circ \tilde{f})_* = p_* \circ  \tilde{f}_*\colon  p_1(Y,y_0) \to  p_1(X,x_0)
    .\] 
    Also ist
    \[
        f_*(\pi_1(X,x_0)) = (p_* \circ  \tilde{f}_*)(\pi_1(Y,y_0)) \subset p_*(\pi_1(E,e_0))
    .\] 
\end{proof}

    % end lectures
    %\input{fragestunden.tex}
\end{document}
