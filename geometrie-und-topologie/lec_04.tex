\lecture{4}{Do 22 Apr 2021 10:15}{}
\begin{example}
    Zur Frage von letzter Woche (wenn wir einen Hausdorff-Raum haben und eine Äquivalenzrelation, deren Klassen abgeschlossen sind, ist dann der Quotient wieder  Hausdorff?): Wähle auf $[0,1]$ die Relation erzeugt von
     \[
    \frac{1}{n} \sim  1 - \frac{1}{n}
    .\] 
    für alle $n\in \N_>0$. Betrachte dann die Abbildung: 
    \[
        [0,1] \twoheadrightarrow [0,1] / \sim 
    .\] 
    Punkturbilder sind endlich, also abgeschlossen. Aber der Raum $[0,1] / \sim $ ist nicht hausdorffsch, denn wri können die Punkte $0,1$ nicht trennen.
\end{example}
\begin{theorem}
    Sei $X$ ein kompakter Raum und  $Y\subset X$ abgeschlossen. Dann ist $Y$ kompakt.
    \label{thm:closed-subset-of-compact-space-is-compact}
\end{theorem}
\begin{proof}
    Sei $\left \{U_i\right\} _{i \in I}$ eine offene Überdeckung von $Y$. Dann existieren $U_i' \subset X$ offen mit $U_i = U_i' \cap Y$. Die Familie
     \[
    \left \{U_i'\right\} _{i \in I}\cup \left \{\underbrace{X \setminus Y}_{\text{offen}}\right\} 
    .\] 
    ist nun eine offene Überdeckung von $X$. Dann existiert $J\subset I$ endlich, so dass
    \[
    \left \{U_j'\right\} _{j\in J} \cup \left \{X \setminus Y\right\} 
    .\] 
    die Menge $X$ überdeckt. Also ist  
    \[
        \left \{\underbrace{U_j' \cap Y}_{U_j}\right\} _{j\in J} \cup \left \{\underbrace{X \setminus Y \cap Y}_{=\emptyset}\right\}  
    \]
    eine endliche Überdeckung für $Y$.
\end{proof}
\begin{theorem}
    Sei $X$ ein Hausdorff-Raum und  $Y\subset X$ kompakt. Dann ist $Y$ abgeschlossen.
    \label{thm:compact-subset-of-hausdorff-space-is-closed}
\end{theorem}

\begin{corollary}
    Ist $X$ kompakt und Hausdorffsch, dann sind äquivalent:
    \label{thm:compact-iff-closed}
\begin{enumerate}[1)]
        \item $Y\subset X$ ist abgeschlossen
        \item $Y$ ist kompakt.
    \end{enumerate}
\end{corollary}
\begin{lemma}
    Sei $X$ ein Hausdorff Raum und  $Y\subset X$ kompakt. Dann existiert $\forall x\in X\setminus Y$ offene Teilmengen $U_{x,Y}$ und $V_{x,Y}$ von $X$ so dass:  $x\in U_{x,Y}$ und $Y\subset V_{x,Y}$ und $U_{x,Y} \cap V_{x,Y} = \emptyset$.
    \label{lm:compact-set-in-hausdorff-space-is-closed}
\end{lemma}
\begin{proof}
    Sei $x\in X\setminus Y$. $\forall y\in Y$ existieren $U_{x,y}$ und $V_{x,y}$ offen mit $x\in U_{x,y}$ und $y\in V_{x,y}$, weil $X$ Hausdorffsch. \\
    Dann ist  $\left \{V_{x,y} \cap Y\right\} _{y\in Y}$ eine offene Überdeckung von $Y$. Also existiert endliche Teilüberdeckung (da  $Y$ kompakt) induziert durch Punkte  $y_1,\ldots,y_n$. Also:
    \[
    Y\subset \bigcup_{i=1}^n V_{x,y_i}
    .\] 
    Sei
    \[
    V_{x,Y} := \bigcup_{i=1}^n V_{x,y_i} \qquad U_{x,Y} := \bigcap_{i=1}^n U_{x,y_i} 
    .\] 
    Es ist auch $x\in U_{x,Y}$, weil $x\in U_{x,y_i}$ für jedes $i$. Wir müssen also noch Disjunktheit prüfen, es ist:
     \[
    U_{x,Y} \cap V_{x,y_i} \subset U_{x,y_i} \cap V_{x,y_i} = \emptyset
    .\] 
    Also auch
    \[
        \emptyset=    U_{x,Y} \cap \bigcup_{i=1}^n V_{x,y_i} = U_{x,Y} \cap V_{x,Y}
    .\]
\end{proof}
    \begin{figure}[H]
    \includegraphics[scale=0.4]{figures/Lemma5.5.pdf}
    \caption{Skizze zum Beweis von Lemma \ref{lm:compact-set-in-hausdorff-space-is-closed}}
\end{figure}
\begin{proof}[Beweis von \ref{thm:compact-subset-of-hausdorff-space-is-closed}]
    Nach dem Lemma existieren $\forall x\in X \setminus Y$ ein $U_{x,Y}$ mit $x\in U_{x,Y}$ und $U_{x,Y} \cap Y = \emptyset$. Also ist
    \[
    X \setminus Y = \bigcup_{x\in X \setminus Y} U_{x,Y}
    .\] 
    offen und somit ist $Y$ abgeschlossen.
\end{proof}


\begin{example}['Gegenbeispiel' zu Satz \ref{thm:compact-subset-of-hausdorff-space-is-closed}]
    Sei $G$ die Gerade mit zwei Urpsrüngen: \\
    Betrachte  $\R\cup \left \{0'\right\} $ mit $U$ Umgebung von  $a\in \R$ falls $\exists ε>0$ mit $(a-ε, a+ε)\subset U$ und $U$ Umgebung von  $0'$ und  $U$ Umgebung von  $0'$, falls  $\exists ε>0$ mit $(-ε,0 \cup (0,ε) \subset U$ und $0' \in U$. \\
    Wir können uns gewissermaßen  $0,0'$ gleichberechtigt vorstellen, nur dass die beiden Punkte verschieden sind. \\
    Dann ist  $[-1,1] \subset G$ kompakt (Übung!), aber nicht abegschlossen, da $0' \in G \setminus [-1,1]$ ist, dies aber keine Umgebung von $0'$ ist.
    \begin{figure}[H]
        \includegraphics[scale=0.4]{figures/line-with-2-origins.pdf}
    \caption{Gerade mit 2 Ursprüngen}
    \end{figure}
\end{example}


\begin{proof}[Beweis von Satz \ref{thm:heine-borel}]
    '$2) \implies 1)$'. Sei $X\subset \R^n$ kompakt. Dann ist sie abgeschlossen nach \ref{thm:compact-subset-of-hausdorff-space-is-closed}. Zudem ist $X\subset \bigcup_{x\in X} U(x,1)$ eine offene Überdeckung. Da $X$ kompakt finden wir endlich viele  $x_1,\ldots,x_n\in X$ mit
    \[
        X \subset \bigcup_{i=1}^n U(x_i,1)
    .\] 
    Also ist
    \[
        \diam(X) \leq  \max \left \{d(x_i,x_j)\right\} +2 < \infty
    .\] 
    und somit ist $X$ auch beschränkt. \\
    ' $1)\implies 2)$'. Da $X$ beschränkt ist,  $\exists m>0$ mit $X\subset [-m,m]^n\subset \R^n$. Da $X$ abgeschlossen ist, genügt es nach \ref{thm:closed-subset-of-compact-space-is-compact} zu zeigen, dass  $[-m,m]^n$ kompakt ist. \\
    Wir führen einen Widerspruchsbeweis, nimm also an, dass  $[-m,m]^n$ nicht kompakt ist. Dann existiert eine offene Überdeckung  $\left \{U_i\right\} _{i \in I}$ ohne endliche Teilüberdeckung. \\
    Unterteile $[-m,m]^n$ in  $2^n$ gleich große Unterwürfel (halbiere jede Seite). Mindestens ein Unterwürfel hat keine endliche Teilüberdeckung. Unterteile diesen Würfel weiter und wähle wieder einen Unterwüfel, der keine endliche Teilüberdeckung hat. \\
    Wir erhalten eine Folge von Würfeln
     \[
         [-m,m]^n =     Q_0 \supset Q_1 \supset Q_2 \supset Q_3 \supset \ldots
    .\] 
    die jeweils keine endliche Teilüberdeckung durch $U_i's$ besitzen. \\
    Sei  $x_i \in Q_i$ beliebig. Dann ist $x_i$ eine Cauchy-Folge, also existiert $x = \lim_{i\to \infty} x_i$, und $x\in Q_0$, da $Q_0$ abgeschlossen. \\
    Somit gibt es ein $U_j$ mit  $x\in U_j$, da die $\left \{U_i\right\} _{i \in I}$ eine Überedeckung von $Q_0$ waren. Damit ist auch $U(x,ε) \subset U_j$ für ein $ε>0$. Wähle einen Würfel $x\in Q_k$ mit Kantenlänge $< \frac{ε}{\sqrt{n} }$, dann ist auch $Q_k \subset U(x,ε) \subset U_j$. Das ist aber ein Widerspruch dazu, dass $Q_k$ keine endliche Teilüberdeckung hat, \contra. \\
    Also ist  $Q_0$ kompakt.
\end{proof}

\begin{theorem}
    Sei $f: X \to  Y$ stetig und surjektiv und $X$ kompakt. Dann ist auch  $Y$ kompakt. 
    \label{thm:image-of-compact-space-is-compact}
\end{theorem}
\begin{proof}
    Sei $\left \{U_i\right\} _{i \in I}$ offene Überdeckung von $Y$. Dann ist
     \[
         \left \{f^{-1}(U_i)\right\} _{i \in I}
    .\] 
    offene Überdeckung von $X$. Da  $X$ kompakt ist, gibt es  $J\subset I$ endlich mit $X = \bigcup_{j\in J} f^{-1}(U_j)$. Dann ist 
    \[
        Y = f(X) = \bigcup_{j\in J} f(f^{-1}(U_j)) = \bigcup_{j\in J} U_j
    .\] 
    Also existiert eine endliche Teilüberdeckung von $Y$.
\end{proof}
\begin{corollary}
    Sei $f: X \to  Y$ stetig, $X$ kompakt und  $Y$ Hausdorff. Dann ist  $f$ abgeschlossen, d.h. $\forall A\subset X$ abgesclhossen ist $f(A) \subset Y$ abgeschlossen.
\end{corollary}
\begin{proof}
    Sei $A\subset X$ abgeschlossen. Dann ist $A$ kompakt. Also ist  $f(A)$ kompakt. Also ist  $f(A)$ abgeschlossen
\end{proof}
\begin{corollary}
    Ist $f: X \to  Y$ stetig und bijektiv, $X$ kompakt und  $Y$ Hausdorff, dann ist  $f$ ein Homöomorphismus.
    \label{cor:compact-to-hausdorff-is-homeomorphism}
\end{corollary}
\begin{proof}
    Wir müssen zeigen, dass die Umkehrabbildung stetig ist. Dafür reicht es zu zeigen, dass $\forall A\subset X$ abgeschlossen auch $f(A) = (f^{-1})^{-1}(A)$ abgeschlossen ist. Das gilt nach vorherigem Korollar.
\end{proof}
\begin{corollary}
    Sei $f: X \to  Y$ stetig und surjektiv, $X$ kompakt und  $Y$ Hausdorffsch. Dann trägt  $Y$ die Quotiententopologie, d.h.  $U\subset Y$ offen genau dann, wenn $f^{-1}(U) \subset X$ offen.
\end{corollary}
\begin{proof}
    '$\implies$' folgt wegen Stetigkeit. \\
    '$\impliedby$' Ist $f^{-1}(U) \subset X$ offen, dann ist $f^{-1}(Y \setminus U ) = U \setminus f^{-1}(U)$ abgeschlossen in $X$, also folgt aus dem Korollar dass
     \[
         Y \setminus U \stackrel{\text{surj.}}{=}   f\left( f^{-1}\left( Y \setminus U \right)  \right) 
    .\] 
    abgeschlossen ist, also ist $U\subset Y$ offen.
\end{proof}
\begin{proof}[Satz 3.3]
    Schon gezeigt:
        \begin{equation*}
        \begin{array}{c c l} 
            [0,1] & \longrightarrow & S_1 \\
        t & \longmapsto &  2^{2\pi it}
        \end{array}
    \end{equation*}
    ist stetig und surjektiv und faktorisiert über
    \[
        [0,1] /\left \{0,1\right\}  \to  S^1
    .\] 
    mit $f$ stetig und bijektiv. Wir wissen nun:  $S^1$ ist Hausdorffsch und  $[0,1]$ ist kompakt. Nach Satz 5.5 ist auch  $[0,1] /\left \{0,1\right\} $ kompakt, also ist $f$ ein Homöomorphismus nach Korollar 5.8.
\end{proof}


\begin{theorem}
    Jeder kompakte Hausdorff-Raum ist normal.
    \label{thm:compact-hausdorff-space-is-normal}
\end{theorem}
\begin{proof}
    Seien $A,B\subset X$ abgeschlossen und disjunkt. Da $X$ kompakt ist, sind  $A,B$ kompakt. Nach Lemma 5.5 existieren  $\forall a\in A$ offene Mengen $U_a, V_a$ mit $a\in U_a, B\subset V_a$ und $U_a \cap V_a = \emptyset$. Dann ist
    \[
    A \subset \bigcup_{a\in A} U_a
    .\] 
    Also existieren $a_1,\ldots,a_n\in A$ mit
    \[
    A\subset \bigcup_{i=1}^n U_{a_i}
    .\] 
    wegen $A$ kompakt. Setze nun
    \[
    U_A := \bigcup_{i=1}^n U_{a_i}\supset A \qquad U_B := \bigcap_{i=1}^n V_{a_i}\supset B
    .\] 
$\forall i$ ist 
\[
    U_{a_i} \cap U_B \subset U_{a_i} \cap V_{a_i} = \emptyset
.\] 
und daraus folgt, dass
\[
U_A \cap U_B = \emptyset
.\] 
\end{proof}

\begin{theorem}
    Sei $X$ kompakt und Hausdorffsch, $q: X \to  Z$ surjektiv, wobei  $Z$ die Quotiententopologie trage. Dann sind äquivalent: 
    \begin{enumerate}[1)]
        \item $Z$ ist Hausdorffsch
        \item  $q$ ist abgeschlossen
    \end{enumerate}
    \label{thm:quotient-space-of-compact-hausdorff-space}
\end{theorem}
\begin{proof}
    Die Richtung '$1) \implies 2)$' ist genau Korollar 5.7  \\
    '$2)\implies_1)$': Jedes $z\in Z$ hat ein Urbild $x\in X$ unter $q$. Es ist  $\left \{x\right\} \subset X$ abgeschlossen, da $X$ hausdorffsch. Wegen  $q$ abgeschlossen folgt nun, dass auch
    \[
        \left \{z\right\}  = q(\left \{x\right\} )
    .\] 
    abgeschlossen ist. Eine Teilmenge $W\subset X$ heißt \vocab{saturiert}, falls $W = q^{-1}(q(W))$ (insbesondere sind alle Urbilder saturiert, und $\iff  \forall x\in X \setminus W : g(x) \in Z \setminus g(W)$). \\
\begin{remark}
    Sei $U\subset X$ offen und saturiert, dann ist $q(U)$ offen. Hierzu schreibe
    \[
        U = q^{-1}(q(U)) \implies q(U) \text{ offen}
    .\] 
\end{remark}
Seien $y\neq z\in Z$. Dann sind $\left \{y\right\} ,\left \{z\right\} $ abgeschlossen und disjunkt. Dann sind auch
\[
    A = q^{-1}(y) \qquad B = q^{-1}(z)
.\] 
abgeschlossen und disjunkt (in  $X$). Nach Annahme ist  $X$ kompakt und Hausdorff, also normal nach Satz \ref{thm:compact-hausdorff-space-is-normal}. Also existieren  $U_1,U_2\subset X$ offen mit $A\subset U_1,B\subset U_2$ und $U_1 \cap U_2 = \emptyset$. Setze
\[
    V_1 := X \setminus q^{-1}(q(X\setminus U_1)) \qquad V_2 := X \setminus q^{-1}(q(X\setminus U_2))
.\] 
\begin{claim}
    Es sind $V_1,V_2$ offen, disjunkt und saturiert und $A\subset V_1$ sowie $B\subset V_2$.
\end{claim}
\begin{proof}
    TODO
\end{proof}
Es folgt, dass $q(V_1),q(V_2)$ offen in $Z$ sind. Weiter ist  $y\in q(A)\subset q(V_1)$ und $z\in q(B) \subset q(V_2)$. Da $V_1,V_2$ disjunkt und saturiert, sind auch $q(V_1),q(V_2)$ disjunkt und wir sind fertig.
\end{proof}
