\lecture[]{Di 11 Mai 2021 12:16}{Zusammenhang}

\section{Zusammenhang, Wegzusammenhang}
\begin{definition}[Zusammenhang]\label{def:zusammenhang}
    Ein topologischer Raum heißt \vocab{Topologischer Raum!zusammenhängend}, wenn er sich \underline{nicht} in zwei nichtleere, disjunkte, offene Teilmengen zerlegen lässt. 
\end{definition}
\begin{dlemma}[Offen-abgeschlossene-Mengen]\label{lm:raum-ist-zusammenhängend-gdw-offen-abgeschlossene-mengen-sind-trivial}
    Ein Raum ist zusammenhängend, wenn die leere Menge und der gesamte Raum die einzigen Teilmengen von $X$ sind, die offen und abgeschlossen sind, d.h.
     \[
    \not \exists  A\subset X, A\neq \emptyset,X \colon \quad A \text{ offen und abgeschlossen}
    .\] 
\end{dlemma}
\begin{proof*}
    Klar.
\end{proof*}
\begin{remark}
    $X$ ist nicht zusammenhängend, genau dann, wenn  $X \cong X_1 \coprod X_2$ eine disjunkte Vereinigung von 2 Räumen $X_1,X_2\neq \emptyset$ ist.
\end{remark}
\begin{example}
    \begin{enumerate}[1)]
        \item $\R\setminus \left \{0\right\}  = (-\infty,0) \cup (0,\infty)$ und $(-\infty,0),(0,\infty)$ sind offen, disjunkt und nicht leer, also ist $\R\setminus \left \{0\right\} $ \underline{nicht} zusammenhängend. 
        \item Betrachte $\Q\subset \R$ mit der Unterraumtopologie. Dann ist
            \[
                \Q = (\Q \cap (-\infty,\sqrt{2})) \cup (\Q \cap (\sqrt{2},\infty))  
            .\] 
            eine Zerlegung in offene, disjunkte, nichtleere Mengen, also ist auch $\Q$ nicht zusammenhängend.
    \end{enumerate}
\end{example}
\begin{remark*}
    Es ist meistens einfacher, zu zeigen, dass ein Raum nicht zusammenhängend ist, die Gegenrichtung erweist sich als schwerer. Deswegen folgender
\end{remark*}
\begin{theorem}[Einheitsintervall]\label{thm:einheitsintervall-ist-zusammenhängend}
    Das Intervall $[0,1]$ ist zusammenhängend.
\end{theorem}
\begin{proof}
    Nimm gegenteilig an, dass $[0,1]$ nicht zusammenhängend ist, schreibe also  $[0,1] = A \cup B$ mit $A,B \neq \emptyset$, offen und disjunkt. OBdA sei $0\in A$. Wegen $B\neq \emptyset$ gibt es $t:= \inf B$. Da  $t$ abgeschlossen (weil  $A$ offen!), ist  $t\in B$, also folgt $[0,t) \subset A$. Aber jede Umgebung von $t\in B$ schneidet $[0,t)$, also  $A$, \contra, weil  $A\cap B = \emptyset$.
\end{proof}
\begin{ddefinition}[Weg]\label{def:weg}
    Sei $X$ ein topologischer Raum und $x,y\in X$. Ein  \vocab{Weg} von $x$ nach $y$ ist eine stetige Funktion $w: [0,1] \to  X$, sodass $w(0) =x$ und  $w(1) = y$.
\end{ddefinition}

\begin{definition}[Wegzusammenhang]\label{def:wegzusammenhang}
    Ein topologischer Raum $X$ heißt  \vocab{wegzusammenhängend}, falls für je zwei Punkte $x,y\in X$ ein \vocab{Weg} von $x$ nach  $y$ existiert.
\end{definition}
\begin{example}
    \begin{enumerate}[1)]
        \item     Die Mengen $(a,b), [a,b), (a,b]$ und $\R$ sind alle wegzusammenhängend. Definiere hierzu
        \begin{equation*}
        w: \left| \begin{array}{c c l} 
            [0,1] & \longrightarrow & \R \\
            t & \longmapsto &  ty + (1-t)x
        \end{array} \right.
    \end{equation*}
   Als Verknüpfung stetiger Funktionen ist $t$ stetig, und wir sehen leicht, dass  $0 \mapsto x, 1 \mapsto y$. 
   \item $\R^n, n\geq 0$ ist wegzusammenhängend. Dazu betrachte vorherige Abbildung auf den einzelnen Komponenten
   \item $\R^n \setminus \left \{0\right\} , n\geq 2$ ist wegzusammenhängend. Seien hierzu $x,y\in \R^n \setminus \left \{0\right\}$.
       \begin{description}
           \item[Fall 1:] Die Strecke von $x$ nach  $y$ liegt in  $\R^n \setminus \left \{0\right\}$. Dann betrachten wir wieder die Abbildung aus 1) und sind fertig.
           \item[Fall 2:] Die Strecke trifft die $0$. Wähle dann einen dritten Punkt $z$, der nicht auf der Geraden durch $x,y$ liegt. Dann gibt es einen Weg von $x$ nach  $z$ und einen von  $z$ nach  $x$, und die Vereinigung der beiden Wege ist dann ein Weg von  $x$ nach  $y$.
       \end{description}
    \end{enumerate}
\end{example}
\begin{lemma}\label{lm:wegzusammenhang-impliziert-zusammenhang}
    Ist $X$ wegzusammenhängend, so ist  $X$ zusammenhängend.
\end{lemma}
\begin{warning}
    Die Umkehrung von \autoref{lm:wegzusammenhang-impliziert-zusammenhang} gilt im Allgemenien nicht. Siehe hierzu Übungsblatt 5, Aufgabe 1.
\end{warning}
\begin{proof}[Beweis von \autoref{lm:wegzusammenhang-impliziert-zusammenhang}]
    Sei $X$ wegzusammenhängend, und nimm gegenteilig an, dass  $X = U_1 \sqcup U_2$ mit $U_i \subset X$ offen und disjunkt. Sei $x_1 \in U_1, x_2\in U_2$. Dann gibt es einen Weg $w$ von  $x_1$ nach $x_2$, und  wir erhalten
    \[
        w^{-1}(U_1) \cup w^{-1}(U_2) = w^{-1}(U_1\cup U_2) = [0,1]
    .\] 
    Allerdings sind $w^{-1}(U_i)$ offen ($w$ ist stetig), disjunkt ($U_1,U_2$ sind disjunkt) und nicht leer ($0\in w^{-1}(U_1)$, $1\in w^{-1}(U_2)$), also ist $[0,1]$ nicht zusammenhängend. \contra mit \autoref{thm:einheitsintervall-ist-zusammenhängend}.
\end{proof}
