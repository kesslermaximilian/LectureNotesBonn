\lecture[Beweis des Lemmas von Urysohn. Urysohn für beliebige Intervalle. Satz von Tietze. 'Quetschen' von stetigen Funktionen auf normalen Räumen. Beweis des Satz von Tietze. Metrisierungssatz von Urysohn.]{Di 18 Mai 2021 12:20}{Urysohn, Tietze}
Wir erinnern uns daran, dass wir gerade dabei waren, \autoref{thm:urysohn} zu beweisen.
\begin{lemma}\label{trennung-von-mengen-in-normalem-raum-für-urysohn-lemma}
    Sei $X$ ein normaler Raum,  $A\subset X$ abgeschlossen und $U\subset X$ offen mit  $A\subset U$. Dann existiert $V\subset X$ offen mit
    \[
    A\subset V\subset \overline{V}\subset U
    .\] 
\end{lemma}
\begin{figure}[ht]
    \centering
    \incfig{trennung-von-abgeschlossenen-mengen-durch-offene-in-normalem-raum}
    \caption{Skizze zu \autoref{trennung-von-mengen-in-normalem-raum-für-urysohn-lemma}}
    \label{fig:trennung-von-abgeschlossenen-mengen-durch-offene-in-normalem-raum}
\end{figure}
\begin{proof}
    Wegen $U$ offen ist  $X\setminus U$ abgeschlossen. Wegen $X$ normal gibt es  $V,V'$ offen mit  $A\subset V$ und $(X\setminus U)\subset V'$ mit $V\cap V'=\emptyset$. Nun ist
    \[
    A\subset V\subset X\setminus V' \subset U
    .\] 
    nach Definition des Abschlusses ist nun $A\subset V\subset \overline{V} \subset X\setminus V' \subset U$.
\end{proof}

\begin{proof}[Beweis von \autoref{thm:urysohn} (\nameref{thm:urysohn})]
    \begin{goal}
        $\forall r\in \Q\cap [0,1]$ konstruiere $V_r \subset X$ offen, sodass
         \begin{enumerate}[1.]
            \item $A\subset V_0$
            \item $B\subset X\setminus V_1$
            \item  $r<r' \implies \overline{V_r}\subset V_{r'}$
        \end{enumerate}
    \end{goal}
Dies genügt, denn dann wissen wir mit \autoref{lm:stetige-abbildung-durch-familie-von-rationalen-offenen-mengen}, dass 
\begin{IEEEeqnarray*}{rCl}
    \exists f: X &\to & [0,1] \text{ stetig} \\
        f(x) & = & 0 \quad \forall x\in V_0 \supset A
        \\ f(x) & = & 1 \quad \forall x \in  X \setminus V_1\supset B
\end{IEEEeqnarray*}
Wähle hierzu eine Abzählung $p_1,p_2,\ldots$ von $\Q\cap [0,1]$, sodass $p_1 = 1$ und $p_2 = 0$. Definiere nun $\left \{V_r\right\} $ rekursiv, wobei wir auch induktiv die Invariante erhalten wollen, dass $r<r' \implies \overline{V_r} \subset V_{r'}$.
\begin{itemize}
    \item $p_1 = 1$. Setze $V_1 \coloneqq X\setminus B$ (offen, weil $B$ abgeschlossen ist)
    \item  $p_2 = 0$. Nach \autoref{trennung-von-mengen-in-normalem-raum-für-urysohn-lemma} mit $A = A$ und  $U = X\setminus B$ finden wir $V_0$ offen mit 
        \[
        A\subset V_0 \subset \overline{V}_0 \subset X\setminus B =: V_1
        .\] 
    \item Sei $n\geq 3$, dann sind also $V_{p_1},V_{p_2},\ldots,V_{p_{n-1}}$ schon definiert. Es ist $\left \{p_1,p_2,\ldots,p_n\right\} $ wohlgeordnet, weil es sich um eine endliche Menge handelt, also gibt es unter ihnen einen direkten Vorgänger $p_i$ von  $p_n$, und einen direkten Nachfolger  $p_j$ von  $p_n$.
         \begin{recap}
            Es könnte z.B.  $n=5$ sein mit \\
            \begin{tabular}{c | c | c | c | c}
                $p_1$ & $p_2$ & $p_3$ & $p_4$ & $p_5$ \\
                1 & 0 & $\frac{1}{2}$ & $\frac{8}{9}$ & $\frac{3}{5}$
            \end{tabular} 
            Dann ist die Menge als $\left \{0,\frac{1}{2},\frac{3}{5},\frac{8}{9},1\right\} $ geordnet, und wir sehen $p_i = = \frac{1}{2} < \frac{3}{5} < \frac{8}{9} = p_j$.
        \end{recap}
        Verwende nun \autoref{trennung-von-mengen-in-normalem-raum-für-urysohn-lemma} mit $A = \overline{V_{p_i}}$ und $U = V_{p_j}$, (hier ist wichtig, dass wegen $p_i < p_j$ beretis  $\overline{V_{p_i}}\subset V_{p_j}$ gilt, sonst können wir das Lemma nicht anwenden.) \\
        Also finden wir $V$ mit  $\overline{V_{p_i}} \subset V \subset \overline{V} \subset V_{p_j}$. Man prüft leicht, dass wir so auch die Invariante der Induktion erhalten haben.
\end{itemize}
Also haben wir wie gewünscht die $V_i$ gefunden, und somit unsere Funktion.
\end{proof}

\begin{corollary}[Urysohn mit beliebigem Intervall]\label{cor:urysohn-mit-beliebigem-intervall}
    Sei $X$ ein normaler Raum und seien  $A,B\subset X$ disjunkt und abgeschlossen, sowie $a\leq b\in \R$ beliebig. Dann 
    \begin{IEEEeqnarray*}{rCl}
        \exists f: X & \to  & [a,b] \\
        f(A) & = &\left \{a\right\}  \\
        f(B) & = & \left \{b\right\} 
    \end{IEEEeqnarray*}
\end{corollary}

\begin{proof}
    Zunächst verwenden wir Urysohn, um eine stetige Funktion $g: X \to  [0,1]$ zu erhalten mit $f(A) = \left \{0\right\} $ und $f(B) = \left \{1\right\}$, dann verknüpfen wir mit der stetigen Abbildung
        \begin{equation*}
        h: \left| \begin{array}{c c l} 
            [0,1] & \longrightarrow & [a,b] \\
            t & \longmapsto &  (1-t)a + tb
        \end{array} \right.
    \end{equation*}
    und wir erhalten sofort die gewünschten Eigenschaften, indem wir $f = h \circ  g$ setzen.
\end{proof}

\section{Der Erweiterungssatz von Tietze}
Wir sehen jetzt das \nameref{thm:urysohn} in Action:
\begin{theorem}[Erweiterungssatz von Tietze]\label{thm:tietze}
    Sei $X$ ein normaler Raum und  $A\subset X$ abgeschlossen. Jede stetige Funktion $f: A \to  [-1,1]$ lässt sich fortsetzen zu einer stetigen Funktion $\overline{f}: X \to  [-1,1]$, d.h. $\overline{f}\mid _{A} \equiv f$.
\end{theorem}

\begin{remark}
    Das Urysohn'sche Lemma ist ein Spezialfall des \nameref{thm:tietze}: \\
    Sei $X$ normal und  $B,C\subset X$ abgeschlossen, disjunkt. Dann betrachte die Funktion
        \begin{equation*}
        f: \left| \begin{array}{c c l} 
            B\cup C & \longrightarrow & [-1,1] \\
        B & \longmapsto &  -1 \\
        C & \longmapsto & 1
        \end{array} \right.
    \end{equation*}
    \begin{question}
        Gibt es eine Fortsetzung $\overline{f}: X \to  [-1,1]$?
    \end{question}
    Für jede solche Fortsetzung muss auch  $\overline{f}\mid _{B} = f$, also $\overline{f}(B) = -1$ und $\overline{f}(C) =1$ gelten, also genau das, was wir von Urysohn fordern. \\
    Allerdings sagt uns der \nameref{thm:tietze} genau, dass wir solche eine Fortsetzung finden.
\end{remark}

\begin{proof}[Beweis von \autoref{thm:tietze} (\nameref{thm:tietze})]
    \label{proof:tietze1}
    \begin{strategy}
        Wir konstruieren eine Folge stetiger Funktionen \\
        $\left \{s_n : X \to  [-1,1]\right\}_{n\geq 1}$, sodass
        \begin{enumerate}[(i)]
            \item $\left \{s_n\right\} $ konvergiert \vocab[Konvergenz!gleichmäßige]{gleichmäßig} gegen eine Funktion $s: X \to  [-1,1]$, d.h. für jedes $ε>0$ existiert  $N\in \N$, sodass 
                \[
                \emphasize{\forall} x \in X,n\geq N \colon\qquad    d(s_n(x), s(x))<ε
                .\] 
                Weil $\left \{s_n\right\} $ gleichmäßig konvergiert, ist $s$ stetig (Übungsblatt 5, Aufgabe 3 (iv)).
            \item  $s\mid _{A} = f$
        \end{enumerate}
    \end{strategy}
    Dazu benötigen wir erstmal einige Lemmata, die wir im folgenden erarbeiten.
\end{proof}

\begin{lemma}\label{lm:kompression-von-funktionen-auf-abgeschlossenen-mengen-in-normalem-raum}
    Sei $X$ normal und  $A\subset X$ abgeschlossen. Sei $\alpha : A \to  [-r,r]$ für $r\in \R_{\geq 0}$ stetig. Dann existiert $g: X \to  \left[ -\frac{1}{3}r, \frac{1}{3}r \right]$ stetig mit $\abs{α(a) - g(a)} \leq  \frac{2}{3}r $ für $a\in A$.
\end{lemma}
\begin{proof}
    Setze $B \coloneqq α^{-1}\left(\left[ -r,-\frac{1}{3}r \right]\right)$ und $C\coloneqq α^{-1}\left(\left[ \frac{1}{3}r,r \right]\right)$. Wegen $α$ stetig sind  $B,C$ abgeschlossen, und sie sind auch disjunkt, weil die Intervalle disjunkt sind. Nach \nameref{cor:urysohn-mit-beliebigem-intervall} finden wir also eine stetige Funktion
    \begin{IEEEeqnarray*}{rCl}
        g: X & \to  & \left[ -\frac{1}{3}r, \frac{1}{3}r \right] \\
        g(B) & = & \left \{-\frac{1}{3}r\right\} \\ 
        g(C) & = & \left \{\frac{1}{3}r\right\} 
    \end{IEEEeqnarray*}
    \begin{tikzpicture}
        \draw (0,-0.2) -- (8,-0.2) node[anchor = west] {$X$};
        \foreach \x in {0,1,2,3} {
            \draw[red,thick,dotted] (0,\x) -- (8,\x);
        }
        \draw (0,0) node[anchor=east] {$-r$};
        \draw (0,1) node[anchor=east] {$-\frac{1}{3}r$};
        \draw (0,2) node[anchor=east] {$\frac{1}{3}r$};
        \draw (0,3) node[anchor=east] {$r$};
        \draw[green!60!black, line width = 2pt] (4,-0.2) -- (5,-0.2) node[midway,anchor=north] {$B$};
        \draw[orange!90!black, line width = 2pt] (1,-0.2) -- (2,-0.2) (2.9,-0.2) -- node[midway, anchor = north] {$C$} (3.5,-0.2) (6,-0.2) -- (7,-0.2);
        \draw[line width = 1.5pt,blue] (1,-0.1) -- (1,-0.3);
        \draw[line width = 1.5pt,blue] (7,-0.1) -- (7,-0.3);
        %Graph of alpha
        \draw[blue!70!white,thick] (1,3) to[out = 0, in = 100] (2,2) to[out = -80, in = 180] (2.6,1.6) to[out = 0, in = 210] (2.9,2) to [out = 30,in=180] (3.1,2.4) to[out = 0,in = 120] (3.5,2) to[,out = -60,in = 100] (4,1) to[out = -80, in = 180] (4.5,0.2) to[out=0, in =240] (5,1) to[out = 60, in = 250] (6,2) to[out =70, in = 180] (6.5,3) to[out = 0, in = 130] node[near end, anchor = south west]{$α$} (7,2);
        %Graph of corresponding g:
        \draw[red, thick] (0.2,0.5) to[in = 230] node[near start, anchor = north west] {$g$} (1,2);
        \draw[red, thick] (2,2) to[out = -70, in = 180] (2.3,1.5) to[out = 0, in = 210] (2.9,2);
        \draw[red, thick] (3.5,2) to[out = -30, in = 130] (4,1);
        \draw[red, thick] (5,1) parabola (6,2);
        \draw[red, thick] (7,2) to[out = -40, in = 180] (7.5,1.3) to[out = 0, in = 250] (8,2.8);
        \draw[red, line width = 1.5pt] (4,1) -- (5,1);
        \draw[red, line width = 1.5pt] (1,2) -- (2,2) (2.9,2) -- (3.5,2) (6,2) -- (7,2);
        \draw[decorate, decoration={brace,mirror,amplitude=4pt}, yshift = -20pt, blue] (1,0) -- (7,0) node[midway, anchor = north, yshift = -5pt] {$A$};
    \end{tikzpicture}
    \begin{claim}
        $g$ erfüllt die Bedingungen unseres Lemmas, d.h.  
        \[
        \abs{α(a) - g(a) } \leq \frac{2}{3}r \qquad \forall a\in A
        .\] 
    \end{claim}
    \begin{subproof}
        \begin{itemize}
            \item Sei $a\in B$, Dann ist $α(a) \in \left[ -r, -\frac{1}{3}r \right]$ und $g(a) = -\frac{1}{3}r$, also gilt die Ungleichung.
            \item Sei $a\in C$. Dann ist $α(a) \in \left[ \frac{1}{3}r,r \right]$ und $g(a) = \frac{1}{3}r$, also, also gilt die Ungleichung.
            \item Sei $a\in A \setminus (B\cup C)$. Dann ist $α(a),g(a) \in \left[ -\frac{1}{3}r,\frac{1}{3}r \right] $, und damit ist der Abstand auch höchstens $\frac{2}{3}r$.
        \end{itemize}
    \end{subproof}
\end{proof}
\begin{remark*}
    In der Vorlesung kam die Frage auf, ob wir manche der gerade bewiesenen Resultate auch auf die Analysis übertragen können, indem wir z.B. den Fixpunktsatz von Banach anwenden.
\end{remark*}
\todo{darüber nachdenken}

\begin{proof}[Fortsetzung des Beweises des \nameref{thm:tietze}]
    Definiere die Folgen $s_n$ induktiv. Dazu \\
    \underline{Schritt 1}: Verwende \autoref{lm:kompression-von-funktionen-auf-abgeschlossenen-mengen-in-normalem-raum} mit $α = f$ und  $r = 1$, also erhalten wir
    \[
    g_1 : X \to  \left[ -\frac{1}{3}r, \frac{1}{3}r \right] 
    .\] 
    settig mit $\abs{f(a) - g(a)} \leq \frac{2}{3}$ für jedes $a\in A$. Setze $s_1 = g_1$. \\
    \underline{Induktionsschritt} Angenommen, wir haben schon stetige Funktion $g_1,\ldots,g_n$ auf $X$ mit
    \[
        g_i : X \to  \left[ -\left( \frac{2}{3} \right) ^{i-1}\cdot \frac{1}{3}, \left( \frac{2}{3} \right) ^{i-1}\cdot \frac{1}{3} \right] 
    .\]
    und 
    \[
        \abs{f(a) - \sum_{i=1}^n g_i(a)} \leq  \left( \frac{2}{3} \right) ^n \qquad \forall a\in A 
    .\] 
    Verwende nun wieder \autoref{lm:kompression-von-funktionen-auf-abgeschlossenen-mengen-in-normalem-raum} mit $α = f- \sum_{i=1}^n g_i\mid _{A}$ und $r = \left( \frac{2}{3} \right) ^n$. \\
    Es gibt also
    \[
        g_{n+1} \colon X \to  \left[ -\frac{1}{3}\cdot \left( \frac{2}{3} \right) ^{n}, \frac{1}{3}\cdot \left( \frac{2}{3} \right) ^n \right] 
    .\] 
    Wir erhalten nun
    \begin{IEEEeqnarray*}{rCl}
        \abs{f(a) - \sum_{i=1}^n g_i(a) - g_{n+1}(a)} & = &  \abs{f(a) - \sum_{i=1}^{n+1} g_i(a)}   \\
                                                      & \leq  & \frac{2}{3}\cdot \left( \frac{2}{3} \right) ^n = \left( \frac{2}{3} \right) ^{n+1}
    \end{IEEEeqnarray*}
    Nun definiere
    \[
    s_{n+1} \coloneqq \sum_{i=1}^{n+1} g_i : X \to \R
    .\] 
    \begin{claim}
        $s_n$ hat Bild in $[-1,1]$ für jedes  $n\in \N$.
    \end{claim}
    \begin{subproof}
        \begin{IEEEeqnarray*}{rCl}
            \abs{s_n(x)} & =&   \abs{\sum_{i=1}^n g_i(x)} \leq  \sum_{i=1}^n  \abs{g_i(x)} \\
                         & \leq& \sum_{i=1}^n \left( \frac{2}{3} \right) ^{i-1}\cdot \frac{1}{3}  \\
                         & \leq  & \sum_{i=1}^{\infty} \left( \frac{2}{3} \right) ^{i-1}\cdot \frac{1}{3}  \\
                         & = & \frac{\frac{1}{3}}{1-\frac{2}{3}} = 1
        \end{IEEEeqnarray*}
        Die andere Schranke zeigt man völlig analog.
    \end{subproof}
    \begin{claim}
        Die Folge $\left \{s_n\right\} _{n\in \N}$ konvergiert gleichmäßig gegen $s$.
    \end{claim}
    \begin{subproof}
        Für $k>n$ ist 
         \begin{IEEEeqnarray*}{rCl}
             \abs{s_k(x) - s_n(x)} & = & \abs{ \sum_{i=n+1}^k g_i(x) } \\
                                   & \leq  & \sum_{i=1}^{n+1} \frac{1}{3}\cdot \left( \frac{2}{3} \right) ^{i-1} \\
                                   & = &  \sum_{i=n+1}^{\infty} \frac{1}{3}\cdot \left( \frac{2}{3} \right) ^{i-1} \\
                                   & = & \frac{1}{3}\cdot \frac{\left( \frac{2}{3} \right) ^n}{1-\frac{2}{3}} \\
                                   & = &\left( \frac{2}{3} \right) ^n
        \end{IEEEeqnarray*}
        \begin{enumerate}[(i)]
            \item Für jedes $x\in X$ ist $\left \{s_n(x)\right\}_{n\geq 1} $ eine Cauchy-Folge, also konvergiert sie zu einem Punkt $s(x) \in [-1,1]$.
            \item Intuitiv reicht es für gleichmäßige Stetigkeit schon zu sehen, dass in obiger Abschätzung kein $x$ vorkommt. Genauer: \\
                Sei  $ε>0$, so  $\exists  N \in \N$ mit $\forall n > N \implies \left( \frac{2}{3} \right) ^n < ε$. Dann ist
                \begin{IEEEeqnarray*}{rCl}
                    \abs{s(x) - s_n(x)} & = & \abs{\lim_{k \to \infty} s_k(x) - s_n(x)} \\
                                        & = & \lim_{k \to \infty} \abs{s_k(x) - s_n(x)} \\
                                        & \leq  & \left( \frac{2}{3} \right) ^n \\
                                        & < &ε
                \end{IEEEeqnarray*}
                also ist die Konvergenz gleichmäßig, und $s$ ist stetig.
        \end{enumerate}
    \end{subproof}
    Setze nun $\overline{f} \coloneqq s$, dann überprüfen wir
    \begin{claim}
        $\overline{f}$ ist eine Fortsetzung von $f$, d.h.  $\overline{f}\mid _{A} = f$.
    \end{claim}
    \begin{subproof}
        Es ist 
        \begin{IEEEeqnarray*}{rCl}
            \abs{f(a) - s_n(a)} = \abs{f(a) - \sum_{i=1}^n g_i(a)} \leq  \left( \frac{2}{3} \right) ^n  
        \end{IEEEeqnarray*}
        also 
        \begin{IEEEeqnarray*}{rCl}
            \lim_{n \to \infty} \abs{f(a) - s_n(a)} \leq  \lim_{n \to \infty}  \left( \frac{2}{3} \right) ^n = 0
        \end{IEEEeqnarray*}
        Also ist 
        \[
            s(a) \coloneqq  \lim_{n \to \infty} s_n(a) = f(a)
        .\] 
        für jedes $a\in A$.
    \end{subproof}
    Wir haben also eine Fortsetzung gefunden, und sind damit fertig.
\end{proof}

\begin{corollary}[Version des Satzes von Tietze]\label{cor:tietze-2}
    Sei $X$ ein normaler Raum und  $A\subset X$ abgeschlossen. Dann gilt folgendes:
    \begin{enumerate}[1.]
        \item Jede stetige Funktion $f: A \to  [a,b]$ lässt sich zu einer Funktion $\overline{f} : X \to  [a,b]$ fortsetzen.
        \item Jede stetige Funktion $f: A \to  \R$ lässt sich fortsetzen zu einer Funktion $\overline{f}\colon X \to \R$.
    \end{enumerate}
\end{corollary}
\begin{proof}
    Übung.
\end{proof}


\section{Der Metrisierungssatz von Urysohn}
\begin{theorem}[Metrisierungssatz von Urysohn]\label{thm:metrisierungsssatz-von-urysohn}
    Jeder normale Raum mit abzählbarer Basis der Topologie ist metrisierbar.
\end{theorem}
\begin{proof}
    \begin{strategy}
        \underline{Schritt 1}: Betrachte $\prod_{\N} [0,1]$ in der Produkttopologie und zeige, dass der Raum emtrisierbar ist. Dieser Raum heißt \vocab{Hilbert-Würfel}.  \\
        \underline{Schritt 2}: Sei $X$ normal mit abzählbarer Basis, wir zeigen, dass wir eine Einbettung  $F : X \to  \prod_{\N} [0,1]$ finden.
        Dann sind wir fertig, da 
        \[
            X \cong F(X) \subset \prod_{\N}[0,1]
        .\] 
        ein Unterraum eines metrischen Raumes ist.
    \end{strategy}
\end{proof}
