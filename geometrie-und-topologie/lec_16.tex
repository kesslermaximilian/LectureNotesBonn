%! TEX root = ./master.tex
\lecture[Bijektion zwischen Faser und Fundamentalgruppe für einfach zusammenhängende Überlagerungen. Fundamentalgruppe von $S^1$. Einfacher Zusammenhang von  $\R^n$, $ninn\N$ und $S^n$, $n\geq 2$. Stereographische Projektion $S^n \setminus \left\{\star\right\}\cong \R^n$. Lokal Wegzusammenhängende Räume und ihre Wegzusammenhangskomponenten. Überlagerungen über lokal wegzusammenhängenden Räumen. Allgemeiner Liftungssatz.]{Do 17 Jun 2021 10:15}{Die Fundamentalgruppe von $S^1$}

\begin{notation*}
    Sei $p\colon  E \to  X$ eine Überlagerung, $w\colon  I \to  X$ ein Weg mit $w(0) = x$ und  $e\in p^{-1} (x)$. Dann notieren wir mit $L(w,e)$ die Hebung von  $w$ mit Anfangspunkt  $e$, die nach dem \nameref{thm:weghebungssatz} eindeutig existiert.
\end{notation*}

\begin{theorem}\label{thm:fundamentalgruppe-durch-überlagerung-mit-einfach-zusammenhängendem-raum}
    Sei $p\colon  E \to  X$ eine Überlagerung, wobei $E$ einfach zusammenhängend sei. Sei $x\in X$ und $e\in p^{-1} (x)$. Dann ist die Abbildung
        \begin{equation*}
        \varphi : \left| \begin{array}{c c l} 
            \pi_1(X,x) & \longrightarrow & p^{-1} (x) \\
            \left[w\right] & \longmapsto &  L(w,e)(1)
        \end{array} \right.
    \end{equation*}
    wohldefiniert und bijektiv.
\end{theorem}

\begin{proof}
    \begin{description}
        \item[Wohldefiniertheit:] Angenommen, $w \simeq w'$ relativ Anfangs- und Endpunkt. Nach dem  \autoref{thm:homotopieliftungssatz} ist dann auch $L(w,e) \simeq L(w',e)$ homotop relativ Anfangs- und Endpunkt. Insbesondere haben sie denselben Endpunkt (dieser bleibt während der Homotopie ja konstant), und somit  $L(w,e)(1) = L(w',e)(1)$. 
        \item[Injektivität:] Angenommen, $[w], [w']\in \pi_1(X,x)$ werden auf den gleichen Endpunkt $L(w,e)(1) = L(w',e)(1)$ abgebildet. 

            Da $E$ einfach zusammenhängend, sind nun  $L(w,e)$ und  $L(w',e)$ homotop (sie haben den gleichen Anfangs- und Endpunkt) relativ Endpunkten. Sei $H$ eine solche Homotopie. Dann ist  $p \circ  H$ eine Homotopie von $w$ nach  $w'$ relativ Anfangs- und Endpunkt, also  $[w] = [w']$ und  $\varphi $ ist wie gewünscht bijektiv.
        \item[Surjektivität:] Sei $e' \in p^{-1} (x)$. Da $E$ als einfach zusammenhängender Raum insbesondere wegzusammenhängend ist, gibt es einen Weg  $\tilde{w}\colon I \to  E$ von $e$ nach  $e'$. Dann ist $w \coloneqq  p \circ  \tilde{w}$ eine Schleife an $x$, weil  $e$, $e' \in p^{-1} (x)$, also ist $\varphi ([w]) = L(w,e)(1) = \tilde{w}(1)$ = e'. Da $e'\in p^{-1} (x)$ beliebig war, ist $\varphi $ surjektiv.
    \end{description}
\end{proof}

\begin{oral}
    Mit dieser Bijektion haben wir ein erstes starkes Werkzeug, mit der wir - mittels geschickter Überlagerungen - schon einmal die Mächtigkeit der Fundamentalgruppe bestimmen können.
\end{oral}

\begin{theorem}\label{thm:fundamentalgruppe-von-s1-kreis}
    Es ist $\pi_1(S^1,1) \cong \Z$.
\end{theorem}

\begin{proof}
    Wir betrachten die Überlagerung $\exp \colon  \R \to  S^1$. Zudem ist $\R$ einfach zusammenhängend. Wir wählen das Urbild $0\in \exp ^{-1}(1)$. Nach \autoref{thm:fundamentalgruppe-durch-überlagerung-mit-einfach-zusammenhängendem-raum} ist nun
        \begin{equation*}
        \varphi : \left| \begin{array}{c c l} 
            \pi_1(S^1,1) & \longrightarrow & \exp ^{-1}(1) = \Z \subset \R \\
            \left[w\right] & \longmapsto &  L(w,0)(1)
        \end{array} \right.
    \end{equation*}
    eine Bijektion.
    \begin{claim}
        $\varphi $ ist ein Gruppenhomomorphismus.
    \end{claim}
    \begin{subproof}
    Seien $[w],[w'] \in \pi_1(S^1,1)$. Es ist zunächst
    \[
        L(w\star w', 0) = L(w,0) \star L(w',L(w,0)(1))
    .\]
    (wir heben zunächst $w$, und müssen verknüpfen mit der Weghebung von  $w'$, die am Endpunkt der Hebung von  $w$, also an  $L(w,0)(1)$, beginnt). Wir interessieren uns für den Endpunkt von $L(w\star w',0)$, denn auf diesen bildet  $\varphi $ den Weg $w \star w'$ ab. Dazu genügt, den Endpunkt von  $L(w', L(w,0)(1))$ zu bestimmen, dies tun wir, indem wir den Weg parametrisieren:
    \[
        L(w',L(w,0)(1))(t) = L(w',0)(t) + L(w,0)(1)
    .\] 
    Wir erhalten hier die Hebung an $0$ mit einer Verschiebung um  $L(w,0)(1)$ - diese ist auch eine Hebung, weil  $\exp (t+n) = \exp (t)$ periodisch ist, und $L(w,0)(1)\in \Z$ ein Vielfaches der Periode 1 ist.

    \begin{remark*}
        An dieser Stelle - nämlich dass $L(w,0)(1)\in \Z$, und wir somit mit $L(w',0)(t) + L(w,0)(1)$ wieder eine Hebung von  $w'$ erhalten - geht maßgeblich in den Beweis ein, dass wir  $1\in S^1$ als Basispunkt für die definierte Abbildung $\varphi $ gewählt haben. $\varphi $ ist zwar auch für andere Basispunkte in $S^1$ eine Bijektion, nicht jedoch ein Gruppenhomomorphismus (das Urbild hat in diesem Fall auch gar keine Gruppenstruktur).
    \end{remark*}
    Alles in allem können wir nun nachrechnen, dass
    \begin{IEEEeqnarray*}{rCl}
        \varphi ([w] \circ [w']) & = & L(w\star w',0)(1) \\
                                 & = & L(w,0)\star L(w',L(w,0)(1))(1)\\
                                 & = & L(w',L(w,0)(1))(1) \\
                                 & = & (t \mapsto L(w',0)(t) + L(w,0)(1))(1) \\
                                 & = & L(w',0)(1) + L(w,0)(1) \\
                                 & = & \varphi ([w']) + \varphi ([w]) \\
                                 & = & \varphi ([w]) + \varphi ([w'])
    \end{IEEEeqnarray*}
    also ist $\varphi $ tatsächlich ein Gruppenhomomorphismus.
    \end{subproof}
    Also ist $\varphi$ bijektiv und ein Gruppenhomomorphismus, also schon ein Isomorphismus von Gruppen.
\end{proof}

\begin{oral}
    Es ist hier ein bisschen Glück bzw. Zufall, dass die Bijektion $\varphi $ sogar ein Gruppenhomomorphismus ist. Im Allgemeinen wir dies nicht so sein, wir werden uns aber im Zuge von \textit{Gruppenwirkungen} dieser Thematik auch im Allgemeineren noch annähern.
\end{oral}


\begin{remark}
    Ein Erzeuger von $\pi_1(S^1,0)$ ist gegeben durch die Abbildung
    \[
        \exp |_{[0,1]} \colon  [0,1] \to  S^1
    .\] 
\end{remark}

\begin{proof}
    Es ist
    \begin{IEEEeqnarray*}{rCl}
        \varphi (\left[ \exp |_{[0,1]} \right] ) & = & L\left( \exp |_{[0,1]},0 \right)(1) \\
                                                 & = & (t \mapsto t)(1) \\
                                                 & = & 1
    \end{IEEEeqnarray*}

    Allgemein kann man so auch zeigen, dass die Abbildung
    \[
        \varphi  (t\mapsto \exp (tk)) = k
    .\] 
\end{proof}

\begin{theorem}\label{thm:r^n-einfach-zusammenhängend}
    $\forall n\in \N$ ist $\R^n$ einfach zusammenhängend, insbesondere $\pi_1(\R^n,0) = 0$.
\end{theorem}

\begin{remark*}
    Vergleiche hierzu auch  \autoref{aufgabe-8.4}, hier zeigen wir das gleiche Resultat, aber mit einem etwas anderen Weg.
\end{remark*}

\begin{theorem}\label{thm:s^n-einfach-zusammenhängend-wenn-n-geq-2}
    Sei $n\in \N$ mit $n\geq 2$. Dann ist $S^n$ einfach zusammenhägend, insbesondere $\pi_1(S^n,1) = 0$. 
\end{theorem}

Als kleine Vorbereitung benötigen wir:

\begin{lemma*}\label{lm :s^n-ohne-punkt-ist-r^n}
Sei $z\in S^n$ beliebig, dann ist $S^n \setminus \left \{z\right\} \cong\R^n$
\end{lemma*}

\begin{proof}
    Wir führen eine \vocab{stereographische Projektion} durch, d.h. wir definieren die Abbildung
        \begin{equation*}
        \varphi : \left| \begin{array}{c c l} 
        S^n \setminus \left \{z\right\}  & \longrightarrow & \left< z \right> ^{\bot} \\
        x & \longmapsto &  z - \frac{1}{\left< x-z,z \right> }(x-z)
        \end{array} \right.
    \end{equation*}
Hierbei ist $\left< z \right> ^{\bot}$ der Unterraum der Dimension $n$ von  $\R^{n+1}$, auf dem $z$ senkrecht steht, und $x$ wir dann abgebildet auf den Schnittpunkt der Geraden durch  $x$ und  $z$ mit diesem Unterraum.

    Ein Alternativer Beweis wäre, zu verwenden, dass
    \[
    S^n \setminus \left \{z\right\} \cong D^n \setminus \left \{0\right\} / \partial D^n \cong  \R^n
    .\] 
    wobei wir im letzten Schritt $x \mapsto  \frac{1-\lVert x \rVert }{\lVert x \rVert^2 }x$ abbilden.
\end{proof}

\begin{figure}[ht]
    \centering
    \incfig{stereographische-projektion}
    \caption{Stereographische Projektion zwischen $S^1\setminus \left \{(0,1)\right\} $ und $\R^1\subset \R^2$}
    \label{fig:stereographische-projektion}
\end{figure}

\begin{remark*}[Wie kommt man auf die Formel der stereographischen Projektion?]
Wir wollen für $z\in S^n$ ein $x\neq z \in S^n$ abbilden auf den Schnittpunkt der Geraden durch $x,z$ und dem zu  $z$ senkrecht stehenden Unterraum $\left< z \right> ^{\bot} \cong \R^n$. Bezeichnen wir das Bild von $x$ unter dieser Abbildung mit  $y$, so ergeben sich folgende beiden Bedingungen
 \begin{itemize}
    \item $\left< x,z \right> =0$, damit $x$ senkrecht zu  $z$ steht, also in der entsprechenden Hyperebene  $\cong \R^n$.
    \item Es sind $x,y,z$ kollinear, d.h. es existiert ein  $λ\in \R$ mit
        \[
            y = z + λ(x-z)
        .\] 
        , indem wir die entsprechende Gerade durch den Fußpunkt $z$ und den Richtungsvektor  $(x-z)$ mit  $λ$ parametrisieren.
\end{itemize}
Einsetzen ineinander ergibt die Bedingung
\begin{IEEEeqnarray*}{rCl}
    0 & = & \left< y,z \right> \\
      & = & \left< z + λ(x-z),z \right> \\
      & = & \left< z,z \right> + λ \left< x-z,z \right>  \\
       & = & 1 + λ\left< x-z,z \right> 
\end{IEEEeqnarray*}
was sich äquivalent umformt zu
\[
λ = -\frac{1}{\left< x-z,z \right> }
.\] 
weswegen wir die obige Form der Abbildung erhalten.
\end{remark*}



\begin{proof}[Beweis von \autoref{thm:s^n-einfach-zusammenhängend-wenn-n-geq-2}]
    \underline{Schritt 1}: Sei  $w\colon I \to  S^n$ eine Schleife an $x$, so dass das Bild von  $w$ nicht gleich  $S^n$ ist.
    \begin{claim}
        Dann ist $w$ homotop relativ Endpunkten zur konstanten Schleife.
    \end{claim}
    \begin{subproof}
        Sei $z\in S^n \setminus \Bild(z)$, solch ein Punkt existiert nach Voraussetzung. Wegen \autoref{lm :s^n-ohne-punkt-ist-r^n}  können wir $w$ als Schleife in  $\R^n$ auffassen, und $w$ ist somit homotop relativ endpunkten zur konstanten Schleife.
    \end{subproof}
    \underline{Schritt 2}: Sei $w$ eine beliebige Schleife an $x$. Wir wollen $x$ in Teile zerlegen, die nicht als Bild die gesamte Kugel haben, um auf Schritt 1 zu reduzieren.

    OBdA sei hierzu  $x\neq (0,0,\ldots,1)$ und auch $x\neq  (0,0,\ldots,-1)$, sonst rotiere die Sphäre. D.h. $x$ ist nicht der 'Nord-' oder 'Südpol' der Kugel. Nun betrachte
     \[
         S^n = \underbrace{S^n \setminus \left \{(0,0,\ldots,1)\right\}}_{\coloneqq  U_1} \cup \underbrace{S^n \setminus \left \{(0,0,\ldots,-1\right\}}_{\coloneqq  U_2} 
    .\] 
    Zudem ist
    \[
    V \coloneqq  U_1 \cap U_2 = S^n \setminus \left \{(0,0,\ldots,1) , (0,0,\ldots,-1)\right\}  \cong S^{n-1}\times \R
    .\] 
    wegzusammenhängend für $n\geq 2$. Nach dem Lebesguelemma existiert $n\in \N$, sodass
    \[
        \forall i\leq n-1 \exists k\in \left \{1,2\right\} \text{ mit } w\left( \left[ \frac{i}{n},\frac{i+1}{n} \right]  \right) \subset U_k
    .\] 
    Sei $0 = j_0 \leq  j_1 \leq  \ldots \leq j_l = n$ mit $j_i \in (0,\ldots,n)$, sodass
    \[
        w\left( \frac{j_i}{n} \right) \in V \text{ und } w\left( \left[ \frac{j_i}{n}, \frac{j_{i+1}}{n} \right]  \right) \in  U_k
    .\] 
    d.h. die $j_i$ sind diejenigen Übergangspunkte, die in  $V$ liegen (Liegt einer der $j_i$ nicht in  $V$, so überspringen wir diesen in obiger Auswahl). Nach  \autoref{thm:r^n-einfach-zusammenhängend} ist $w|_{\left[ \frac{j_i}{n}, \frac{j_{i+1}}{n} \right] }$ relativ Endpunkten homotop zu einem Weg in $V$, hierzu wählen wir einen Weg in  $V$, der Anfangs- und Endpunkt verbindet, und beide Wege liegen dann in einem $U_k$, also einem einfach zusammenhängendem Raum.

    Zusammensetzen der so erhaltenen Wege $j_i \to  j_{i+1}$ in $V$ liefert, dass $w$ homotop ist relativ Endpunkten zu einer Schleife in  $V$ ist. Nach Schritt 1 ist somit $w$ homotop relativ Endpunkten zur konstanten Schleife, denn  $V \subsetneq S^n$.
\end{proof}

\begin{oral}
    Der Beweis scheitert für $S^1$, weil wir wir zwar auch  $U_1,U_2\subset S^1$ als einfach zusammenhängende Teile konstruieren können, allerdings der Schnitt $V = U_1 \cap U_2$ nicht mehr zusammenhängend ist. Wir können also in $U_1,U_2$ jeweils Teilstücke des Weges zusammenziehen, diese aber nicht nach $V$ bringen, weswegen uns das nichts nützt.
\end{oral}

\begin{remark*}
    Es gibt tatsächlich Wege $w\colon [0,1] \to  S^n$, die als Bild die gesamte Sphäre haben. Im ersten Moment erscheint das unintuitiv, weil das von den Dimensionen nicht passt, allerdings gibt es sogenannte \textit{Raumfüllende Kurven}. Siehe hierzu auch \href{https://en.wikipedia.org/wiki/Space-filling_curve}{https://en.wikipedia.org/wiki/Space-filling\_curve}.

    Haben wir nun z.B. einen surjektiven Weg $w\colon [0,1] \to  \R^n$, so können wir diesen auch nach $S^n$ surjektiv abbilden, indem wir 
     \[
     I \stackrel{w}{\longrightarrow} \R^n \twoheadrightarrow  D^n \stackrel{p}{\longrightarrow} \faktor{D^n}{\left \{\partial D^n\right\} } \cong  S^n
    .\] 
    verknüpfen. Hierbei genügt eine beliebige stetige surjektive Projektion $\R^n \to  D^n$, z.B.
    \[
    x \mapsto \begin{cases}
        x & x \in D^n \\
        \frac{x}{\lVert x \rVert } & x \not\in D^n 
    \end{cases}
    .\] 
\end{remark*}

\section{Überlagerungen Teil 2}
\begin{definition}\label{def:lokal-wegzusammenhängend}
    Ein Raum $X$ heißt  \vocab{lokal wegzusammenhängend}, falls für jeden Punkt $x\in X$ und jede Umgebung $U$ von  $x$ eine Umgebung  $V\subset U$ existiert, die wegzusammenhängend ist. 
\end{definition}

\begin{example}
\begin{warning}
    Es gibt wegzusammenhängende Räume, die nicht lokal wegzusammenhängend sind.
\end{warning}
Hierzu betrachte wieder die Sinuskurve des Topologen, füge aber einen weiteren Weg ein, der die beiden Wegzusammenhangskomponenten verbindet, also in etwa so:

\begin{minipage}{\textwidth}
    \centering
            \begin{tikzpicture}[domain=0.001:1, xscale = 6]
                \draw[color=blue!30!white,smooth,samples=100,domain=0.001:0.01,line width = 0.1pt] plot[id=gnuplots/topologists-sine-curve-1] function{sin(1/x)};
                \draw[color=blue!30!white,smooth,samples=1000,domain=0.01:0.1, line width = 0.1pt] plot[id=gnuplots/topologists-sine-curve-2] function{sin(1/x)};
                \draw[color=blue!30!white,smooth,samples=100,domain=0.1:1, line width = 0.1pt] plot[id=gnuplots/topologists-sine-curve-3] function{sin(1/x)};
                \draw[color=red,thick] (0,-1) -- (0,1);
                \draw[->] (0,0) -- (1,0);
                \foreach \x in {1,2,3,4,5,6,7,8,9} {
                    \draw (0.1*\x,-0.1) node[anchor=north]{0,\x} -- (0.1*\x, 0.1);
                }
                \draw (-0.01,-1) node[anchor = east] {-1} -- (0.01,-1);
                \draw (-0.01,1) node[anchor = east] {1} -- (0.01,1);
                \draw[color=blue!30!white] (0,0) to[out=180, in = 180, looseness = 0.3] (0,-1.5) -- (1,-1.5) to[out = 0, in = -2, looseness = 0.2] (1,0.84147);
            \end{tikzpicture}
\end{minipage}
Wir wissen bereits, dass die ursprünglich Kurve 2 Wegzusammenhangskomponenten besitzt, da wir diese allerdings verbinden, erhalten wir nun nur noch eine. Betrachten wir aber vom Punkt $(0,0)$ eine Umgebung mit z.B. Radius  $\frac{1}{2}$, so gibt es keine in $U\left((0,0),\frac{1}{2}\right)$ enthaltene wegzusammenhängende Umgebung:

Jede solche Umgebung enthält Punkte beider ursprünglicher Zusammenhangskomponenten der Sinuskurve, alleridngs kann keine solche Umgebung einen Weg zwischen zwei Punkten dieser Zusammenhangskomponenten enthalten - denn jeder solche muss zwingend außerhalb der ursprünglichen Sinuskurve verlaufen (also den neuen Pfad benutzen), dieser liegt aber nicht gänzlich in der Umgebung $U((0,0),\frac{1}{2})$.
\end{example}

\begin{example}
    Das ganze gilt natürlich erst recht nicht in die andere Richtung, es ist z.B. $S^0 = \left \{-1,1\right\} $ lokal wegzusammenhängend, aber sicherlich nicht wegzusammenhängend.
\end{example}


\begin{theorem}\label{thm:wegkomponenten-in-lokal-wegzusammenhängendem-raum-sind-offen}
    Sei $X$ lokal wegzusammenhängend. Dann sind alle Wegekomponenten offen in  $X$.
\end{theorem}

\begin{corollary*}\label{cor:lokal-wegzusammenhängende-räume-sind-koprodukt-ihrer-wegkomponenten}

    Sei $X$ lokal wegzusammenhängend. Dann ist  $X$ die disjunkte Vereinigung seiner Wegekomponenten (topologische gesehen, nicht nur als Mengen).
\end{corollary*}
\begin{proof*}
    Seien $C_i$ die Wegzusammenhangskomponenten von  $X$. Nach universeller Eigenschaft induzieren die Inklusionen $C_i \subset X$ eine Abbildung $f\colon  \coprod C_i \to  X$, die auch offensichtlich bijektiv ist. Wir prüfen die Stetigkeit von $f^{-1}$ auf der kanonischen Subbasis von $\coprod C_i$, d.h. für die Elemente der Form
    \[
        U_j \times \prod_{\substack{ i\in I \\ i\neq j}} C_i \subset \coprod C_i
    .\] 
    Das entsprechende Urbild unter $f^{-1}$, d.h. das Bild unter $f$, ergibt sich als
     \[
    U_j \cup \bigcup_{\substack{i\in I\\i\neq j} } C_j
    .\] 
    Es sind nun aber die $C_i$ nach  \autoref{thm:wegkomponenten-in-lokal-wegzusammenhängendem-raum-sind-offen} offen, und da $U_j \subset C_j \subset X$ jeweils offen sind, ist auch $U_j\subset X$ offen.

    Dieser Beweis funktioniert natürlich für eine beliebige Partition eines beliebigen Raumes in offene Mengen.
\end{proof*}

\begin{proof}[Beweis von \autoref{thm:wegkomponenten-in-lokal-wegzusammenhängendem-raum-sind-offen}]
    Sei $C\subset X$ eine Wegkomponente von $X$, und $x\in C$. Da $X$ lokal wegzusammenhängend ist, hat  $x$ eine wegzusammenhängende Umgebung  $V$. Dann folgt $V\subset C$, weil $C$ die größte wegzusammenhängende Umgebung von  $x$ ist, also enthält  $C$ eine Umgebung von  $x$, d.h.  $C$ ist Umgebung aller inneren Punkte, und somit ist  $C$ offen in  $X$.
\end{proof}

\begin{theorem}\label{thm:überlagerung-über-lokal-wegzusammenhängendem-raum-zerfällt-in-wegzusammenhängende-komponenten-von-e}
    Sei $p\colon E\to X$ eine Überlagerung und $X$ lokal wegzusammenhägend sowie wegzusammenhängend.
    \begin{enumerate}[1)]
        \item Dann ist $E$ lokal wegzusammenhängend. 
        \item Sei $C\subset E$ eine Wegekomponenten. Dann ist auch
            \[
            p|_C \colon  C \to X
            .\] 
            eine Überlagerung.
    \end{enumerate}
\end{theorem}

\begin{oral}
    Wir können die Implikation $1)$ nicht durch Wegzusammenhang von $E$ ersetzen, dazu betrachte als Gegenbeispiel eine triviale Überlagerung $S^1 \times F \to  S^1$ für $F\neq \left \{\star\right\} $.
\end{oral}

\begin{proof}
    \begin{enumerate}[1)]
        \item Sei $e\in E$ und $e\in U$ eine Umgebung. Sei $U' \subset E$ offen, $e\in U'$, so dass
            \[
                p|_{U'}\colon U' \to p(U')
            .\] 
            ein Homöomorphismus ist und $p(U')\subset X$ offen (das existiert, weil $p$ ein lokaler Homöomorphismus ist). Sei  $U'' \coloneqq  U \cap U'$. Dann ist $p(U'')$ eine Umgebung von  $p(e) \in X$. Da $X$ lokal wegzusammenhängend existiert eine Umgebung  $V'$ von  $p(e)$ mit  $V'\subset p(U'')$, und da $p|_{U'}$ ein lokaler Homöomorphismus, ist $V \coloneqq  p^{-1} (V')$ eine wegzusammenhängende Umgebung von $e$.

\begin{figure}[ht]
    \centering
    \incfig{beweis-von-überlagerung-eines-wegzusammenhängenden-raumes}
    \caption{Beweisskizze zu \autoref{thm:überlagerung-über-lokal-wegzusammenhängendem-raum-zerfällt-in-wegzusammenhängende-komponenten-von-e}}
    \label{fig:beweis-von-überlagerung-eines-wegzusammenhängenden-raumes}
\end{figure}


        \item  $E$ ist nach 1) lokal wegzusammenhängend. Dann ist aber bereits  $C\subset E$ offen nach \autoref{thm:wegkomponenten-in-lokal-wegzusammenhängendem-raum-sind-offen}.

            Sei $x\in X$ beliebig, da $p\colon  E \to  X$ eine Überlagerung ist, existiert eine Umgebung $U$ von  $x$, über der  $p$ trivial ist. Da $X$ lokal wegzusammenhängend finden wir  eine Umgebung $V\subset U$ von $x$, sodass  $V$ wegzusammenhängend. Es kommutiert dann auch
            \[
            \begin{tikzcd}
                p^{-1} (V) \ar{rr}{u} \ar[swap]{dr}{} & & V\times p^{-1} (x) \ar{dl}{} \\
            & V
            \end{tikzcd}
            \]
            weil das Ganze schon über $U$ galt. Da  $V$ wegzusammenhängend und  $C$ eine Wegekomponente ist, liegt  $u^{-1}(V\times \left \{e\right\} )$ für jedes $e\in p^{-1} (x)$ entweder ganz in $C$ oder ganz in  $E\setminus C$. Also ist
            \[
                p|_{C}^{-1}(V) = \bigcup_{e\in p^{-1} (x)\cap C} u^{-1}(V\times \left \{e\right\} ) = u^{-1}(V\times p|_{C}^{-1}(x)) 
            .\] 
            und somit ergibt sich das Diagramm:
            \[
            \begin{tikzcd}
                p\mid _C^{-1}(V) \ar{rr}{u}[swap]{\cong} \ar[swap]{dr}{} & & V\times p|_{C}^{-1}(x) \ar{dl}{} \\
            & V
            \end{tikzcd}
            \]
            Es bleibt noch zu zeigen, dass $p|_C$ auch tatsächlich surjektiv ist. Es ist  $C\neq \emptyset$, weil es sich um eine Wegekomponente handelt. Sei $c\in C$ und $x\in X$, wir wollen ein Urbild von $X$ finden. Sei  $w$ ein Weg von $p(c)$ nach  $x$, dieser existiert, weil  $X$ wegzusammenhängend ist. Sei  $\tilde{w}\colon  I \to E$ ein Lift von $w$ mit Anfangspunkt $c$. Dann ist  $\tilde{w}(t) \in C$ für alle $t$, weil  $C$ eine Wegekomponente ist, und somit
             \[
                 p|_C(\tilde{w}(1)) = x
            .\] 
            und somit ist $p|_C$ surjektiv.
    \end{enumerate}
\end{proof}

\begin{theorem}[Allgemeiner Liftungssatz]\label{thm:allgemeiner-liftungssatz}
    Sei $p\colon  E \to X$ eine Überlagerung, $x_0\in X$, $e_0\in p^{-1} (x_0)$. Sei $Y$ wegzusammenhängend und lokal wegzusammenhängend. Sei  $y_0\in Y$ und $f\colon  (Y,y_0) \to  (X,x_0)$ eine punktierte Abbildung.

    Dann sind äquivalent:
    \begin{enumerate}[1)]
        \item Es gibt eine Hebung $\tilde{f}\colon  Y \to  E$ mit $\tilde{f}(y_0) = e_0$
        \item Es ist $f_*(\pi_1(Y,y_0)) \subset p_*(\pi_1(E,e_0))$
    \end{enumerate}
    Ist eine der Bedingungen erfüllt, so ist $\tilde{f}$ eindeutig.
\end{theorem}

\begin{proof}
    '1)$\implies$2)' Sei $\tilde{f}\colon (Y,y_0) \to  (E,e_0)$ eine Hebung, also $p \circ  \tilde{f} = f$. Wegen Funktorialität von $\pi_1$ ergibt sich dann
    \[
        f_* = (p \circ \tilde{f})_* = p_* \circ  \tilde{f}_*\colon  \pi_1(Y,y_0) \to  \pi_1(X,x_0)
    .\] 
    Also ist
    \[
        f_*(\pi_1(Y,y_0)) = (p_* \circ  \tilde{f}_*)(\pi_1(Y,y_0)) \subset p_*(\pi_1(E,e_0))
    .\] 
\end{proof}
