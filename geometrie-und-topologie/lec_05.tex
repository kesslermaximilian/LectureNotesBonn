\lecture{5}{Di 27 Apr 2021 12:16}{}
\begin{proof}[Beweis der Behauptung]
    Es ist klar, dass $V_1, V_2$ offen sind. Für Disjunktheit sehen wir mit
    \[
        X \setminus U_i \subset  q^{-1}(q(X\setminus U_i))
    .\] 
    dass $U_i \supset X \setminus q^{-1}(q(X\setminus U_i)) = V_i$ 
    Für Saturiertheit genügt es zu sehen, dass $g^{-1}(C)$ saturiert ist für alle $C\subset Z$, da
    \[
        q^{-1}(q(q^{-1}(C))) = q^{-1}(C)
    .\] 
    weil $q$ surjektiv ist. Wegen
     \[
         A\subset U_1 \implies X \setminus A \supset X \setminus U_1 \implies q(X\setminus A) \supset q(X\setminus U_1) \implies \underbrace{q^{-1}(q(X\setminus A))}_{=X\setminus A} = q^{-1}q(X\setminus U_1)
    .\]
    Komplementbildung liefert unser gewünschtes Eregbnis.
\end{proof}


\begin{example}
    $\R \mathbb{P}^n$ ist Hausdorffsch. 
    \begin{proof}
        Es ist $\R \mathbb{P}^n \cong S^n / x \sim  - x$. Sei
        \[
        q : S^n \to  S^n / x\sim -x
        .\] 
    \end{proof}
    die Projektion. Da $S^n$ kompakt und Hausdorffsch ist, ist  $\R \mathbb{P}^n$ Hausdorffsch genau dann, wenn $q$ abgeschlossen ist. Ist  $A\subset S^n$, so ist $q^{-1}(Q(A)) = A \cup -A$. \\
Da $-: S^n \to  S^n$ ein Homöomorphismus ist, ist $-A$ abgeschlossen, wenn  $A$ abgeschlossen ist. Dann ist auch  $A \cup -A$ abgeschlossen.
\end{example}
\begin{corollary}
    Sei $\sim $ auf $D^n = \left \{x \in \R^n \mid  \lVert x \rVert \leq 1\right\} $ erzeugt durch $x \sim -x$ für alle $x\in S^{n-1}\subset D^n$. Dann ist
    \[
    D^n / \sim  \cong \R \mathbb{P}^n
    .\] 
    Insbesondere ist 
    \[
        \R\mathbb{P}^1 \cong D^1 / \left \{-1,1\right\} \cong [0,1] / \left \{0,1\right\}  \cong S^1
    \]
    \label{cor:}
\end{corollary}
\begin{proof}
    Betrachte die stetige Abbildung
        \begin{equation*}
        f: \left| \begin{array}{c c l} 
        D^n & \longrightarrow & S^n \\
        x& \longmapsto &  (x,\sqrt{1-\lVert x \rVert ^2}) 
        \end{array} \right.
    \end{equation*}

    \todo{        Skizze einfügen für 'ausbeulen' der Abbildung.}
    Wir erhalten das Diagramm
    \begin{figure}[h]
        \centering
    \begin{tikzcd}
        D^n \ar{r} \ar{d} & S^n \ar{d} \\
    D^n / \sim \ar[dashed,swap]{r}{\overline{f}} & S^n / (x\sim -x) \cong \R\mathbb{P}^n
    \end{tikzcd}
    \end{figure}
    Wir sehen leicht, dass $\overline{f}$ bijektiv ist. Da $D^n$ kompakt, ist auch  $D^n / \sim $ kompakt, und $\R\mathbb{P}^n$ ist Hausdorffsch, also handelt es sich um einen Homöomorphismus (mit \ref{cor:compact-to-hausdorff-is-homeomorphism})
\end{proof}

\begin{corollary}
    Sei $X$ kompakt und Hausdorffsch und  $A\subset X$. Dann sind äquivalent
    \begin{enumerate}[1)]
        \item $X / A$ ist Hausdorffsch
        \item  $A$ ist abgeschlossen.
    \end{enumerate}
\end{corollary}
\begin{proof}
    '$1)\implies_2)$'. Ist $X // A$ Hausdorffsch, so ist die einpunktige Menge $\left \{A\right\} $ abgeschlossen (nach \ref{thm:point-in-hausdorff-space-is-closed}). Also ist $q^{-1}(A) = A$ abgeschlossen. \\
    '$2)\implies 1)$' Nach \ref{thm:quotient-space-of-compact-hausdorff-space} genügt es zu zeigen, dass $q: X \to  X / A$ abgeschlossen ist. Für $B\subset X$ abgeschlossen ist
    \[
        q^{-1}(q(B)) = \begin{cases}
            B & \text{falls } B\cap A = \emptyset \\
            B \cup A & \text{falls }B \cap  A \neq  \emptyset
        \end{cases}
    .\] 
    abgeschlossen, weil $A$ abgeschlossen ist. 
\end{proof}


\begin{example}
    \begin{enumerate}[a)]
        \item
Es ist $D^n / S^{n-1}$   Hausdorffsch. Alternativ können wir auch sehen, dass $D^n / S^{n-1} \cong S^n$ ist. Hierzu betrachte die Projektion:
    \begin{equation*}
    \begin{array}{c c l} 
    D^n & \longrightarrow & S^n \\
    x & \longmapsto &  \begin{cases}
        (2x, \sqrt{1-\lVert 2x \rVert ^2} & 0 \leq  \lVert x \rVert \leq \frac{1}{2}  \\
        \left( \frac{2-2\lVert x \rVert }{\lVert x \rVert }\cdot x, - \sqrt{1-(2-2\lVert x \rVert )^2}  \right) & \frac{1}{2} \leq  \lVert x \rVert  \leq 1
    \end{cases}
    \end{array}
\end{equation*}
Diese ist stetig, denn falls $\lVert x \rVert =\frac{1}{2}$ ist
\item Wir erhalten nun eine Abbildung 
\[
\frac{2-2\lVert x \rVert }{\lVert x \rVert } = \frac{2-1}{\frac{1}{2}} = 2
.\] 
und
\[
    \sqrt{1-\lVert 2x \rVert ^2} = \sqrt{1-1} =0 = - \sqrt{0} = -\sqrt{1-(2-2\lVert x \rVert )^2}  
.\] 
Ist $\lVert x \rVert =1$, so ist
 \[
\frac{2-2\lVert x \rVert }{\lVert x \rVert } = 0
.\] 
und somit ist $f(x) = (0,-1) \in \R^n \times \R$. Also faktorisiert $f$ über  $\overline{f} : D^n / S^{n-1} \to  S^n$. Wir sehen wieder leicht, dass $\overline{f}$ stetige Bijektion ist. Da $D^n / S^{n-1}$ kompakt und $S^n$ Hausdorffsch, folgt wieder, dass  $\overline{f}$ ein Homöomorphismus ist. \\
    \begin{tikzcd}
        S^n \ar{r}{q} & S^n / (x\sim -x) \cong \R\mathbb{P}^n \cong D^n / (x\sim -x) \ar{r} &  D^n / S^{n-1} \cong S^n
    \end{tikzcd}
    \end{enumerate}
\end{example}
\todo{Abbildung skizzieren}

\def\Base{\mathcal{S}} %temporary
\section{Basen und Subbasen}
\begin{definition}
    Sei $(X, \mathcal{O})$ ein topologischer Raum. Sei $\Base \subset \mathcal{O}$ eine Menge offener Mengen. Dann heißt $\Base$
    \begin{description}
        \item[\vocab{Basis}], falls $\forall U\subset \mathcal{O}$ existiert $S_i \in \Base$ mit $U = \bigcup_{i\in I} S_i$ 
        \item[\vocab{Subbasis}], falls $\forall U\in \mathcal{O}$ existieren $I, K_i$ endlich sowie  $S_k \in  \Base$ mit 
            \[
            U = \bigcup_{i\in I} \bigcap_{k\in K_i} S_k 
            .\] 
    \end{description}
\end{definition}
\begin{remark}
Ist     $\Base$ eine Basis, so ist $\Base$ eine Subbasis.
\end{remark}
\begin{example}
    Ist $(X,d)$ ein metrischer Raum, so ist
     \[
         \Base = \left \{U(x,ε) \mid  x\in X, ε>0\right\} 
    .\] 
    eine Basis der Topologie.
\end{example}
\begin{theorem}
    Sei $X$ eine Menge,  $\Base \subset \mathcal{P}(X)$ eine Menge von Teilmengen. Dann existiert genau eine Topologie auf  $X$, für die  $\Base$ eine Subbasis ist, nämlich:
     \[
    \mathcal{O} = \left \{U\subset X \mid  U = \bigcup_{i \in  I} \bigcap_{k\in K_i} S_k \text{ mit } \abs{K_i}<\infty, S_k \in  \Base  \right\} 
    .\] 
\end{theorem}
\begin{proof}
    Übung.
\end{proof}
\begin{notation}
    Wir nennen $\mathcal{O}$ die \vocab[Topologie!von $\Base$ erzeugte]{von $\Base$ erzeugte Topologie}.
\end{notation}
\begin{lemma}
    Sei $f: X \to  Y$ eine Abbildung zwischen topologischen Räumen, $\Base$ eine Subbasis von $Y$. Dann sind äquivalent:
    \begin{enumerate}[1)]
        \item $f$ ist stetig
        \item  $f^{-1}(S)$ ist offen für alle $S\in \Base$
    \end{enumerate}
\end{lemma}
\begin{proof}
    '$1) \implies 2)$' ist klar, da Subbasiselemente offen sind. \\
    '$2) \implies 1)$'. Sei $U \subset Y$ offen, dann $\exists K_i$ endlich und $S_k \in \Base$ mit
    \[
    U = \bigcup_{i \in  I} \bigcap_{k\in K_i} S_k
    .\] 
    Dann ist aber genau
    \[
        f^{-1}(U) = \bigcup_{i \in  I} \bigcap_{k\in K_i} \underbrace{f^{-1}(S_k)}_{\text{offen}} 
    .\] 
    offen, weil endliche Schnitte und beliebige Vereinigung offener Mengen offen sind. Also ist $f$ stetig.
\end{proof}

\begin{theorem}
    Eine Subbasis $\Base$ von  $(X, \mathcal{O})$ ist eine Basis genau dann, wenn
    \[
    \forall S_1, S_2 \in \Base \;\exists S_i \in \Base \colon S_1 \cap S_2 = \bigcup_{i \in I} S_i
    .\] 
\end{theorem}
\begin{proof}
'$\implies$'    Da $S_1,S_2 \in \Base$ sind diese offen. Dann ist auch $S_1\cap S_2$ offen. Ist $\Base$ Basis, dann gibt es also  $S_i \in  \Base$ mit 
\[
S_1 \cap  S_2 = \bigcup_{i \in  I} S_i
.\] 
'$\impliedby$' Angenommen, $U\subset X$ ist offen und von der Form
\[
    U = \bigcup_{i \in  I} \left( \bigcap_{k\in K_i} S_k \right) 
.\] 
mit $K_i$ endlich und  $S_k \in  \Base$. Nach Annahme ist
\[
\bigcap_{k\in K_i} = \bigcup_{j\in J_i} S_j  
.\] 
und damit ist
\[
U = \bigcup_{i\in I} \bigcup_{j\in J_i} S_j  
.\] 
\end{proof}
\begin{remark}
    Nach Annahme ist eigentlich erstmal der Schnitt von 2 Mengen die Vereinigung von $S_i$. Allerdings kann man dies per Induktion leicht auf  $n$ Teilmengen verallgemeinern, wenn wir
     \[
         \bigcap_{k=1}^n S_k = S_1 \cap  \bigcap_{k=2}^{n} S_k = S_1 \cap \bigcup_{i\in I} S_i = \bigcap_{i\in I} (S_i \cap S_k) = \bigcup_{i\in I} \bigcup_{j\in J_i} S_j  
    .\]
    für geeignete $S_i, S_j\in \Base$ schreiben.
\end{remark}
\begin{theorem}[Satz von Alexander]
    \label{thm:alexander}
    Sei $X$ ein topologischer Raum und  $\Base$ eine Subbasis. Dann ist  $X$ kompakt genau dann, wenn jede Überdeckung durch Elemente aus  $\Base$ eine endliche Teilüberdeckung besitzt.
\end{theorem}
\begin{proof}
    '$\implies$' ist klar. \\
    '$\impliedby$' Angenommen, $X$ ist nicht kompakt, dann betrachte die Menge
     \[
    \mathcal{C} := \left \{U \mid  U \text{ offene Überdeckung \underline{ohne} endliche Teilüberdeckung}\right\} \neq \emptyset
    .\] 
    Es ist $\mathcal{C}$ partiell geordnet, indem wir $U\leq U'$ für $U\subset U'$ setzen. \\
    Ist $U_1\subset U_2\subset \ldots$ eine Kette, so ist $\bigcup_{U_i}\in \mathcal{C}$, denn
    \begin{itemize}
        \item Offenbar ist $\bigcup_{i \in  I} U_i$ eine offene Überdeckung.
        \item Hat $\bigcup_{i \in  I} U_i$ eine endliche Teilüberdeckung, so ist diese schon in einem $U_i$ enthalten, und damit enthielte auch dieses  $U_i$ bereits eine endliche Teilüberedckung \contra
    \end{itemize}
Wir können also das Lemma von Zorn anwenden, und somit existiert ein maximales Elment $U\in \mathcal{C}$.
\begin{claim}
    Ist $V\subset X$ offen und  $V\not\in U$, so hat $U\cup \left \{V\right\} $ eine endliche Teilüberdeckung
\end{claim}
\begin{subproof}
    Sonst wäre $U \cup \left \{V\right\} \in \mathcal{C}$ und somit wäre $U$ nicht maximal
\end{subproof}
\begin{claim}
    $U \cap \Base$ ist keine Überdeckung
\end{claim}
\begin{subproof}
    Sonst hätte $U$ eine endliche Teilüberedckung nach Annahme.
\end{subproof}
Wegen Behauptung 2 existiert $x\in X$, der nicht von $U \cap \Base$ überdeckt wird. Sei $W\in U$ mit $x\in W$. Da $W$ offen ist, folgt
 \[
W = \bigcup_{i \in  I} \bigcap_{k\in K_i} S_k
.\] 
mit $K_i$ endlich und  $S_k \in \Base$. Dann existieren also $S_1,\ldots,S_n$ mit 
\[
x \in  \bigcap_{i=1}^n S_i \subset W 
.\] 
Da $x$ nicht von  $U \cap \Base$ überdeckt wird, ist $S_i \not\in U$. Aus der ersten Behauptung wissen wir nun aber, dass es $U_1^i, \ldots, u_{n_i}^i \in U$ mit
\[
\left \{U_j ^i\right\} _{j=1}^n \cup \left \{S_i\right\}  \quad \text{ ist Überdeckung von } X
.\] 
Sei nun 
\[
\hat{U} := \left \{U_j ^i \mid  1\leq i\leq n, 1\leq j\leq n_i\right\} \subset U
.\] 
Für alle $i$ gilt also
 \[
X \subset \bigcup_{V\in \hat{U}} V \cup S_i 
.\] 
Also folgt
\[
X \setminus \bigcup_{V\in \hat{U}} V \subset S_i 
.\] 
und damit ist auch
\[
X\setminus \bigcup_{V\in \hat{U}} V \subset S_1 \cap \ldots \cap S_n \subset W \in U 
.\] 
Also ist $\hat{U}\cup \left \{W\right\} $ eine endliche Teilüberdeckung von $U$, \contra.
\end{proof}
