\subsection*{Blatt 1}
\begin{aufgabe}
Es sei $(X,d)$ ein metrischer Raum und $x\in X$ ein Punkt. Dann ist die Abbildung
\begin{align*}
d_x\colon X& \to \mathbb R\\
y& \mapsto d(x,y)
\end{align*}
stetig.
\end{aufgabe}

\begin{aufgabe}
Wir betrachten die Menge $\mathbb N_{>0}$ mit der euklidischen Metrik $d_1$, d.h.\ $d_1(n,m):=|n-m|$, der diskreten Metrik $d_2$ und der Metrik $d_3$ gegeben durch $d_3(n,m):=|\tfrac{1}{n}-\tfrac{1}{m}|$.
\begin{enumerate}[i)]
	\item Die Metriken $d_1, d_2$ und $d_3$ sind paarweise nicht äquivalent.
	\item Die Metriken $d_1, d_2$ und $d_3$ induzieren dieselbe Topologie auf $\N_{>0}$.
\end{enumerate}
\end{aufgabe}

\begin{aufgabe}
	Auf $\mathbb{N}$ betrachten wir die Menge von Teilmengen $\cO_{ko-endl}$ für die gilt: $U\in \cO_{ko-endl}$ genau dann wenn $U$ leer oder $\N\setminus U$ endlich ist.
\begin{enumerate}[i)]
	\item $\cO_{ko-endl}$ ist eine Topologie auf $\N$ (die ko-endliche Topologie).
	\item Es seien $U_1, U_2 \in \cO_{ko-endl}$ nicht leer. Dann ist auch $U_1\cap U_2$ nicht leer.
	\item Sei $(X,d)$ ein metrischer Raum. Dann ist jede stetige Abbildung $f\colon (\mathbb N,\cO_{ko-endl})\to (X,d)$ konstant.
	\item $(\mathbb N,\cO_{ko-endl})$ ist nicht metrisierbar.
\end{enumerate}
\end{aufgabe}

\begin{aufgabe}
Es sei $Y= \{a,b\}$, mit der Topologie $\mathcal T= \{ \emptyset, \{a\}, Y\}$. Zudem sei $X$ ein topologischer Raum.
\begin{enumerate}[i)]
	\item Eine Abbildung $f\colon X\to Y$ ist stetig genau dann, wenn $f\inverse(a)\subset X$ offen ist. 
	\item Die Zuordnung 
		\begin{align*}
		\{ \textup{stetige Abbildungen } X\to Y\} & \to \{ \textup{offene Teilmengen in } X\} \\
		f& \mapsto f\inverse (a)
		\end{align*}
		ist bijektiv.
\end{enumerate}
\end{aufgabe}

