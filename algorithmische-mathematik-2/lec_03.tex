\lecture[Diskrete Verteilungen. Gleichverteilung. Fixpunkte von Permutationen. Empirische Verteilung.]{Mo 19 Apr 2021 10:23}{Gleichverteilung, empirische Verteilung}
Wir stellen fest, dass es im letzten Beispiel auch genügt hätte, $\mathbb{P}(\left \{k\right\} )$ für $k=1,\ldots,10$ anzugeben, das motiviert Folgendes:
\begin{theorem}\label{thm:wahrscheinlichkeitsverteilungen-auf-abzaehlbaren-raeumen}
    \begin{enumerate}[label=\protect\circled{\alph*}]
        Sei $\Omega$ abzählbar.
        \item Sei $p(\omega)\in [0,1], \omega\in \Omega$, sodass
            \[
                \sum_{\omega\in \Omega}p(\omega) = 1
            .\] 
            Dann ist $\mathbb{P}$ definiert durch:
                \begin{equation*}
                \mathbb{P}: \left| \begin{array}{c c l} 
                    \mathcal{P}(\Omega) & \longrightarrow & [0,1] \\
                    A & \longmapsto &  \sum_{\omega\in A}p(\omega)
                \end{array} \right.
            \end{equation*}
            eine Wahrscheinlichkeitsverteilung auf $(\Omega,\mathcal{P}(\emptyset))$. 
        \item Jede Wahrscheinlichkeitverteilung $\mathbb{P}$ auf $(\Omega,\mathcal{P}(\Omega))$ hat obige Form, wobei $p(\omega) = \mathbb{P}(\left \{\omega\right\} )$.
    \end{enumerate}
\end{theorem}

\begin{remark}
    $p:\emptyset\to [0,1]$ heißt Massenfunktion der Wahrscheinlichkeitverteilung $\mathbb{P}$. \\
    \warning Der Satz gilt nicht für $\Omega$ überabzählbar.
\end{remark}


\begin{remark}
    Sei $A$ abzählbar und  $p(\omega) \geq 0$ für $\omega\in A$. Dann definiert
    \[
        \sum_{\omega\in A} p(\omega) := \sum_{k\geq 1} p(\omega_k)
    .\] 
    mit einer beliebigen Abzählung $\omega_1,\omega_2, \ldots, \omega_k$ von $A$ eine wohldefinierte Summe der  $p(\omega)$. Es ist wichtig, dass hier $p(\omega) \geq 0$, sonst ist obiges nicht wohldefiniert.
\end{remark}

\begin{lemma}\label{lm:approximation-von-massen-durch-endliche-teilmengen}
        Sei  $A$ abzählbar.
    \begin{enumerate}[label=\protect\circled{\alph*}]
        \item Sei $p(\omega) \in [0,1]$ für alle $\omega$. Dann ist
            \[
                \sum_{\omega\in A} p(\omega) \in [0,\infty]
            .\] 
            wohldefiniert. Setzen wir
            \[
                \mathbb{P}(A) := \sum_{\omega\in A} p(w)
            .\] 
            so gilt
            \[
                P(A) = \sup_{\substack{F\subset A \\ \abs{F} <\infty}} P(F)
            .\] 
            und $P(A) \leq  P(B)$ für $A\subset B$.
        \item Ist $A = \bigsqcup_{k=1}^{\infty} A_k$ eine disjunkte Vereinigung, so ist
            \[
                P(A) = \sum_{k=1}^{\infty}P(A_k)
            .\] 
    \end{enumerate}
\end{lemma}
\begin{proof}
    \begin{enumerate}[label=\protect\circled{\alph*}]
        \item Sei $\omega_1,\omega_2,\ldots$ eine beliebige Abzählung von $A$. Dann ist die Funktion
             \[
                 n \longmapsto \sum_{k=1}^n p(\omega_k)
            .\] 
            monoton wachsend. Also ist
            \[
                \sum_{k=1}^{\infty} p(\omega_k) := \lim_{n\to \infty}\sum_{k=1}^{n} p(\omega_k) = \sup_{n\in \N} \sum_{k=1}^n p(\omega_k) \in [0,\infty]
            .\]
            wohldefiniert. \\
            Wir wollen nun noch zeigen, dass 
            \[
                P(A) := \sum_{k=1}^{\infty} p(\omega_k) \stackrel{!}{=} \sup_{\substack{F\subset A \\ \abs{F} <\infty} }P(F)
            .\] 
            Die Ungleichung '$\leq $' folgt sofort, da wir mit $F_n := \left \{\omega_1, \ldots, \omega_k\right\} $ feststellen, dass
            \[
                \sum_{k=1}^n p(\omega_k) = \sum_{\omega\in F_n} p(\omega) = P(F_n) \leq  \sup_{\substack{F\subset A \\ \abs{F} <\infty} } P(F)
            .\] 
            Also ergibt sich im Limes genau wie gewünscht
            \[
                P(A) = \lim_{n\to \infty} \sum_{k=1}^n p(\omega_k) \leq  \lim_{n\to \infty} \sup_{\substack{F\subset A\\ \abs{F} <\infty} }P(F) = \sup_{\substack{F\subset A\\ \abs{F} <\infty} }P(F)
            .\] 
            Für '$\geq $' stellen wir fest, dass es für jedes $F\subset A$ endlich ein $n\in \N$ gibt, sodass $F\subset \left \{\omega_1,\ldots,\omega_n\right\} $, und somit ist
            \[
                P(F) = \sum_{\omega\in F} P(\omega) \leq  \sum_{k=1}^n p(\omega_k) \leq  \sum_{k=1}^{\infty} p(\omega_k) = P(A)
            .\] 
            und somit ist das Supremum der $P(F)$ für $F\subset A, \abs{F} <\infty$ durch $P(A)$ beschränkt. \\
            Für die letzte Behauptung sehen wir mit $A\subset B$ leicht, dass
            \[
                P(A) = \sup_{\substack{F\subset A \\ \abs{F} <\infty} } P(F) \leq  \sup_{\substack{ F\subset B \\ \abs{F} <\infty}} = P(B)
            .\] 
        \item ($\sigma$-Additivität) Wir unterscheiden zwei Fälle:
            \begin{enumerate}[1)]
                \item Falls $\abs{A} <\infty$, so ist $A = \bigsqcup_{k=1}^n A_k$ für ein $n$, und somit ist
                    \begin{equation}
                        \begin{split}
                            P(A) &= \sum_{l=1}^{\abs{A} }p(\omega_l) = \sum_{l=1}^{\abs{A} } \sum_{k=1}^{n} p(\omega_l) \mathbb{1}_{A_k}(\omega_l)  \\
                                          &= \sum_{k=1}^{n} \sum_{l=1}^{\abs{A} } p(\omega_l) \mathbb{1}_{A_k}(\omega_l) = \sum_{k=1}^n P(A_k)
                        \end{split}
                    \end{equation}
                \item Sei nun $\abs{A} =\infty$. Wir zeigen zunächst '$\leq $'. Für ein endliches $F\subset A$ ist
                    \[
                        F = \bigcup_{k=1}^{\infty} (F \cap A_k)
                    .\] 
                    eine disjunkte Vereinigung mit endlich vielen Termen, also ist
                     \[
                         P(F) = \sum_{k=1}^{\infty} P(F \cap A_k) \leq  \sum_{k=1}^{\infty} P(A_k)
                    .\] 
                    und somit liefert das Supremum über beide Seiten, dass
                    \[
                        P(A) = \sup_{\substack{F\subset A \\ \abs{F}<\infty } } P(F) \leq  \sum_{k=1}^{\infty} P(A_k)
                    .\] 
                    Wir zeigen nun '$\geq $'.
                    \begin{idea}
                        Wir können $P(A_k) = \sup_{\substack{ F_k \subset A_k \\ \abs{F_k}]\infty} } P(F_k)$ schreiben und 'optimieren' nun jedes einzelne $F_k$.
                    \end{idea}
                    Seien also $F_k \subset A_k$ jeweils endlich. Dann ist $F_k \cap F_l \subset  A_k \cap A_l = \emptyset$, also sind auch die $F_k$ paarweise disjunkt, und wir lernen
                    \begin{equation}
                         \sum_{k=1}^n P(A_k) = \sum_{k=1}^n \sup_{\substack{F_k \subset A_k \\ \abs{F_k} <\infty} } P(F_k) = \sup_{\substack{F_1\subset A_1\\abs{F_1<\infty} } } \ldots \sup_{\substack{ F_k \subset A_k \\ \abs{F_k} <\infty}} \sum_{k=1}^n P(F_k)
                    \end{equation}
                    Also ist
                    \[
                        \sum_{k=1}^n P(F_k) = P\left( \bigcup_{k=1}^n F_k \right) \leq P\left( \bigcup_{k=1}^{\infty}F_k \right) \leq P\left( \bigcup_{k=1}^{\infty}A_k \right)  \stackrel{\text{def}}{=} P(A)
                    .\] 
                    setzen wir dies nun in die rechte Seite von (1) ein, so ergibt sich
                    \[
                        \sum_{k=1}^n P(A_k) \leq  P(A) \qquad \implies \qquad \sum_{k=1}^{\infty} P(A_k) \leq  P(A)
                    .\] 
            \end{enumerate}
    \end{enumerate}
\end{proof}
\begin{proof}[Beweis von \autoref{thm:wahrscheinlichkeitsverteilungen-auf-abzaehlbaren-raeumen}]
    \begin{enumerate}[label=\protect\circled{\alph*}]
        \item Es gilt
            \[
                \sum_{\omega\in \Omega}p(\omega) = P(\Omega) = 1
            .\] 
            nach Voraussetzung. Die $\sigma$-Additivität folgt nun aus \autoref{lm:approximation-von-massen-durch-endliche-teilmengen}. Deswegen ist $P(A)$ eine Wahrscheinlichkeitsverteilung.
        \item Da $P$ $\sigma$-additiv ist, ist $\forall A \subset \Omega$:
            \[
                PA) = P\left(\bigcup_{\omega\in A} \left \{\omega\right\} \right) = \sum_{\omega\in A} P(\left \{\omega\right\} )
            .\] 
            und dies hat genau die angegebene Form mit $p(\omega) := P(\left \{\omega\right\} )$
    \end{enumerate}
\end{proof}


\subsection{Die Gleichverteilung}
Sei $\Omega$ endlich $(\neq \emptyset)$ und betrachte $\sigma$-Algebra $\mathcal{F} = \mathcal{P}(\Omega)$. \\
Die \vocab{Gleichverteilung} ist die Wahrscheinlichkeitverteilung, die ein unifromes "Gewicht" (Massenfunktion) auf die Elementarereignisse verteilt:
\[
    \forall \omega\in \Omega : p(w) = \mathbb{P}\left( \left \{\omega\right\}  \right) = \frac{1}{\abs{\Omega} }
.\] 
Aus \autoref{thm:wahrscheinlichkeitsverteilungen-auf-abzaehlbaren-raeumen} folgt dann bereits, dass $\forall A\subset \Omega\colon$
\[
    \mathbb{P}(A) = \sum_{\omega\in A} p(\omega) = \frac{\abs{A} }{\abs{\Omega} }
.\] 
\begin{example}
    \begin{enumerate}[label=\protect\circled{\alph*}]
        \item Betrachte $n$ Würfe eines fairen Würfels. In diesem Fall ist  $\Omega = \left \{1,2,3,4,5,6\right\} ^n = \left \{\omega = (\omega_1,\ldots,\omega_n) \mid  \omega_k \in \left \{1,\ldots,6\right\} \right\} $ und somit $\abs{\Omega}=6^n$ und die Gleichverteilung ist gegeben durch
            \[
                \mathbb{P}(\omega) = \frac{1}{6^n}
            .\] 
        \item (Zufällige Permutationen). 
            \begin{itemize}
                \item Eine Permutation $\sigma \in \mathfrak{S}_n$ von $\left \{1,\ldots,n\right\} $ ist eine Abbildung von $\left \{1,\ldots,n\right\} $ nach $\left \{1,\ldots,n\right\} $, die bijektiv ist. Oft schreiben wir 
                     \[
                         \sigma = \begin{pmatrix} 1 & 2 & 3 & 4 \\ 4 & 3 & 1 & 2 \end{pmatrix} 
                    .\] 
                    und meinen damit $\sigma(1) = 4$,  $\sigma(2) = 3$,  $\sigma(3) = 1$,  $\sigma(4)=2$. Manschmal schreiben wir dann auch
                     \[
                         \sigma = (4,3,1,2) = (\sigma_1, \sigma_2, \sigma_3,\sigma_4)
                    .\] 
                \item Sei $\Omega = \mathfrak{S}_n$ die Menge aller Permutation von $\left \{1,\ldots,n\right\} $. Dann ergibt sich
                    \[
                    \abs{\mathfrak{S}_n} =n!
                    .\] 
                    Also ergibt sich für die Gleichverteilung eine Wahrschenlichkeit von
                    \[
                        \mathbb{P}(\sigma) = \frac{1}{n!} \quad \forall \sigma \in \mathfrak{S}_n
                    .\] 
            \end{itemize}
    \end{enumerate}
\end{example}
\begin{example}
    Sei $N$ die Anzahl von Karten eines Kartenspiels, die  \underline{gut gemischt} sind, d.h. jede Reihenfolge ist gleich wahrscheinlich. 
    \begin{enumerate}[(1)]
        \item Was ist die Wahrscheinlichkeit, dass die $k$-te Karte auf der $l$. Stelle ist? D.h, was ist:
             \[
                 \mathbb{P}(\left \{\omega\in \mathfrak{S}_n \mid  \omega(k) = l\right\} )
            .\]
            \begin{solution}
                
            Es ergibt sich
            \[
                \mathbb{P}(\left \{\omega\in \mathfrak{S}_n \mid  \omega(k) = l\right\} ) = \frac{\abs{\left \{\omega\in \mathfrak{S}_n \mid  \omega(k) = l\right\} } }{\abs{\Omega} } = \frac{(n-1)!}{n!} = \frac{1}{n}
            .\] 
            \end{solution}
        \item Was ist die Wahrscheinlichkeit, dass eine Karte 'auf ihrer Stelle' ist, dh.
            \[
                \mathbb{P}(\left \{\omega \mid  \exists k \colon \omega(k) = k\right\} )
            .\] 
            \begin{solution}
                
            Definiere die Ereignisse $A_k := \left \{\omega(k) = k\right\} $. Diese sind nicht disjunkt für verschiedene $k$. Es ergibt sich:
            \begin{equation}
                \begin{split}
                    \mathbb{P}(\exists k \colon \omega(k) &= k) = \mathbb{P}\left(\bigcup_{k=1}^n A_k\right)  \\
                                                          &= \sum_{k=1}^n (-1)^{k-1} \sum_{1\leq i_1<\ldots<i_k \leq n} \underbrace{\mathbb{P}(A_{i_1} \cap A_{i_2} \cap \ldots\cap A_{i_k})}_{= \frac{(n-k)!}{n!}} \\
                                                          &= \sum_{k=1}^n (-1)^{k-1} \frac{(n-k)!}{n!} \underbrace{\sum_{1\leq i_1<\ldots<i_k \leq n} 1}_{= \binom{n}{k}} \\
                                                          &=\sum_{k=1}^n (-1)^{k-1}\frac{(n-k)!}{n!} \cdot  \frac{n!}{(n-k)! k!} \\
                                                          &= -\sum_{k=1}^n \frac{(-1)^{k} }{k!} \\
                                                          &= 1-\frac{1}{e} + \sum_{k=n+1}^{\infty} \frac{(-1)^k}{k!}
                \end{split}
            \end{equation}
            Für $n\to \infty$ geht das gegen $1-\frac{1}{e}\in (0,1)$ .
            \end{solution}
    \end{enumerate}
\end{example}
\begin{remark*}
    Wir verwenden hier die Exponentialreihe, d.h.
    \[
e^x := \exp (x) := \sum_{k=0}^{\infty} \frac{x^k}{k!}
    .\] 
    konvergiert absolut und auf ganz $\R$, insbesondere für $x=-1$ gegen  $\frac{1}{e}$
\end{remark*}
\subsection{Die empirische Verteilung}
Diese wird aus den Beobachtungen definiert.
\begin{definition*}[Empirische Verteilung]\label{def:empirische-verteilung}
Seien $x_1,x_2,\ldots,x_n \in \Omega$ $n$ Beobachtungen. Setze
 \[
     N(A) := \abs{\left \{k\in \left \{1,\ldots,n\right\} \mid  x_k \in A\right\} } 
.\] 
Denn definiert
\[
    \mathbb{P}(A) = \frac{N(A)}{n}
.\] 
die \vocab{empirische Häufigkeit} von $A$. $\mathbb{P}$ nennen wir die \vocab{empirische Verteilung}. Zudem ist
\[
    p(\omega) = \frac{N(\left \{\omega\right\} )}{n}
.\] 
die \vocab{relative Häufigkeit} von $\omega\in \Omega$.
\end{definition*}

\begin{example}
    Die empirische Verteilung von $n$ Zufallswürfeln eines Würfels wird gegeben durch $x_1,\ldots,x_n \in \left \{1,\ldots,6\right\} $. Die Plots für $p_k := \frac{N(k)}{n}$ für verschieden $n$ sehen wie folgt aus:
\end{example}
    \todo{Plots einfügen}
