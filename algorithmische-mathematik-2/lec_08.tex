\lecture[Mehrstufie Modelle. Produktmodelle.]{Mi 05 Mai 2021 10:11}{Mehrstufige Modelle}
\subsection{Mehrstufige Modelle}
Sei eine Folge von $n$ Zufallsexperimenten in den Wahrscheinlichkeitsräumen  $\Omega_1,\Omega_2,\ldots,\Omega_n$ gegeben. Wir definieren ein \vocab{$n$-stufiges Zufallsexperiment} durch
\begin{itemize}
    \item $\Omega = \Omega_1\times \Omega_2\times \ldots\times \Omega_n = \left \{\omega=(\omega_1,\ldots,\omega_n) \mid  \omega_k \in \Omega_k, 1\leq k\leq n\right\} $ 
    \item $\mathcal{F} = \mathcal{P}(\Omega)$.
    \item Definiere die Zufallsvariablen
        \[
            X_k(\omega) = \omega_k \qquad 1\leq k\leq n
        .\] 
        Den Index $k$ interpretieren wir hierbei als Zeit.  $k \mapsto  X_k$ ist eine Trajektion von $X = (X_1,\ldots,X_n)$ ?
    \item $\mathbb{P}=?$. Wir konstruieren $\mathbb{P}$ auf $(\Omega,\mathcal{F})$ mit
        \begin{enumerate}[label=\protect\circled{\alph*}]
            \item Der Anfangsverteilung $\mathbb{P}(X_1=x_1) := p_1(x_1)$ für alle $x_1\in \Omega_1$.
            \item Den bedingten Verteilungen
                 \[
                     \mathbb{P}(X_k = x_k \mid  X_1 = x_1, X_2 = x_2, \ldots, X_{k-1} = x_{k-1}) =: p_k(x_k\mid x_1,\ldots,x_{k-1})
                .\] 
                für alle $x_l\in \Omega_l$, $1\leq l\leq k-1$, sodass $\mathbb{P}(X_1=x_1,\ldots,X_{k-1}=x_{k-1})\neq 0$.
        \end{enumerate}
\end{itemize}i
\begin{remark*}
    Man kann das Allgemeiner machen, indem wir $\mathcal{F}$ als die Produktsigmalalgebra der $\mathcal{F}_i$ wählen.
\end{remark*}
\begin{theorem}\label{thm:stufenmodell}
    Sei $p_i(\cdot )$ die Massenfunktion einer Wahrscheinlichkeitsverteilung auf $\Omega_i$ und $p_k(\cdot \mid x_1,\ldots,x_{k-1})$ für alle $1\leq k\leq n$ mit $x_1\in \Omega_1, \ldots,x_{k-1} \in \Omega_{k-1}$ eine Massenfunktion auf $\Omega_k$. \\
    Dann existiert eine eindeutige Wahrscheinlichkeitsverteilung $\mathbb{P}$ auf $(\Omega,\mathcal{F})$, sodass
    \begin{enumerate}[label=\protect\circled{\alph*}]
        \item $\mathbb{P}(X_1=x_1) = p_1(x_1) \quad \forall x_1\in \Omega_1$
        \item $\mathbb{P}(X_k = x_k) \mid  X_1 = x_1,\ldots,X_{k-1}=x_{k-1}) = p_k(x_k \mid  x_1,\ldots,x_{k-1})$
    \end{enumerate}
    Die Wahrscheinlichkeitsverteilung $\mathbb{P}$ hat die Massenfunktion
    \[
        p(x_1,\ldots,x_n) = p_1(x_1) p_2(x_2\mid x_1) \cdot \ldots\cdot p_n(x_n \mid x_1,\ldots, x_{n-1})
    .\] 
\end{theorem}
\begin{proof}
    \begin{enumerate}[1)]
        \item Nimm zunächst an, dass solch ein Maß existiert, wir zeigen die letzte Aussage. Sei $\mathbb{P}$ sodass \circled{a} und \circled{b} erfüllt sind. Dann ist
             \[
                 \forall 1\leq k\leq n \colon \mathbb{P}(X_1=x_1,\ldots,X_k=x_k) = p(x_1,\ldots,x_k)
            .\] 
            \begin{itemize}
                \item Für $k=1$ gilt das (aus \circled{a}).
                \item Falls es für $k-1$ gilt, so haben wir die Fälle
                \item  $p(x_1,\ldots,x_{k-1})=0$, dann ist $0=0$ wahr.
                \item Falls $p(x_1,\ldots,x_{k-1})\neq 0$, so ist
                    \[
                        \begin{split}
                        &\; \mathbb{P}(X_1=x_1,\ldots,X_k=x_k) \\
                        &= \mathbb{P}(X_k = x_k \mid  X_1 = x_1, \ldots, X_{k-1} = x_{k-1})\cdot \mathbb{P}(X_1=x_1, \ldots, X_{k-1} = x_{k-1}) \\
                        &= p_1(x_1)\cdot \ldots p_{k-1}(x_{k-1}\mid x_1,\ldots,x_{k-2}r\cdot p_k(x_k\mid x_1,\ldots,x_{k-1}) = p(x_1,\ldots,x_k)
                        \end{split}
                    .\] 
            \end{itemize}
    \end{enumerate}
    Normierung: $\forall x\in \Omega, x = (x_1,\ldots,x_n)$ mit $x_k \in \Omega_k$ ist
    \[
        \begin{split}
            \sum_{x\in \Omega} p(x) &= \sum_{x_1\in \Omega_1} \ldots \sum_{x_n \in \Omega_n} p(x_1,\ldots.,x_n) \\
                                    &= \sum_{x_1\in \Omega_1} p(x_1) \sum_{x_2\in \Omega_2} p(x_2\mid x_1) \ldots \sum_{x_n \in \Omega_n} p(x_n \mid  x_1,\ldots,x_{n-1}) = 1
        \end{split}
    .\] 
    Für Eigenschaft \circled{b} ist
    \begin{equation*}
        \begin{split}
            \mathbb{P}(X_1&=x_1,\ldots,X_k=x_k) = \sum_{x_{k+1}\in \Omega_{k+1}} \ldots \sum_{x_n \in \Omega_n} p(x_1,\ldots,x_n) \\
                          &= p_1(x_1)\ldots p_{k-1}(x_{k-1}\mid  x_1,\ldots,x_{k-2}) p_k(x_k \mid  x_1,\ldots,x_{k-1}) \\
                          &= \mathbb{P}(X_1=x_1,\ldots,X_{k-1}=x_{k-1})p_k(x_k \mid  x_1,\ldots,x_{k-1})
        \end{split}
    \end{equation*}
    Also erhalten wir
    \[
        \mathbb{P}(X_k = x_k \mid  X_1 = x_1,\ldots,X_{k-1} = x_{k-1}) = p_k(x_k \mid  x_1,\ldots,x_{k-1})
    .\] 
\end{proof}
\todo{Beweis nochmal sortieren}
\begin{note}
    Mir ist noch nicht klar, wo wir im Beweis des Satzes jetzt gezeigt haben wollen, dass solch ein Wahrscheinlichkeitsmaß existiert, das muss ich noch ausarbeiten.
\end{note}
\begin{remark}
    Falls $p_k(x_k \mid  x_1,\ldots,x_{k-1})$ nur eine Funktion von $x_{k-1},\ldots,x_{k-m-1}$, dann sagen wir, dass unser Modell ein Gedächtnis von $m$ Schritten hat.
\end{remark}

\subsubsection{Produktmodelle}
Falls $p_k(x_k \mid  x_1 = \ldots = x_{k-1}) = p_k(x_k)$, d.h. $x_k$ hängt nicht von den Werten  $x_1,\ldots,x_{k-1}$ ab. Dann erhalten wir aus \autoref{thm:stufenmodell}, dass
\[
    p(x_1,\ldots,x_n) = \prod_{k=1}^n p_k(x_k)
.\] 

\begin{definition}[Produktmodell]\label{def:produktmodell}
    Die Wahrscheinlichkeitsverteilung $\mathbb{P}$ auf $\Omega = \Omega_1\times \ldots\times \Omega_n$ mit Massenfunktion
    \[
        p(x_1,\ldots,x_n) = \prod_{k=1}^n p_k(x_k)
    .\] 
    heißt \vocab{Produkt von $\mathbb{P}_1,\ldots,\mathbb{P}_n$}. ($\mathbb{P}_k$ hat Massenfunktion $p_k$).
\end{definition}
\begin{notation}
    Wir schreiben $\mathbb{P} = \mathbb{P} \otimes \mathbb{P}_2 \otimes \ldots\otimes  \mathbb{P}_n$, wenn $\mathbb{P}$ das Produkt von $\mathbb{P}_1,\ldots,\mathbb{P}_n$ ist.
\end{notation}
\begin{example}
    Seien $n$ unabhängige  0-1-Experimente mit Erfolgswahrscheinlichkeit  $p$ gegeben. Also
     \[
    \Omega_1=\ldots=\Omega_n =\left \{0,1\right\} 
    .\] 
    und $p_k(1) = p = 1-p_k(0)$ für  $k=1,\ldots,n$. dann ist $p_k(x) = (1-p)\left( \frac{p}{1-p} \right) ^x$ für $x\in \left \{0,1\right\} $. Die entstehende Verteilung
    \[
        p(x_1,\ldots,x_n) = (1-p)^n \prod_{k=1}^n \left( \frac{p}{1-p} \right) ^{x_k}
    .\] 
    ist die \vocab{$n$-dimensionale Bernoulli-Verteilung mit Parameter  $p$ }.
\end{example}
\begin{theorem}
    Sei $(\Omega, \mathcal{F}, \mathbb{P})$ ein Produktmodell. Dann ist für beliebige Ereignisse $A_k \subset \Omega_k$, $k=1,\ldots,n$:
    \[
        \mathbb{P}(A_1\times \ldots\times A_n ) = \prod_{k=1}^n \mathbb{P}_k(A_k)
    .\] 
    und $\mathbb{P}(\tilde{A}_k) = \mathbb{P}_k(A_k)$, wobei
    \[
    \tilde{A}_k := \Omega_1\times \ldots\times \Omega_{k-1}\times A_k \times \Omega_{k+1} \times \ldots\times \Omega_n
    .\] 
    Deswegen sind $\tilde{A}_1,\ldots,\tilde{A}_n$ unabhängige Ereignisse.
\end{theorem}
\begin{proof}
    Es ist
    \begin{equation*}
        \begin{split}
            \mathbb{P}(A_1\times \ldots\times A_n) &= \mathbb{P}((X_1,\ldots,X_n)\in A_1\times \ldots\times A_n))  \\
                                                   &= \sum_{(x_1,\ldots,x_n) \in A_1\times \ldots\times A_n} p(x_1,\ldots,x_n) \\
                                                   &= \sum_{x_1\in A_1} \ldots\sum_{x_n\in A_n} p_1(x_1)\cdot \ldots p_n(x_n) \\
                                                   &=\prod_{k=1}^n \sum_{x_k\in A_k} p_k(a_k) \\
                                                   &= \prod_{k=1}^n \mathbb{P}(A_k)
        \end{split}
    \end{equation*}
    Es ergibt sich leicht
    \begin{IEEEeqnarray*}{rCl}
        \mathbb{P}(\tilde{A}_k) &=& \mathbb{P}(X_k \in A_k, X_l \in \Omega_l \; \forall l\neq k) \\
                                & = & \left( \prod_{l\neq k}\underbrace{ \mathbb{P}_l(X_l \in \Omega_l) }_{=1} \right)\cdot \mathbb{P}_k(X_k\in A_k) \\
                                & = & \mathbb{P}k(A_k)
    \end{IEEEeqnarray*}
    Damit ergibt sich schlussendlich für beliebiges $I = \left \{i_1,\ldots,i_l\right\}$:
    \begin{IEEEeqnarray*}{rCl}
    \mathbb{P}(\tilde{A}_{i_1} \cap  \ldots \cap \tilde{A}_{i_l}) & = & \mathbb{P}\left(\bigcap_{i \in  I} \left \{X_i \in A_i\right\} \cap \bigcap_{j\neq I} \left \{x_j \in \Omega_j\right\} \right) \\
                                                                  & = & \prod_{i\in I} \mathbb{P}_i(A_i) \cdot  \prod_{j\neq I} \underbrace{\mathbb{P}_j(\Omega_j)}_{=1} \\
                                                                  & = & \prod_{i \in I}\mathbb{P}_i(A_i) \\
                                                                  & \stackrel{\mathbb{P}(A_i) = \mathbb{P}(\tilde{A}_i)}{=} & \prod_{k=1}^l \mathbb{P}(\tilde{A}_{i_k})
    \end{IEEEeqnarray*}
    
    
\end{proof}
\begin{remark}
    \begin{itemize}
        \item Es ist $\Omega = \Omega_1\times \ldots\times \Omega_n$
        \item Eigentlich müssen wir $(\Omega_i, \mathcal{F}_i, \mathbb{P}_i)$ als entsprechende Wahrscheinlichkeitsräume betrachten, wir unterdrücken aber oft die Notation $\mathcal{F}_i, \mathbb{P}_i$
    \end{itemize}
Im Allgemeinen setzen wir 
\[
\mathcal{F} = \mathcal{F}_1 \otimes \ldots\otimes \mathcal{F}_n
.\] 
wobei $\mathcal{F}$ dann die $\sigma$-Algebra ist, die von  $A_1\times \ldots\times A_n$ mit $A_i \in \mathcal{F}_i$ erzeugt ist. Im Spezialfall $F_i = \mathcal{P}(\Omega_i)$ ergibt sich insbesondere wieder der uns bekannte Fall $\mathcal{F} = \mathcal{P}(\Omega)$. \\
Für Produktmodelle erhalten wir also $\Omega = \Omega_1\times \ldots\times \Omega_n$, $\mathcal{F} = \mathcal{F}_1 \otimes  \ldots \otimes  \mathcal{F}_n$ sowie $\mathbb{P} = \mathbb{P}_1 \otimes  \ldots \otimes  \mathbb{P}_n$. \\
Beachte, dass $\mathcal{F}_1 \otimes  \ldots \otimes  \mathcal{F}_n \neq  \mathcal{F}_1 \times  \ldots \times  \mathcal{F}_n$ im Allgemeinen, wie das folgende Beispiel zeigt.
\end{remark}
\begin{example}
    Im Fall $n=2$ ergibt sich Beispielsweise folgende Situation:
     \begin{tikzpicture}
         \draw[->] (-0.5,0) -- (5,0);
         \draw[->] (0,-0.5) -- (0,5);
         \coordinate (A1) at (2,3);
         \coordinate (A2) at (4,4.5);
         \draw (A1) -- (A1) -| (A2) node[anchor = west] {$A_1\times A_2$} -- (A2) -- (A2) -| (A1) -- (A1);
         \coordinate (B1) at (1,1);
         \coordinate (B2) at (3,4);
         \draw (B1) node [anchor = north] {$B_1\times B_2$}-- (B1) -| (B2)  -- (B2) -- (B2) -| (B1) -- (B1);
    \end{tikzpicture}
\end{example}


